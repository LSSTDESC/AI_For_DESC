% !TEX root = main_edited.tex

\documentclass[modern]{lsstdescnote_customized}
\usepackage{lsstdesc_macros}
\usepackage{xcolor}
\usepackage{soul}
\newcommand{\themekey}[1]{\textbf{\textcolor{Ink}{#1}}}
\newcommand{\themebullet}{\noindent\textcolor{ThemeBlue}{\textbf{·}}\hspace{4pt}}
\definecolor{DESCred}{rgb}{0.63,0.00,0.20}
\hypersetup{colorlinks=true,breaklinks=true,
  citecolor=DESCred,filecolor=DESCred,linkcolor=DESCred,urlcolor=DESCred}
  
% ====== PREAMBLE SNIPPET ======
% Dependencies
\usepackage[dvipsnames,table]{xcolor}
\usepackage[most]{tcolorbox}
\tcbuselibrary{skins,breakable}

\usepackage{booktabs}   % nicer horizontal rules
\usepackage{tabularx}   % auto-width table with X columns
\usepackage{array}      % extended column definitions
\usepackage{ragged2e}   % better ragged-right in X columns

\usepackage{hyperref}
\usepackage[acronym]{glossaries}
% \newglossary[mlg]{methods}{mls}{mln}{AI/ML Methods}
% \newglossary[clg]{challenges}{cls}{cln}{Cross-cutting Challenges}
\makenoidxglossaries
% methods_challenges.tex
%
% Assumes in main preamble:
%   \usepackage{xcolor}
%   \usepackage[acronym]{glossaries}
%   \makeglossaries
%
% Then include this file with:
%   \input{methods_challenges}
%
% In the body, use:
%   \meth{sbi}, \challenge{uq-calibration}, ...
% and print the indices with e.g.:
%   \printglossary[type=methods,title={AI/ML Methods Index}]
%   \printglossary[type=challenges,title={Cross-cutting Challenges Index}]

% ------------------------------------------------------------------
% Glossary types
% ------------------------------------------------------------------

\newglossary[mlg]{methods}{mls}{mln}{AI/ML Methods}
\newglossary[clg]{challenges}{cls}{cln}{Cross-cutting Challenges}

% Macros to invoke entries in the text
\newcommand{\meth}[1]{\gls{#1}}
\newcommand{\challenge}[1]{\gls{#1}}

% ------------------------------------------------------------------
% Color code for METHOD families (Cool/Professional - Anchored on Rhino)
% ------------------------------------------------------------------
% F1: Probabilistic & Bayesian Inference         -> aiMethBayes (Rhino - User Anchor)
% F2: Generative Modeling                        -> aiMethGen (Muted Purple)
% F3: Deep Learning Architectures                -> aiMethDeep (Muted Cyan)
% F4: Physics-Informed & Differentiable          -> aiMethPhys (Muted Green)
% F5: Representation & Discovery                 -> aiMethRepr (Muted Teal)
% F6: Data Processing & Vision                   -> aiMethVis (Steel Blue)

\definecolor{aiMethBayes}{HTML}{263654} % Rhino (User Anchor)
\definecolor{aiMethGen}{HTML}{5D4C76}   % Muted Purple
\definecolor{aiMethDeep}{HTML}{4A7A8C}  % Muted Cyan
\definecolor{aiMethPhys}{HTML}{5C7A63}  % Muted Green
\definecolor{aiMethRepr}{HTML}{3E8E7E}  % Muted Teal
\definecolor{aiMethVis}{HTML}{5B7C99}   % Steel Blue

\newcommand{\aimethbayes}[1]{\textcolor{aiMethBayes}{\textsf{#1}}}
\newcommand{\aimethgen}[1]{\textcolor{aiMethGen}{\textsf{#1}}}
\newcommand{\aimethdeep}[1]{\textcolor{aiMethDeep}{\textsf{#1}}}
\newcommand{\aimethphys}[1]{\textcolor{aiMethPhys}{\textsf{#1}}}
\newcommand{\aimethrepr}[1]{\textcolor{aiMethRepr}{\textsf{#1}}}
\newcommand{\aimethvis}[1]{\textcolor{aiMethVis}{\textsf{#1}}}

% ------------------------------------------------------------------
% Color code for CHALLENGE families (5 categories)
% ------------------------------------------------------------------
% C1: Covariate Shifts                           -> aiChCovariate
% C2: Uncertainty Quantification                 -> aiChUQ
% C3: Scalability                                -> aiChScale
% C4: Data Sparsity & Rare Events                -> aiChSparsity
% C5: Metrics & Evaluation                       -> aiChMetrics

\definecolor{aiChCovariate}{HTML}{C2814C}  % Tussock (warm orange)
\definecolor{aiChUQ}{HTML}{92402D}         % Mule Fawn (deep red-brown)
\definecolor{aiChScale}{HTML}{7D7068}      % Dark Pumice/Taupe
\definecolor{aiChSparsity}{HTML}{5D4C76}   % Muted Purple
\definecolor{aiChMetrics}{HTML}{C06C84}    % Muted Pink

\newcommand{\aichcovariate}[1]{\textcolor{aiChCovariate}{\textsf{#1}}}
\newcommand{\aichuq}[1]{\textcolor{aiChUQ}{\textsf{#1}}}
\newcommand{\aichscale}[1]{\textcolor{aiChScale}{\textsf{#1}}}
\newcommand{\aichsparsity}[1]{\textcolor{aiChSparsity}{\textsf{#1}}}
\newcommand{\aichmetrics}[1]{\textcolor{aiChMetrics}{\textsf{#1}}}

% ------------------------------------------------------------------
% METHOD ENTRIES (IDs used with \meth{...})
% ------------------------------------------------------------------

% F1: Probabilistic & Bayesian Inference

\newglossaryentry{sbi}{%
  type=methods,
  name={\aimethbayes{Simulation-based inference (SBI)}},
  text={\aimethbayes{SBI}},
  description={Implicit-likelihood Bayesian inference using forward simulations and neural density estimators}
}

\newglossaryentry{neural-density-estimation}{%
  type=methods,
  name={\aimethbayes{Neural density estimation}},
  text={\aimethbayes{Neural density estimation}},
  description={Neural networks trained to approximate probability densities, often used within SBI}
}

\newglossaryentry{npe}{%
  type=methods,
  name={\aimethbayes{Neural posterior estimation (NPE)}},
  text={\aimethbayes{NPE}},
  description={Direct neural approximation to the posterior distribution over parameters given simulated data}
}

\newglossaryentry{hierarchical-bayes}{%
  type=methods,
  name={\aimethbayes{Bayesian hierarchical modeling}},
  text={\aimethbayes{Hierarchical Bayes}},
  description={Hierarchical probabilistic models that share information across objects or populations}
}

\newglossaryentry{variational-inference}{%
  type=methods,
  name={\aimethbayes{Variational inference (VI)}},
  text={\aimethbayes{VI}},
  description={Optimization-based approach to approximating posterior distributions}
}

\newglossaryentry{ensembles}{%
  type=methods,
  name={\aimethbayes{Deep ensembles}},
  text={\aimethbayes{Ensembles}},
  description={Collections of models combined for improved performance and uncertainty quantification}
}

\newglossaryentry{gaussian-process}{%
  type=methods,
  name={\aimethbayes{Gaussian processes}},
  text={\aimethbayes{Gaussian processes}},
  description={Non-parametric Bayesian regression and emulation for scalar or functional outputs}
}

% F2: Generative Modeling

\newglossaryentry{diffusion-model}{%
  type=methods,
  name={\aimethgen{Diffusion / score-based models}},
  text={\aimethgen{Diffusion models}},
  description={Generative models that iteratively denoise data from a noise process, often used for images and fields}
}

\newglossaryentry{normalizing-flow}{%
  type=methods,
  name={\aimethgen{Normalizing flows}},
  text={\aimethgen{Normalizing flows}},
  description={Invertible neural networks used as flexible density models in likelihoods, posteriors, or simulators}
}

\newglossaryentry{vae}{%
  type=methods,
  name={\aimethgen{Variational autoencoders (VAEs)}},
  text={\aimethgen{VAEs}},
  description={Latent-variable generative models trained via variational inference}
}

\newglossaryentry{emulator}{%
  type=methods,
  name={\aimethgen{Emulators}},
  text={\aimethgen{Emulators}},
  description={Surrogate models (often GP- or NN-based) that approximate expensive simulations or likelihoods}
}

\newglossaryentry{neural-surrogate}{%
  type=methods,
  name={\aimethgen{Neural surrogates}},
  text={\aimethgen{Neural surrogates}},
  description={Neural network models trained to mimic the input--output behaviour of complex physical simulations}
}

% F3: Deep Learning Architectures

\newglossaryentry{cnn}{%
  type=methods,
  name={\aimethdeep{Convolutional neural networks (CNNs)}},
  text={\aimethdeep{CNNs}},
  description={Convolution-based neural networks for image-like data, widely used for classification and regression}
}

\newglossaryentry{transformer}{%
  type=methods,
  name={\aimethdeep{Transformers}},
  text={\aimethdeep{Transformers}},
  description={Attention-based neural architectures for sequences, time series, or generic sets of tokens (including Vision Transformers)}
}

\newglossaryentry{rnn}{%
  type=methods,
  name={\aimethdeep{Recurrent neural networks (RNNs)}},
  text={\aimethdeep{RNNs}},
  description={Sequence models (e.g.\ LSTMs, GRUs) for time series and light curves}
}

\newglossaryentry{gnn}{%
  type=methods,
  name={\aimethdeep{Graph neural networks (GNNs)}},
  text={\aimethdeep{GNNs}},
  description={Neural networks designed to operate on graph-structured data, such as cosmic webs or halo catalogs}
}

\newglossaryentry{deep-network}{%
  type=methods,
  name={\aimethdeep{Deep neural networks}},
  text={\aimethdeep{Deep networks}},
  description={Generic deep learning architectures not otherwise categorized}
}

% F4: Physics-Informed & Differentiable

\newglossaryentry{differentiable-programming}{%
  type=methods,
  name={\aimethphys{Differentiable programming}},
  text={\aimethphys{Differentiable programming}},
  description={Formulation of simulations and models as differentiable programs amenable to gradient-based inference}
}

\newglossaryentry{physics-informed}{%
  type=methods,
  name={\aimethphys{Physics-informed ML}},
  text={\aimethphys{Physics-informed ML}},
  description={Machine learning models that incorporate physical laws or constraints (e.g., PINNs, Hamiltonian NN)}
}

\newglossaryentry{symbolic-regression}{%
  type=methods,
  name={\aimethphys{Symbolic regression}},
  text={\aimethphys{Symbolic regression}},
  description={Learning analytic mathematical expressions that describe data or physical relationships}
}

% F5: Representation & Discovery

\newglossaryentry{neural-compression}{%
  type=methods,
  name={\aimethrepr{Neural compression}},
  text={\aimethrepr{Neural compression}},
  description={Learned low-dimensional representations (summary statistics or latents) of complex data}
}

\newglossaryentry{self-supervised}{%
  type=methods,
  name={\aimethrepr{Self-supervised learning}},
  text={\aimethrepr{Self-supervised learning}},
  description={Learning representations from unlabeled data by solving pretext tasks (e.g., masking, contrastive learning)}
}

\newglossaryentry{som}{%
  type=methods,
  name={\aimethrepr{Self-Organizing Maps (SOMs)}},
  text={\aimethrepr{SOMs}},
  description={Topology-preserving maps used for exploratory analysis and domain coverage characterization}
}

\newglossaryentry{anomaly-detection}{%
  type=methods,
  name={\aimethrepr{Anomaly detection}},
  text={\aimethrepr{Anomaly detection}},
  description={Identifying rare or out-of-distribution examples in data, crucial for new physics discovery}
}

\newglossaryentry{active-learning}{%
  type=methods,
  name={\aimethrepr{Active learning}},
  text={\aimethrepr{Active learning}},
  description={Strategies that adaptively select the most informative data points for labeling or follow-up}
}

% F6: Data Processing & Vision

\newglossaryentry{instance-segmentation}{%
  type=methods,
  name={\aimethvis{Instance segmentation}},
  text={\aimethvis{Instance segmentation}},
  description={Pixel-level segmentation of individual objects, relevant for deblending crowded scenes}
}

\newglossaryentry{object-detection}{%
  type=methods,
  name={\aimethvis{Object detection}},
  text={\aimethvis{Object detection}},
  description={Identifying and localizing objects in images (e.g., YOLO)}
}

\newglossaryentry{deblending}{%
  type=methods,
  name={\aimethvis{Deblending algorithms}},
  text={\aimethvis{Deblending}},
  description={Separating overlapping light profiles from multiple sources}
}

% ------------------------------------------------------------------
% CHALLENGE ENTRIES (IDs used with \challenge{...})
% ------------------------------------------------------------------

% C1: Covariate Shifts

\newglossaryentry{covariate-shift}{%
  type=challenges,
  name={\aichcovariate{Covariate Shifts}},
  text={\aichcovariate{Covariate shifts}},
  description={Distribution mismatches between training and target data (e.g.\ spectroscopic selection bias, sim-to-real gaps, model misspecification). Addressed in \autoref{sec4:model-misspec}, \autoref{sec4:hybrid-gen-phys}, and \autoref{sec:discovery}}
}

% C2: Uncertainty Quantification

\newglossaryentry{uq}{%
  type=challenges,
  name={\aichuq{Uncertainty Quantification}},
  text={\aichuq{UQ}},
  description={Obtaining well-calibrated posteriors and propagating uncertainties to cosmological constraints. Addressed in \autoref{sec4:Bayes}, \autoref{sec4:sbi}, and \autoref{sec4:validation}}
}

% C3: Scalability

\newglossaryentry{scalability}{%
  type=challenges,
  name={\aichscale{Scalability}},
  text={\aichscale{Scalability}},
  description={Handling LSST-scale data volumes, real-time alert processing, and high-dimensional inference. Addressed in \autoref{sec4:Bayes}, \autoref{sec4:sbi}, and \autoref{sec4:hybrid-gen-phys}}
}

% C4: Data Sparsity & Rare Events

\newglossaryentry{data-sparsity}{%
  type=challenges,
  name={\aichsparsity{Data Sparsity \& Rare Events}},
  text={\aichsparsity{Data sparsity}},
  description={Limited labeled samples, rare transients, class imbalance, and challenging edge cases like blending. Addressed in \autoref{sec:discovery}}
}

% C5: Metrics & Evaluation

\newglossaryentry{metrics}{%
  type=challenges,
  name={\aichmetrics{Metrics \& Evaluation}},
  text={\aichmetrics{Metrics}},
  description={Task-relevant metrics, validation frameworks, benchmarking, and stress tests for DESC science. Addressed in \autoref{sec4:validation} and \autoref{sec4:physics-constraints}}
}


% % Glossary wrappers
% \newcommand{\meth}[1]{\gls{#1}}
% \newcommand{\challenge}[1]{\gls{#1}}

% Softer color palette
\definecolor{ThemeBlue}{HTML}{2563EB}      % Modern blue
\definecolor{AccentBlue}{HTML}{DBEAFE}     % Light blue background
\definecolor{ThemeMint}{HTML}{10B981}      % Fresh green
\definecolor{AccentMint}{HTML}{D1FAE5}     % Light green background
\definecolor{DESCRed}{HTML}{a71437}
\definecolor{ThemeRed}{HTML}{c24b60}
\definecolor{Ink}{HTML}{1F2937}
\colorlet{InkMuted}{black!60}

\tcbset{
  before skip=12pt, after skip=12pt,
  boxsep=0pt, left=12pt, right=12pt, top=10pt, bottom=10pt,
  breakable, enhanced,
}

% Option 1: Minimal left accent bar (clean and modern)
\newtcolorbox{ThemeBoxA}[1][]{
  colback=white,
  colframe=ThemeRed,
  boxrule=1pt,
  leftrule=0pt,
  borderline west={3pt}{0pt}{DESCRed},
  fonttitle=\sffamily\bfseries, 
  coltitle=DESCRed,
  title={#1},
  arc=2pt,
}

% ============================================
% Collaboration Macros
% ============================================

% Toggle to show/hide section instructions
% Set to 'true' to show instructions, 'false' to hide
\newif\ifshowinstructions
\showinstructionstrue  % Change to \showinstructionsfalse to hide

% Section instructions macro
\newcommand{\sectioninstr}[1]{%
  \ifshowinstructions%
    {\color{blue!70!black}\small\textit{Instructions: #1}}%
  \fi%
}

% Section coordinator macro - shows who is responsible for the section
\newcommand{\coordinator}[1]{%
  \ifshowinstructions%
    {\color{purple!70!black}\small\textbf{Coordinator: #1}}%
  \fi%
}

% Author color macros - customize colors and names as needed
\newcommand{\eiffl}[1]{\textcolor{red}{#1}}
\newcommand{\authortwo}[1]{\textcolor{blue}{#1}}
\newcommand{\authorthree}[1]{\textcolor{green!50!black}{#1}}
\newcommand{\authorfour}[1]{\textcolor{purple}{#1}}
\newcommand{\authorfive}[1]{\textcolor{orange!80!black}{#1}}

\renewcommand{\sectionautorefname}{Section}
\renewcommand{\subsectionautorefname}{Section}
\renewcommand{\subsubsectionautorefname}{Section}

% Comments and notes
\newcommand{\todo}[1]{{\color{red}[TODO: #1]}}
\newcommand{\note}[1]{{\color{orange!80!black}[NOTE: #1]}}
\newcommand{\authornote}[2]{{\color{blue!70!black}\textbf{[#1: #2]}}}

\shorttitle{AI/ML for DESC}
\shortauthors{LSST~DESC}

\begin{document}

\title{Opportunities in AI/ML for the Rubin LSST\\Dark Energy Science Collaboration}
% \collaboration{The LSST Dark Energy Science Collaboration (LSST DESC)}
%\collaboration{\color{red}\textbf{Please explicitly validate your contribution information in the following document.} \\ \color{red}\textbf{It is our source of truth for authorship:}\\ \url{ https://tinyurl.com/ai-for-desc}.}

\collaboration{Version 1.0 -- January 2026}

\author{The LSST Dark Energy Science Collaboration (DESC)}
\noaffiliation

\author[0000-0002-5592-023X]{Eric Aubourg}
\affiliation{Université Paris Cité, CNRS, CEA, Astroparticule et Cosmologie,
F-75013 Paris, France}

\author[0000-0001-8868-0810]{Camille Avestruz}
\affiliation{Department of Physics, University of Michigan, Ann Arbor, MI 48109, USA}
\affiliation{Leinweber Institute of Theoretical Physics, University of Michigan, Ann Arbor, MI 48109, USA}

\author[0000-0001-7774-2246]{Matthew R. Becker}
\affiliation{Argonne National Laboratory, 9700 South Cass Avenue, Lemont, IL 60439, USA}

\author[0000-0002-7342-087X]{Biswajit Biswas}
\affiliation{Argonne National Laboratory, 9700 South Cass Avenue, Lemont, IL 60439, USA}

\author[0000-0002-5741-7195]{Rahul Biswas}
\affiliation{Independent}

\author[0000-0002-9836-603X]{Adam S. Bolton}
\affiliation{SLAC National Accelerator Laboratory, Menlo Park, CA 94025, USA}

\author[0000-0003-4383-2969]{Clecio R. Bom}
\affiliation{Centro Brasileiro de Pesquisas Físicas, Rio de Janeiro, Brazil}

\author[0009-0006-5127-7394]{Raphaël Bonnet-Guerrini}
\affiliation{Department of Computer Science, University of Milan, Milan, Italy}

\author[0000-0001-7387-2633]{Alexandre Boucaud}
\affiliation{Université Paris Cité, CNRS, Astroparticule et Cosmologie,
F-75013 Paris, France}

\author[0000-0002-1590-6927]{Jean-Eric Campagne}
\affiliation{Université Paris-Saclay, CNRS/IN2P3, IJCLab,  91405 Orsay, France}

\author[0000-0002-7887-0896]{Chihway Chang}
\affiliation{Department of Astronomy and Astrophysics, University of Chicago, Chicago, IL 60637, USA}
\affiliation{Kavli Institute for Cosmological Physics, University of Chicago, Chicago, IL 60637, USA}
\affiliation{NSF-Simons AI Institute for the Sky (SkAI), 172 E. Chestnut St., Chicago, IL 60611, USA}

\author[0000-0003-1281-7192]{Aleksandra \'Ciprijanovi\'c}
\affiliation{Fermi National Accelerator Laboratory, P.O. Box 500, Batavia, IL 60510, USA}
\affiliation{Department of Astronomy and Astrophysics, University of Chicago, Chicago, IL 60637, USA}
\affiliation{NSF-Simons AI Institute for the Sky (SkAI), 172 E. Chestnut St., Chicago, IL 60611, USA}

\author[0000-0001-9022-4232]{Johann Cohen-Tanugi}
\affiliation{Universit\'e Clermont-Auvergne, CNRS, LPCA, 63000 Clermont-Ferrand, France}

\author[0000-0002-8262-2924]{Michael W. Coughlin}
\affil{School of Physics and Astronomy, University of Minnesota, Minneapolis, MN 55455, USA}

\author[0000-0002-2495-3514]{John Franklin Crenshaw}
\affiliation{Kavli Institute for Particle Astrophysics and Cosmology, Stanford University, Stanford, CA  94305, USA}
\affiliation{Department of Physics, Stanford University, 382 Via Pueblo Mall, Stanford, CA 94305, USA}
\affiliation{SLAC National Accelerator Laboratory, Menlo Park, CA 94025, USA}


\author[0000-0002-7566-0412]{Juan C. Cuevas-Tello}
\affiliation{Engineering Faculty, Universidad Autonoma de San Luis Potosi, Zona Universitaria, San Luis Potosi, 78290, Mexico}

\author[0000-0001-8318-6813]{Juan de Vicente}
\affiliation{Centro de Investigaciones Energ\'eticas, Medioambientales y Tecnol\'ogicas (CIEMAT), Madrid, Spain}

\author[0000-0002-5296-4720]{Seth W. Digel}
\affiliation{SLAC National Accelerator Laboratory, Menlo Park, CA 94025, USA}
\affiliation{Kavli Institute for Particle Astrophysics and Cosmology, Stanford University, Stanford, CA 94305, USA}

\author[0000-0002-4773-1463]{Steven Dillmann}
\affiliation{Department of Physics, Stanford University, 382 Via Pueblo Mall, Stanford, CA 94305, USA}
\affiliation{SLAC National Accelerator Laboratory, Menlo Park, CA 94025, USA}

\author[0000-0001-8251-933X]{Alex Drlica-Wagner}
\affiliation{Fermi National Accelerator Laboratory, P.O. Box 500, Batavia, IL 60510, USA}
\affiliation{Department of Astronomy and Astrophysics, University of Chicago, Chicago, IL 60637, USA}
\affiliation{Kavli Institute of Cosmological Physics, University of Chicago, Chicago, IL 60637, USA}
\affiliation{NSF-Simons AI Institute for the Sky (SkAI), 172 E. Chestnut St., Chicago, IL 60611, USA}

\author[0000-0001-5717-2688]{Sydney Erickson}
\affiliation{Department of Physics, Stanford University, 382 Via Pueblo Mall, Stanford, CA 94305, USA}
\affiliation{SLAC National Accelerator Laboratory, Menlo Park, CA 94025, USA}

\author[0000-0003-4906-8447]{Alexander~T.~Gagliano}
\affiliation{The NSF AI Institute for Artificial Intelligence and Fundamental Interactions}
\affiliation{Center for Astrophysics \textbar{} Harvard \& Smithsonian, 60
Garden Street, Cambridge, MA 02138, USA}
\affiliation{Department of Physics and Kavli Institute for Astrophysics and Space Research, Massachusetts Institute of Technology, Cambridge, MA 02139, USA}

\author[0000-0002-7950-6076]{Christos Georgiou}
\affiliation{Institut de Física d'Altes Energies (IFAE), The Barcelona Institute of Science and Technology, Campus UAB, 08193 Bellaterra (Barcelona), Spain}

\author[0000-0002-2525-9647]{Aritra Ghosh}
\affil{Department of Astronomy \& DiRAC Institute, University of Washington, Seattle, WA 98195, USA}

\author[0000-0002-6741-983X]{Matthew Grayling}
\affiliation{Institute of Astronomy and Kavli Institute for Cosmology, University of Cambridge, Madingley Road, Cambridge, CB3 0HA, UK}

\author[0000-0003-3255-7340]{Kirill A. Grishin}
\affiliation{Université Paris Cité, CNRS, Astroparticule et Cosmologie, F-75013 Paris, France}

\author[0000-0003-1586-2773]{Alan Heavens}
\affiliation{Imperial Centre for Inference and Cosmology (ICIC), Imperial College London, Blackett Laboratory, Prince Consort Road, London SW7 2AZ, UK}

% \author[0000-0003-1468-8232]{Katrin Heitmann}
% \affiliation{Argonne National Laboratory, 9700 South Cass Avenue, Lemont, IL 60439}

\author[0000-0002-1496-6514]{Lindsay R. House}
\affiliation{NSF-Simons AI Institute for the Sky (SkAI), 172 E. Chestnut St., Chicago, IL 60611, USA}
\affiliation{Data Science Institute, The University of Chicago, Chicago, IL 60615, USA}

\author[0000-0002-6024-466X]{Mustapha Ishak}
\affiliation{Department of Physics, The University of Texas at Dallas, Richardson, TX 75080, USA}

% \author[0000-0002-8896-6496]{Christian Kragh Jespersen}
% \affiliation{Princeton}

\author[0009-0001-6501-4564]{Wassim Kabalan}
\affiliation{Université Paris Cité, CNRS, Astroparticule et Cosmologie,
F-75013 Paris, France}

\author[0000-0001-8783-6529]{Arun Kannawadi}
\affiliation{Department of Physics, Duke University, Durham, NC 27708, USA}

\author[0000-0001-7956-0542]{François Lanusse}
\affiliation{Université Paris-Saclay, Université Paris Cité, CEA, CNRS, AIM, F-91191 Gif-sur-Yvette, France}

% \author[0000-0001-5541-2887]{Pierre-François Léget}
% \affiliation{Princeton University}

\author[0000-0002-7810-6134]{C. Danielle Leonard}
\affiliation{School of Mathematics, Statistics and Physics, Newcastle University, Newcastle upon Tyne, NE1 7RU, United Kingdom }

\author[0000-0003-2221-8281]{Michelle Lochner}
\affiliation{Department of Physics and Astronomy, University of the Western Cape, Bellville, Cape Town, 7535, South Africa}

% \author[0000-0003-4692-4607]{Thomas Loredo}
% \affiliation{Cornell University}

% \author[0000-0003-2271-1527]{Rachel Mandelbaum}
% \affiliation{Carnegie Mellon University}

\author[0000-0002-1200-0820]{Yao-Yuan Mao}
\affiliation{Department of Physics and Astronomy, University of Utah, Salt Lake City, UT 84112, USA}

\author[0000-0002-8873-5065]{Peter Melchior}
\affiliation{Department of Astrophysical Sciences, Princeton University, Peyton Hall, Princeton, NJ 08544, USA}

% \author[0000-0002-6313-4597]{Ismael Mendoza}
% \affiliation{University of Maryland}

\author[0009-0005-7923-054X]{Grant Merz}
\affiliation{Department of Astronomy, University of Illinois Urbana Champaign, 1002 W. Green St., Urbana, IL, 61801, USA}

\author[0000-0001-7051-497X]{Martin Millon}
\affiliation{Institute for Particle Physics and Astrophysics, ETH Zürich, Wolfgang-Pauli-Strasse 27, CH-8093 Zurich, Switzerland}

\author[0000-0001-8211-8608]{Anais Möller}
\affiliation{Swinburne University of Technology, Hawthorn, Victoria 3122, Australia}

\author[0000-0001-6022-0484]{Gautham Narayan}
\affiliation{Department of Astronomy, University of Illinois Urbana Champaign, 1002 W. Green St., Urbana, IL, 61801, USA}
\affiliation{NSF-Simons AI Institute for the Sky (SkAI), 172 E. Chestnut St., Chicago, IL 60611, USA}

\author[0000-0002-0963-7310]{Yuuki Omori}
\affiliation{Department of Astronomy and Astrophysics, University of Chicago, Chicago, IL 60637, USA}
\affiliation{Kavli Institute for Cosmological Physics, University of Chicago, Chicago, IL 60637, USA}
\affiliation{NSF-Simons AI Institute for the Sky (SkAI), 172 E. Chestnut St., Chicago, IL 60611, USA}

\author[0000-0002-2519-584X]{Hiranya Peiris}
\affiliation{Institute of Astronomy and Kavli Institute for Cosmology, University of Cambridge, Madingley Road, Cambridge, CB3 0HA, UK}

\author[0000-0003-3544-3939]{Laurence Perreault-Levasseur}
\affiliation{D\'epartement de Physique, Universit\'e de Montr\'eal, 1375 Avenue Th\'er\`ese-Lavoie-Roux, Montr\'eal, QC, H2V 0B3, Canada}
\affiliation{Ciela - Montr\'eal Institute for Astrophysical Data Analysis and Machine Learning, Montréal, QC H2V 0B3, Canada}
\affiliation{Mila - Quebec Artificial Intelligence Institute, Montréal, QC H2S 3H1, Canada}

\author[0000-0002-2598-0514]{Andrés~A.~Plazas~Malagón}
\affiliation{Kavli Institute for Particle Astrophysics and Cosmology, Stanford University, Stanford, CA 94305, USA}
\affiliation{SLAC National Accelerator Laboratory, Menlo Park, CA 94025, USA}
\affiliation{Department of Astrophysical Sciences, Princeton University, Peyton Hall, Princeton, NJ 08544, USA}

% \author[0000-0002-7599-966X]{Natalia Porqueres}
% \affiliation{CEA Saclay}

\author[0000-0001-7772-0346]{Nesar Ramachandra}
\affiliation{Argonne National Laboratory, 9700 South Cass Avenue, Lemont, IL 60439, USA}

\author[0000-0002-0978-5612]{Benjamin Remy}
\affiliation{Department of Astronomy and Astrophysics, University of Chicago, Chicago, IL 60637, USA}
\affiliation{NSF-Simons AI Institute for the Sky (SkAI), 172 E. Chestnut St., Chicago, IL 60611, USA}

\author[0000-0002-9641-4552]{Cécile Roucelle}
\affiliation{Université Paris Cité, CNRS, Astroparticule et Cosmologie,
F-75013 Paris, France}

\author[0000-0003-2497-6334]{Stefan Schuldt}
\affiliation{Dipartimento di Fisica, Universit\`a  degli Studi di Milano, via Celoria 16, I-20133 Milano, Italy}
\affiliation{Finnish Centre for Astronomy with ESO (FINCA), University of Turku, FI-20014 Turku, Finland}
\affiliation{Department of Physics, P.O. Box 64, University of Helsinki, FI-00014
Helsinki, Finland}
\affiliation{INAF - IASF Milano, via A. Corti 12, I-20133 Milano, Italy}

% \author[0000-0002-0517-7943]{Stephen Serjeant}
% \affiliation{The Open University}

\author[0000-0002-1831-1953]{Ignacio Sevilla-Noarbe}
\affiliation{Centro de Investigaciones Energ\'eticas, Medioambientales y Tecnol\'ogicas (CIEMAT), Madrid, Spain}


\author[0009-0009-1590-2318]{Ved G. Shah}
\affiliation{Department of Physics and Astronomy, Northwestern University, Evanston, IL, USA}
\affiliation{Center for Interdisciplinary Exploration and Research in Astrophysics, Northwestern University, Evanston, IL, USA}
\affiliation{NSF-Simons AI Institute for the Sky (SkAI), 172 E. Chestnut St., Chicago, IL 60611, USA}

\author[0000-0003-2539-8206]{Tjitske Starkenburg}
\affiliation{Department of Physics and Astronomy, Northwestern University, Evanston, IL, USA}
\affiliation{Center for Interdisciplinary Exploration and Research in Astrophysics, Northwestern University, Evanston, IL, USA}
\affiliation{NSF-Simons AI Institute for the Sky (SkAI), 172 E. Chestnut St., Chicago, IL 60611, USA}

\author[0009-0005-6323-0457]{Stephen Thorp}
\affiliation{Institute of Astronomy and Kavli Institute for Cosmology, University of Cambridge, Madingley Road, Cambridge, CB3 0HA, UK}

\author[0000-0002-8313-7875]{Laura Toribio San Cipriano}
\affiliation{Centro de Investigaciones Energ\'eticas, Medioambientales y Tecnol\'ogicas (CIEMAT), Madrid, Spain}

\author[0000-0003-3520-2406]{Tilman Tröster}
\affiliation{Institute for Particle Physics and Astrophysics, ETH Zürich, Wolfgang-Pauli-Strasse 27, CH-8093 Zurich, Switzerland}

\author[0000-0002-3415-0707]{Roberto Trotta}
\affiliation{Theoretical and Scientific Data Science, International School for Advanced Study, Via Bonomea 265, I-34136 Trieste, Italy}
\affiliation{Astrophysics Group, Blackett Lab, Imperial College London, Prince Consort Road, London SW7 2AZ, UK }

\author[0000-0001-8638-2780]{Padma Venkatraman}
\affiliation{Department of Astronomy, University of Illinois Urbana Champaign, 1002 W. Green St., Urbana, IL, 61801, USA}

\author[0000-0002-4186-6164]{Amanda Wasserman}
\affiliation{Department of Astronomy, University of Illinois Urbana Champaign, 1002 W. Green St., Urbana, IL, 61801, USA}
\affiliation{NSF-Simons AI Institute for the Sky (SkAI), 172 E. Chestnut St., Chicago, IL 60611, USA}

\author[0000-0001-5535-0452]{Tim White}
\affiliation{Department of Statistics, University of Michigan, Ann Arbor, MI 48109, USA}

\author[0000-0001-6002-5128]{Justine Zeghal}
\affiliation{D\'epartement de Physique, Universit\'e de Montr\'eal, 1375 Avenue Th\'er\`ese-Lavoie-Roux, Montr\'eal, QC, H2V 0B3, Canada}
\affiliation{Mila - Quebec Artificial Intelligence Institute, Montréal, QC H2S 3H1, Canada}

\author[0000-0002-5596-198X]{Tianqing Zhang}
\affiliation{Department of Physics and Astronomy and PITT PACC, University of Pittsburgh, Pittsburgh, PA 15260, USA}

\author[0000-0001-5969-4631]{Yuanyuan Zhang}
\affiliation{NSF NOIRLab, 950 N. Cherry Ave., Tucson, AZ 85719, USA}


\begin{abstract}
\small
The Vera C. Rubin Observatory's Legacy Survey of Space and Time (LSST) will produce unprecedented volumes of heterogeneous astronomical data, including images, catalogs, and alerts, that challenge traditional analysis pipelines. The mission of the LSST Dark Energy Science Collaboration (DESC) is to convert these data into robust constraints on the dark sector, which requires methods that are statistically powerful, scalable, and operationally reliable. Recent advances in Machine Learning (ML) (data-driven models learning mappings) and Artificial Intelligence (AI) (Large Language Models and agentic systems that orchestrate code and complex analyses) show great promise. However, their scientific utility for precision cosmology hinges on trustworthy uncertainty quantification and reproducible integration within DESC workflows. Without these elements, these methods cannot meet the stringent requirements for cosmological inference. This white paper surveys current capabilities and future opportunities for AI/ML across DESC and outlines the methodological and infrastructure requirements for their successful, reliable deployment. We review the intersection of AI/ML with key DESC science analyses, and detail the methodological research in ML motivated by DESC's needs, focusing on Bayesian inference, uncertainty quantification, physics-informed approaches, and novelty detection. With an eye on emerging techniques we also explore specifically the potential of the latest Foundation Model methodologies and agentic AI systems. Finally, we discuss critical infrastructure requirements for software, computing, and data necessary to the successful deployment of these new methodologies, and consider associated risks as well as opportunities for broader coordination with external actors.
\end{abstract}

\maketitle

%  just temporary command for table of content to have a general view/idea of the current structure of the paper (sections and subsections) as it develops.
% Make TOC more compact
{
\setlength{\parskip}{0pt}
\makeatletter
\renewcommand{\l@section}[2]{%
  \ifnum \c@tocdepth >\z@
    \addpenalty\@secpenalty
    \addvspace{2pt plus 1pt}%
    \setlength\@tempdima{1.5em}%
    \begingroup
      \parindent \z@ \rightskip \@pnumwidth
      \parfillskip -\@pnumwidth
      \leavevmode \bfseries
      \advance\leftskip\@tempdima
      \hskip -\leftskip
      #1\nobreak\hfil \nobreak\hb@xt@\@pnumwidth{\hss #2}\par
    \endgroup
  \fi}
\renewcommand{\l@subsection}[2]{%
  \ifnum \c@tocdepth >\@ne
    \addpenalty\@secpenalty
    \addvspace{1pt}%
    \setlength\@tempdima{2.3em}%
    \begingroup
      \parindent \z@ \rightskip \@pnumwidth
      \parfillskip -\@pnumwidth
      \leavevmode
      \advance\leftskip\@tempdima
      \hskip -\leftskip
      #1\nobreak\hfil \nobreak\hb@xt@\@pnumwidth{\hss #2}\par
    \endgroup
  \fi}
\renewcommand{\l@subsubsection}[2]{%
  \ifnum \c@tocdepth >2
    \addvspace{0.5pt}%
    \setlength\@tempdima{3.8em}%
    \begingroup
      \parindent \z@ \rightskip \@pnumwidth
      \parfillskip -\@pnumwidth
      \leavevmode
      \advance\leftskip\@tempdima
      \hskip -\leftskip
      #1\nobreak\hfil \nobreak\hb@xt@\@pnumwidth{\hss #2}\par
    \endgroup
  \fi}
\makeatother
\tableofcontents
}


% ============================================
% Document Sections
% ============================================
% Each section is in a separate file in the sections/ directory
% This makes collaboration easier and keeps the main file clean

\newpage
\section{Executive Summary}
\label{sec:exec_summary}

%This executive summary distills the key recommendations, opportunities, and strategic takeaways emerging from this document that can help inform a coherent AI/ML strategy for DESC. We start by establishing below the \textbf{core driving principles} that guide our strategic thinking:

%\begin{itemize}
 %   \item \textbf{Facilitating the delivery of DESC science} through the integration of AI/ML tools, while \textbf{fulfilling the stringent requirements} of precision cosmology.

 %   \item Building and maintaining a \textbf{sustainable AI/ML ecosystem} required for the collaborative development, validation, and deployment \textbf{of production-grade AI/ML tools}.

 %   \item Seeking out opportunities to leverage \textbf{DESC’s unique position} as a large international collaboration using community-accessible data and pursuing extremely demanding scientific objectives, to \textbf{establish DESC as a pioneer in the development of robust AI/ML practices for fundamental physics}.

 %   \item Integrating AI/ML into DESC in ways that \textbf{preserve and amplify the human-centric nature of research}, strengthen collaboration quality, and \textbf{keep contributors’ work at the center}.
%\end{itemize}

The \acrlong{lsst} \acrlong{desc} (\acrshort{lsst} \acrshort{desc}) is an international collaboration whose mission is to measure the cosmic expansion history and the growth of structure using data from the Vera C. Rubin Observatory, thereby constraining the nature of dark energy and dark matter. Achieving these science goals requires jointly analyzing multiple cosmological probes---weak and strong gravitational lensing, galaxy clusters, Type Ia supernovae, and large-scale structure---each presenting distinct analysis challenges at LSST's unprecedented data volumes. Extracting robust cosmological constraints demands methods that deliver trustworthy uncertainty quantification, remain robust to systematic effects and model misspecification, and scale to the full petabyte-scale survey. These requirements motivate the integration of \acrfull{ai} and \acrfull{ml} into DESC pipelines. DESC's combination of community-accessible data, mature simulation infrastructure, and rigorous scientific standards makes the collaboration an excellent testbed for developing robust AI/ML practices for fundamental physics. 

Recognizing this situation, the DESC formed the \textit{AI for DESC Task Force} with the following charge:
\begin{itemize}
    \item Catalog the AI/ML needs, use cases, and projects in DESC.
    \item Identify current gaps in the adoption of AI/ML methodologies by leveraging expert domain knowledge.
    \item Identify the computational resources, storage, data access, and human research and managerial time needed to take full advantage of AI/ML-related opportunities.
    \item Identify either qualitatively or quantitatively the projected gains in DESC’s science that would result from pursuing AI/ML-related opportunities.
\end{itemize}
The response to the task force charge is presented in this white paper. It demonstrates the breadth and importance of AI and ML research within DESC, and highlights the challenges and promising pathways for future work. %Throughout this work, we adopt the following definitions for AI/ML:
% \begin{itemize}
% \item \textbf{\acrfull{ai}} refers to the field of computer science concerned with building systems capable of performing tasks that typically require human intelligence, including reasoning, perception, learning, and decision-making.
% \item \textbf{\acrfull{ml}} is a subfield of AI in which algorithms learn patterns and relationships from data to make predictions or decisions. This includes both classical methods (e.g., random forests) and deep learning methods (multi-layer neural networks).
% \end{itemize}
In this Executive Summary, we synthesize key recommendations and opportunities into a coherent AI/ML strategy for DESC. Three core principles guide this strategy: 

\begin{itemize}
    \item \textbf{AI/ML tools should be carefully integrated into DESC pipelines} to facilitate scientific analyses while fulfilling the stringent requirements of precision cosmology and preserving scientific accountability and transparency.

    \item A \textbf{durable AI/ML ecosystem should be built} within DESC and maintained over the survey lifetime, for the collaborative development, validation, and deployment of production-grade AI/ML tools.

    \item AI/ML must be integrated into DESC in ways that \textbf{preserve and support the human-centric nature of research}, improve accessibility, strengthen collaboration quality, and amplify rather than supplant members' contributions.
\end{itemize}

We have defined a series of recommendations (R) and opportunities (O) in several key areas within DESC in support of these principles. \textit{Recommendations} are actions that the collaboration should undertake to meet its scientific requirements and ensure robust integration of AI and ML into DESC pipelines. \textit{Opportunities} indicate areas where DESC can extend beyond its requirements and assume a leadership role, influence broader community standards, or explore higher-risk, higher-reward efforts. We summarize these below, along with references throughout the paper where they are discussed.

\paragraph{Advancing Key Methodological Research Directions} Challenges such as uncertainty quantification, robustness to model misspecification, and novelty detection recur across DESC science cases. Progress on these foundational challenges will benefit all probes and merit dedicated effort.

\begin{itemize}
    \item \textbf{R1: Prioritize Fundamental Methodological Research.}  Foster collaboration-wide research in several critical areas: quantification of systematic and statistical uncertainties, simulation-based inference robustness, physics-informed modeling (hybrid generative-physical architectures), validation of neural posteriors, and novelty detection. Progress on these fundamental challenges will have an outsized impact across many DESC science cases. (\autoref{sec3:use_case_for_aiml}, \autoref{sec4:aiml_research})

    \item \textbf{O1: Methodological Leadership in Trustworthy AI.} The challenges DESC faces (robust inference under misspecification, calibrated uncertainty quantification at scale, physics-informed learning) are frontier problems in machine learning broadly, creating natural opportunities to attract specialist collaborators and position DESC as a leader in trustworthy AI for fundamental science. (\autoref{sec4:aiml_research}, \autoref{sec7:broader_coordination})
    
    \item \textbf{O2: DESC Simulation Assets as Community Benchmarks.} DESC's combination of petabyte-scale community data, stringent scientific requirements, and rich simulation assets---e.g.\ the \acrfull{plasticc}, \acrfull{elasticc}, and \acrfull{cosmodc2}---makes it an ideal testbed for pioneering robust AI/ML practices. Benchmarks and governance standards developed here can become reference implementations for fundamental physics, and attract colleagues in mathematics and computer science who see DESC's frontier challenges as compelling application areas for new methods. (\autoref{sec3:use_case_for_aiml}, \autoref{sec4:aiml_research}, \autoref{sec7:broader_coordination})
\end{itemize}


\paragraph{Foundation Models} Foundation models, which produce generalizable representations of large-scale, heterogeneous, and multi-modal datasets, are transforming AI capabilities. DESC must develop both the infrastructure to deploy them and the benchmarks to validate them for precision cosmology.
\begin{itemize}

\item \textbf{R2: Develop Shared Foundation Model Infrastructure.} Build a shared foundation model backbone for DESC, consistent across data modalities and of production-grade quality, and served behind stable APIs. (\autoref{sec5:emerging_tech}, \autoref{sec6:infra_requirements})

\item \textbf{R3: Establish DESC-specific Foundation Model Validation Standards.} Create benchmarks that go beyond industry practice: uncertainty calibration, robustness to systematics, sensitivity to training biases, stress tests under distribution shift (temporal, spatial, cross-survey). 
Develop astronomy-specific interpretability tools to verify the physically meaningful structure preserved within model representations. (\autoref{sec5:emerging_tech}, \autoref{sec:aiml_risks})

\item \textbf{O3: Leadership of Rubin-wide Development of Foundation Models.} DESC could play a central role in coordinating foundation model development across Rubin Science Collaborations and \acrfull{lincc}, leveraging distributed computing resources from universities to \acrfull{doe} or \acrfull{eurohpc} facilities. (\autoref{sec5:emerging_tech}, \autoref{sec7:broader_coordination})
\end{itemize}

\paragraph{Large Language Models \& Agentic AI} \Acrfullpl{llm} and agentic AI offer avenues to accelerate research and lower the barrier to entry for complex cosmological analyses in DESC. Harnessing this potential responsibly will require thoughtful governance and rigorous validation frameworks.

\begin{itemize}

  \item \textbf{R4: Establish Governance for LLMs and Agentic Systems.} Coordinate DESC-wide 
  activities involving LLMs and agents, establish best practices including evaluation, review, and tiger-team review of pilot studies. Include critical discussions of the technology's limitations and effects on human researchers, with input from experts across domains. Engage with Rubin Data Management to ensure that agentic AI can interface with data products effectively and reliably. (\autoref{sec5:emerging_tech}, \autoref{sec:aiml_risks})

  \item \textbf{R5: Build Natural Language Interfaces to DESC Resources.} Develop \acrfull{rag}-based interfaces to DESC documentation, simulations, and data products, lowering onboarding barriers and democratizing access to complex pipelines. (\autoref{sec5:emerging_tech}, 
  \autoref{sec6:infra_requirements})

  \item \textbf{O4: Pioneering Agentic AI for Scientific Rigor and Reproducibility.} An important application of this work could be 
  ``DESC research agents" that automate execution, documentation, and validation of analyses 
  against standardized benchmarks, coupling these systems to clear governance and tiger-team review procedures so that agentic workflows enhance transparency, provenance, and trust in DESC results. (\autoref{sec5:emerging_tech}, \autoref{sec:aiml_risks})
\end{itemize}


\paragraph{Infrastructure \& Software} DESC has a mature ecosystem of cosmological analysis pipelines. Building on this foundation, strategic development of AI software stacks, differentiable programming, and computing infrastructure can act as multipliers that benefit all science cases.

\begin{itemize}
  \item \textbf{R6: Establish a Durable AI Software Stack.} Adopt a coherent set of frameworks, tooling, and model export standards. The stack should be portable across DESC computational facilities, sustainable over the 10-year survey, and prioritize open governance to avoid proprietary lock-in. (\autoref{sec6:infra_requirements})

  \item \textbf{R7: Develop a Differentiable Programming Ecosystem.} Adoption of an interoperable differentiable programming ecosystem (e.g. based on JAX) will act as a multiplier, simultaneously enabling gradient-based sampling, GPU acceleration, hybrid physics-ML models, and end-to-end optimization across DESC pipelines. (\autoref{sec3:use_case_for_aiml}, \autoref{sec4:aiml_research}, \autoref{sec6:infra_requirements})

\item \textbf{R8: Secure Access to Emerging Computing Infrastructure.} Significant new AI-oriented 
computing is becoming available: DOE infrastructure such as the \acrfull{amsc}; the \acrfull{idac} 
network; and EuroHPC systems such as Leonardo in Italy, \acrfull{lumi} in Finland, and the \acrfull{jupiter} exascale system in Germany. 
DESC should engage early to shape these resources for cosmology and secure allocations for 
foundation model training at scales infeasible on current systems. 
(\autoref{sec6:infra_requirements}, \autoref{sec7:broader_coordination})
\end{itemize}

\paragraph{Organizational Structure \& Governance} The DESC is organized into computing, technical, and analysis \acrfullpl{wg}\footnote{\url{https://lsstdesc.org/pages/organization.html}}, with analysis working groups primarily aligned with key cosmological probes. Effective AI/ML integration across these groups requires consistent coordination mechanisms and clear standards for development, validation, and deployment.

\begin{itemize}
  \item \textbf{R9: Develop DESC-wide AI/ML Coordination Mechanisms.} Establish structures (e.g., standing working group, cross-WG task forces, regular interchange meetings) to share methodological innovations across probes, tackle common challenges collectively, and minimize duplication. Facilitate rapid dissemination through workshops, tutorials, and methodological discussions. (\autoref{sec3:use_case_for_aiml})

  \item \textbf{R10: Develop AI/ML Best Practice Guidelines.} Create guidelines to help DESC members develop robust AI/ML analyses, covering topics such as reproducibility, provenance tracking, validation checks, and comprehensive benchmarking—particularly for foundation models and other shared deliverables whose broad applicability demands thorough vetting before widespread adoption. (\autoref{sec3:use_case_for_aiml}, 
  \autoref{sec4:aiml_research}, \autoref{sec5:emerging_tech})
\end{itemize}

\paragraph{Human Capital \& Sustainability} The promise of AI/ML for accelerating cosmology with LSST will not be realized without training and support of DESC members. Sustainable adoption of AI/ML also requires attention to the growing computational demands and resulting footprint these methods entail. 
\begin{itemize}
  \item \textbf{R11: Focus on AI/ML for Augmenting Rather Than Replacing Understanding.} DESC must strengthen and maintain the technical literacy of the collaboration in AI/ML applications as tools for science rather than supplanting understanding. (\autoref{sec:aiml_risks})
  
  \item \textbf{R12: Track and Optimize Resource Footprint.} DESC should develop tools for monitoring and optimizing computational resource usage of AI/ML models, enabling the collaboration to maximize scientific productivity and make informed decisions about resource allocation and environmental impact. (\autoref{sec:aiml_risks})
\end{itemize}

\paragraph{External Coordination \& Partnerships} DESC operates within a rich ecosystem of other Rubin science collaborations, AI institutes, cosmology experiments, and alert brokers that filter streaming Rubin data. Deliberate coordination between these groups will amplify impact and avoid duplicated effort.
\begin{itemize}

  \item \textbf{R13: Coordinate Across Science Collaborations.} Partner with other LSST collaborations and other cosmology experiments. The former include 
  the \acrfull{issc}, \acrfull{tvs}, \acrfull{slsc}, \acrfull{agnsc}, and \acrfull{galsc}.  The latter include the \acrfull{desi}, the \acrfull{4most}, the \acrfull{esa} \textit{Euclid} Mission science teams, and the \textit{Nancy Grace Roman Space Telescope} science collaborations.  Areas of coordination should include methodological development, time-series and broker stress-testing, deblending/morphology benchmarks, and sharing tools and best practices. (\autoref{sec7:broader_coordination})

  \item \textbf{R14: Engage with AI Institutes and Networks.} \acrfull{nsf}--Simons AI Institutes 
  (with explicit LSST/cosmology themes), and European networks such as the \acrfull{eucaif}\footnote{\url{https://eucaif.org/}} and \acrfull{ellis}\footnote{\url{https://ellis.eu}}, are natural partners. 
  Build systematic engagement through co-funded postdocs, shared workshops, joint proposals, 
  and benchmark datasets. These efforts would connect DESC to the broader AI-for-science ecosystem. 
  (\autoref{sec7:broader_coordination})

    \item \textbf{R15: Develop the Human-Machine Interface.} Develop close connections between DESC, other LSST science collaborations, in-kind follow-up programs, alert broker teams, LSST data management, and citizen scientists, to facilitate active learning for classification, anomaly detection, and human-in-the-loop interpretability. (\autoref{sec4:aiml_research}, \autoref{sec7:broader_coordination})

  \item \textbf{O5: DESC Integration with the Broker Ecosystem.} DESC members are embedded in 
  all seven Rubin Community Broker teams—tight coordination gives direct leverage over SN~Ia sample 
  purity, selection effects, and host-galaxy priors, plus an on-ramp from research prototypes 
  to community-facing services. (\autoref{sec3:use_case_for_aiml}, \autoref{sec7:broader_coordination})
\end{itemize}
  
%Implementing these recommendations and capitalizing on these opportunities would position DESC to fully exploit LSST's statistical power for cosmology while uncovering unexpected phenomena in the largest optical astronomical dataset ever collected. This will require sustained investment in researchers who bridge domain science and AI/ML methodology. It would also advance AI practices and benefits for fundamental science more broadly.

Implementing these recommendations and capitalizing on these opportunities would position DESC to fully exploit LSST's statistical power for cosmology while uncovering unexpected phenomena in the largest optical astronomical dataset ever collected. This will require sustained investment in researchers who bridge domain science and AI/ML methodology. Such investment would benefit not only DESC, but the broader effort to advance AI as a tool for fundamental scientific discovery.

% \hline

% \paragraph{Key Opportunities and Recommendations} 

% \begin{itemize}

  % \item \textbf{R1: Develop DESC-wide AI/ML best practices.} Extend Publication Policy with an AI/ML checklist, consolidate a small set of supported AI/ML stacks, establish DESC-wide model registry with automated robustness tests, establish a standard of full reproducibility for AI/ML results. 

  % \item \textbf{R2: Establish governance for LLMs and agentic systems.} Coordinate DESC-wide activities involving LLM/agents, establish best practices and evaluation/review/tiger-team review of pilot studies, include critical discussions of the limits of the technology and its effects on human researchers, involving experts in other domains. DESC should engage with the LSST Data Management team to allow developed agentic AI to interface with data products effectively and reliably.

  % \item \textbf{R3: Champion Differentiable Programming and Hybridization of Physical models with Generative Modeling.} Promote differentiable programming and hybrid physics-ML models that embed cosmological theory and simulators directly into AI architectures.

  % \item \textbf{R4: Build Strategic Methodological Partnerships in AI and Computer Science.} Establish long-term collaborations with computer science departments and AI institutes focused on the foundational challenges identified in this document (robust uncertainty quantification, model misspecification and covariate shift, validation of neural posteriors and generative models).

   % \item \textbf{Standardization and benchmarking.} Rigorous protocols for benchmarking, validation, domain adaptation, and uncertainty quantification must be established as standard practice for all DESC machine learning projects and outlined within AI/ML policy guidelines. Furthermore, cross-working group deliverables that use machine learning, such as foundation models and simulations, should be supported by comprehensive benchmarks that are solicited from DESC members and represent a broad array of science cases.
% \item \textbf{Developing the human-machine interface.} Close connections should be developed between DESC, other LSST science collaborations, in-kind follow-up programmes, alert broker teams, LSST data management and citizen scientists, to facilitate active learning for classification and anomaly detection.
% \item \textbf{Compute and workforce needs.} Investment in computational infrastructure and specialized technical expertise must be secured to keep pace with the escalating demand for AI research and deployment within DESC.

  % \item \textbf{O1: Lead Rubin-wide Development of Foundation Models.} DESC can play a key role in coordinating various actors, ranging from other Rubin LSST Science Collaborations to a wide range of academic institutions and AI institutes, on the development of frontier AI models for the Rubin Community.

  % \item \textbf{O2: Pioneer Agentic AI for Scientific Rigor and Reproducibility.} Develop “DESC research agents” that automate the execution, documentation, and validation of analyses against standardized DESC benchmarks, coupling these systems to clear governance and tiger-team review procedures so that agentic workflows enhance transparency, provenance, and trust in DESC results.

  % \item \textbf{O3: Establish DESC as a Hub for AI-for-Cosmology Training.} Create a sustained Rubin-focused AI/ML training program (schools, tutorials, and mentoring) built around DESC’s open-source software and challenging datasets, in partnership with leading AI institutes and industry, to train the next generation of researchers in trustworthy, physics-informed AI for cosmology.

 % \item \textbf{Strategic leadership in inference.} DESC should capitalize on its leadership position in simulation-based inference (SBI) and broader Bayesian methodologies, with a specific focus on mitigating model misspecification. To sustain this expertise, collaboration meetings and Dark Energy Schools should be used to facilitate knowledge transfer between research groups.
% \end{itemize}
\newpage
\section{Introduction}

The Vera C. Rubin Observatory's \acrshort{lsst} will produce unprecedented volumes of heterogeneous data (images, catalogs, alerts) whose full scientific exploitation demands continued methodological innovation. The mission of the \acrshort{desc} is to convert these data into robust constraints on the dark sector by jointly measuring the cosmic expansion history and the growth of structure, thereby shedding much-needed light on dark energy, dark matter, and possible deviations from general relativity. Delivering on these objectives requires methods that are statistically powerful, scalable, and operationally reliable. Recent advances in \acrshort{ai} and \acrshort{ml} show great promise for critical data analysis roles but still need to meet stringent requirements to be truly useful. \acrlong{ai} (\acrshort{ai}) refers broadly to systems capable of performing tasks that typically require human intelligence, including reasoning, perception, learning, and decision-making. \acrlong{ml} (\acrshort{ml}) is a subfield of AI in which algorithms learn patterns and relationships from data to make predictions or decisions, encompassing both classical methods (e.g., random forests, Gaussian processes) and deep learning (multi-layer neural networks). In the DESC context, ML methods learn mappings between variables (e.g., photometry to redshift, galaxy fields to cosmological parameters) and can be deployed at multiple stages of analysis. Their scientific utility, however, hinges on trustworthy uncertainty quantification and reproducible integration within DESC workflows; without these elements, ML methods cannot meet the stringent requirements of cosmological inference. In parallel, we use AI to denote systems capable of complex cognitive tasks---such as reasoning, knowledge synthesis, and natural language understanding---that can potentially orchestrate tools, generate code, and reshape scientific workflows. As with ML, turning recent AI advances into reliable accelerators of discovery remains an open problem that requires careful evaluation and governance.

The strategic question for DESC is therefore how to develop and integrate AI/ML \textit{the right way}, so that these approaches become dependable components of LSST-era analyses. Intrinsically, this question is relevant to a broad range of scientific endeavors and collaborations. Still, DESC is well positioned to pioneer robust AI/ML practices for fundamental physics as an international collaboration that works with community-accessible data, has a strong open-source culture, and pursues extremely demanding scientific objectives. In this white paper, we set out a strategic framework for how DESC should organize its AI/ML efforts, prioritize methodological investments, and respond effectively to new opportunities arising from rapid AI/ML progress. This paper is structured around four interconnected perspectives on AI/ML within DESC, each building on the previous to articulate a comprehensive strategy:

\paragraph{\autoref{sec3:use_case_for_aiml}: The Current Landscape: ML Across DESC Science} Machine learning is not a future aspiration for DESC: it is already deeply embedded in current science workflows. Here, we survey how ML methodologies intersect with DESC's primary cosmological probes: strong and weak gravitational lensing, galaxy clusters, \acrfull{snia} cosmology, large-scale structure, as well as cross-cutting analysis components including simulations, theory and modeling, deblending, \acrfull{photoz} estimation, and shape measurement. This inventory reveals a striking pattern: the same core methodologies (e.g. simulation based inference, differentiable programming, deep learning) appear repeatedly across disparate science cases, while the same fundamental challenges (e.g. uncertainty quantification, robustness to covariate shift and to model misspecification) represent concrete challenges across multiple working groups (see \autoref{fig:chord-diagram}).


\begin{figure}
    \centering
    \includegraphics[width=\linewidth]{figures/chord-diagram}
    \caption{Transversal connections between DESC science applications (left), AI/ML methodologies (top), and shared challenges (right), as surfaced by \autoref{sec3:use_case_for_aiml}. The recurring appearance of the same methods and challenges across disparate science cases motivates collaboration-wide coordination of AI/ML efforts rather than siloed development within individual working groups. An interactive version of this diagram is available at \url{https://codepen.io/EiffL/full/ByKxMaa}}
    \label{fig:chord-diagram}
\end{figure}


\paragraph{\autoref{sec4:aiml_research}: Lifting the Limits of ML: Methodological Research Priorities} Building on the challenges surfaced in Section~3, we identify the key methodological research axes where targeted investment can lift current limitations and enable ML methods to meet the precision and reliability standards demanded by LSST-era cosmology. We organize these priorities around several interconnected themes:
\begin{itemize}
    \item \textbf{Bayesian inference and \acrfull{acr:uq}:} Developing fast and scalable inference techniques that may unlock promising high-dimensional hierarchical models. Improving methods for reliably estimating uncertainty arising from limited training data and model limitations.
    \item \textbf{\Acrfull{acr:sbi} and model misspecification:} Advancing \acrfull{nde} techniques, optimal summarization methods, and diagnostics for detecting and mitigating the biases introduced when training simulations imperfectly represent real observations or when training datasets fail to capture the full diversity of LSST data.
    \item \textbf{Physics-informed approaches:} Hybridizing explicit physical models with flexible generative components (flows, diffusion models) and advancing differentiable programming frameworks that embed cosmological theory directly into ML architectures, ensuring that learned components remain interpretable, physically consistent, and robust to extrapolation.
    \item \textbf{Novelty detection and discovery:} Developing representation learning and active human--AI collaboration frameworks capable of identifying rare, previously unmodeled phenomena in LSST's vast data volumes.
\end{itemize}
These research directions are not purely academic exercises; they directly address the technical barriers that limit the deployment of ML at scale for DESC's most ambitious analyses. We articulate not only what needs to be developed, but why these specific advances matter for cosmological inference and how DESC can contribute to the broader AI research ecosystem by presenting demanding, scientifically motivated benchmarks.

\paragraph{\autoref{sec5:emerging_tech}: Looking Forward: Foundation models and Agentic AI}
While Sections~3 and~4 focus on current ML applications and their refinement, Section~5 adopts a forward-looking perspective, examining how two emerging AI paradigms (\textit{data foundation models} and \textit{LLM-based agentic systems}) have the potential to reshape large sections of DESC workflows in ways that go qualitatively beyond incremental improvements to existing methods.
\begin{itemize}
    \item \textbf{\Acrfullpl{fm}}, trained at scale on heterogeneous data modalities (images, spectra, time series, catalogs), offer the promise of \textit{reusable representations} that can be rapidly fine-tuned or directly deployed across a wide range of downstream tasks (classification, regression, anomaly detection, simulation-based inference) without retraining from scratch for each application. For DESC, this paradigm shift could enable unified, survey-scale feature extractors that serve as common backbones for weak lensing, photometric redshifts, transient classification, and more, dramatically reducing duplication of effort while ensuring cross-probe consistency. However, realizing this vision requires careful attention to uncertainty propagation, robustness to distribution shifts, architectural choices suited to astronomical data, and rigorous, community-governed benchmarking to ensure that foundation models meet DESC's validation standards.

    \item \textbf{LLM-driven agentic systems} are rapidly evolving from research prototypes into tools capable of orchestrating complex scientific workflows: querying databases, generating and executing code, synthesizing literature, and autonomously iterating on analyses. These systems offer tantalizing possibilities for accelerating exploratory research, onboarding new collaboration members, and scaling human oversight across large analysis campaigns. Yet they also introduce new risks: biased recommendations, irreproducible results, and erosion of scientific understanding if deployed without governance. Section~5 outlines both the transformative potential and the implementation requirements (provenance tracking, human-in-the-loop validation, benchmark design, and clear policies on data rights and model transparency) necessary to integrate agentic AI into DESC in ways that guarantee scientific rigor.
\end{itemize}
The forward-looking stance of Section~5 is deliberate: DESC must not only respond to today's ML methods but actively shape the trajectory of emerging AI technologies by setting clear scientific requirements, contributing demanding use cases to the broader AI research community, and pioneering governance frameworks that other collaborations can learn from.

\paragraph{\autoref{sec6:infra_requirements}, \ref{sec7:broader_coordination}, \ref{sec:aiml_risks}: Operationalizing AI/ML: Infrastructure, Coordination, and Risk Management}
Even the most sophisticated AI/ML methods will have limited impact if they cannot be reliably deployed, maintained, and integrated into DESC's production pipelines. The final sections of this white paper address the operational foundations required to translate research prototypes into dependable cosmological infrastructure. Section~6 details the \textit{infrastructure requirements} across software, computing, and data:
\begin{itemize}
    \item \textbf{Software:} Establishing a robust, collaboration-endorsed AI software stack (frameworks, experiment tracking, model registries, continuous integration/continuous deployment for models) that ensures reproducibility, portability across DESC computing facilities, and long-term sustainability over the LSST decade. This includes strategies for integrating AI components into DESC analysis pipelines and for managing the rapidly evolving ecosystem of LLMs and agentic frameworks.
    \item \textbf{Computing:} Securing the \acrfull{gpu} allocations, distributed training capabilities, and co-located data access necessary for foundation model development, large-scale simulation-based inference campaigns, and real-time alert stream processing. This involves strategic coordination with national labs such as the \acrfull{alcf} and \acrfull{olcf}, emerging initiatives (\acrshort{amsc}, \acrshortpl{hpdf}), and international partners (\acrshort{eurohpc}, \acrshortpl{idac}).
    \item \textbf{Data:} Ensuring that LSST data products, multi-survey training datasets, and simulation outputs are accessible, well-documented, and equipped with the interfaces---e.g., \acrfullpl{api}, streaming services, tokenization strategies---required for efficient AI/ML workflows. This includes establishing shared repositories, benchmark datasets, and provenance standards.
\end{itemize}

Section~7 broadens the scope to examine \textit{opportunities for coordination beyond DESC}: with the other Rubin LSST Science Collaborations, Stage-IV experiments (in particular \acrshort{desi}, \acrshort{4most}, Roman, Euclid), AI institutes (e.g., NSF--Simons institutes, CosmicAI), European networks (e.g., \acrshort{eucaif}, \acrshort{ellis}), and the Rubin alert broker ecosystem. These partnerships offer opportunities for shared training data, cross-survey foundation models, joint benchmark development, and access to specialized compute resources and expertise. DESC is uniquely positioned to act as both a consumer and a driver of AI methodologies within this broader ecosystem, articulating the demanding requirements of precision cosmology while contributing validated methods and datasets that benefit the wider community.

Finally, Section~8 confronts the \textit{risks and challenges} inherent in DESC's increasing reliance on AI/ML: model miscalibration, opaque failure modes, reproducibility challenges, data governance complexities, and the potential erosion of human scientific understanding. We outline concrete mitigation strategies (validation protocols, redundancy in critical analyses, provenance tracking, training programs, and governance structures) that apply the same rigor to AI components as to any other element of the cosmological inference pipeline.


\paragraph{\autoref{sec:conclusion}: Conclusion} Viewed as a whole, this paper shows that AI/ML is already central to DESC science (\autoref{sec3:use_case_for_aiml}), but unlocking its full potential requires targeted methodological research (\autoref{sec4:aiml_research}), proactive engagement with emerging technologies (\autoref{sec5:emerging_tech}), and robust operational foundations (\autoref{sec6:infra_requirements}--\ref{sec:aiml_risks}). The transversality of methods and challenges across DESC working groups demands deliberate coordination to prevent fragmented effort and ensure that best practices, validated tools, and lessons learned propagate rapidly throughout the collaboration. By articulating this vision (grounded in current capabilities, guided by research priorities, forward-looking in its engagement with foundation models and agentic AI, and operationally realistic about infrastructure and risk) DESC can position itself not only to meet its own science goals but to pioneer robust AI/ML practices for fundamental physics that serve as a model for the broader community. 

\newpage
\section{Intersection of AI/ML Techniques with DESC Science}
\label{sec3:use_case_for_aiml}

The science goals of DESC place unusually stringent demands on statistical methodology. Extracting percent-level constraints on dark energy and tests of gravity from Rubin LSST data requires not only exquisite control of observational systematics, but also analysis pipelines that can efficiently exploit information distributed across billions of galaxies, multiple probes, and heterogeneous data modalities (images, catalogs, time series, simulations). AI and ML are already embedded in many of these workflows and their importance will only grow as analyses become more ambitious and data volumes increase.

In this section, we survey existing intersections of AI/ML methods with DESC science and highlight which aspects of the science case stand to benefit most from methodological advances. Across these applications, common methodological themes emerge: the need for trustworthy uncertainty quantification and robust calibration; scalability to LSST-sized datasets and high-dimensional parameter spaces; methods that can operate coherently across multiple surveys; and tight integration between learned models and physically motivated forward models. By organizing the landscape along both science goals and methodological challenges, our aim is to clarify where targeted investment in AI/ML will yield the greatest scientific returns for DESC, and to identify opportunities for cross-cutting solutions that can be shared across probes and working groups.

\subsection{Impact at the Analysis Level}

\subsubsection{Photometric redshifts} \label{sec3:photo-z}
\begin{ThemeBoxA}[]
\themebullet \themekey{Methodology.} \meth{gaussian-process}, \meth{neural-density-estimation}, \meth{som}, \meth{transformer}, \meth{hierarchical-bayes}, \meth{neural-surrogate} \\
\themebullet \themekey{Challenges.}   \challenge{uq-calibration}, \challenge{scalability},
  \challenge{covariate-shift} \\
\themebullet \themekey{Opportunities.} Multi-survey training, simulation infrastructure, hierarchical inference
\end{ThemeBoxA}

The inference of photometric redshifts (photo-$z$) represents a foundational challenge for the Vera C. Rubin Observatory's Legacy Survey of Space and Time (LSST), where the vast majority of tens of billions of detected galaxies will lack spectroscopic redshift measurements due to both observational time constraints and the intrinsic faintness of the sample. Photometric redshifts are derived by establishing empirical or physically motivated mappings between broadband photometry (including colors, magnitudes), morphology, and redshift. This process is fundamentally limited by our incomplete knowledge of galaxy spectral energy distributions, stellar population synthesis models, and dust attenuation physics. However, the accuracy and reliability of photo-$z$ estimation is critical across virtually all extragalactic LSST science cases, including weak gravitational lensing, large-scale structure, galaxy cluster cosmology, and supernova surveys. To achieve the DESC goals for constraining the dark energy equation of state, calibration of photo-z estimates must reach the $0.002 \times (1+z)$ level for the first year of LSST data \citep{v1DESC-SRD}. Achieving these benchmarks necessitates not only accurate point predictions but also \textit{well-calibrated uncertainty quantification}, motivating an emphasis on probabilistic methods.


\paragraph{Supervised Photo-z Estimation} The empirical approach to  photo-$z$ inference is to learn a mapping from observed broadband photometry or imaging to redshift, leveraging spectroscopic training samples. From catalog-level photometry, empirical regressors such as random forests TPZ~\citep{carrasco2013tpz}, gradient boosting machines FlexZBoost~\citep{izbicki2017converting} and Gaussian processest GPz ~\citep{almosallam2016gpz} have demonstrated competitive performance by constructing mappings from color-magnitude space to redshift. Neural networks, including both fully connected and specialized architectures, have proven particularly adept at capturing complex relationships in high-dimensional photometric data. On the other hand, nearest-neighbor approaches such as kNN and CMNN~\citep{graham2017photometric} take the training as a reference sample and compute the average redshift of neighbors of the target galaxy. In a similar way, DNF~\citep{de2016dnf} performs a regression on the neighborhood, which allows the construction of a local linear model for each galaxy. \\ 
Contemporary approaches have evolved from point estimators to full probabilistic models capable of capturing the full conditional distribution $p(z \mid \text{photometry})$, with neural density estimation (NDE) techniques \citep[e.g. PZFlow][]{Crenshaw2024} enabling flexible, well-calibrated redshift PDFs via maximum likelihood training. Complementing catalog-based methods, image-based inference circumvents the information bottleneck imposed by aperture photometry by operating directly on multi-band pixel data, delegating feature extraction to deep neural networks that leverage morphology and spatial structure inaccessible to catalogs. The DeepDISC framework \citep{Merz23DeepDISC,Merz25DeepDISCpz} exemplifies this approach, integrating object detection, segmentation, and redshift estimation into a unified pipeline using vision transformers (MViTv2) as the backbone feature extractor coupled with Mixture Density Networks for probabilistic PDF estimation. \\
Despite their sophistication, \textit{all supervised ML methods remain fundamentally limited by the quality and representativeness of their spectroscopic training samples}: spectroscopic incompleteness, magnitude-limited surveys, and selection biases induce systematic offsets and distortions in the learned photo-$z$ mapping, particularly at faint magnitudes and high redshifts where spectroscopic follow-up is most incomplete \citep{newman2022}. This represents the main challenge for photo-$z$ today and has motivated dedicated calibration strategies.
A further problem is the ``implicit prior'' imposed by each photo-z method \citep{schmidt2020}.
These priors, which have a large impact on photo-z estimates, are opaque and difficult to quantify, making it difficult to compare and combine photo-z posteriors provided by different methods.


\paragraph{Calibration Strategies to Account for Covariate Shifts}  Self-Organizing Maps (SOMs) have emerged as the preeminent unsupervised learning technique for diagnosing and mitigating the biases caused by covariate shifts by performing non-linear dimensionality reduction of photometric feature vectors onto a discrete two-dimensional grid. SOM-based calibration approaches, such as those deployed in DES Y3 \citep{Myles2021} and KiDS analyses \citep{vanDenBusch2022}, directly assign photometric galaxies the empirical redshift distribution of spectroscopic galaxies in their SOM cell while down-weighting or even rejecting regions of color space poorly represented in the spectroscopic catalog. More sophisticated SOM-guided data augmentation strategies selectively populate under-represented SOM cells with simulated galaxies from mock catalogs improving ML model performance where spectroscopic coverage is deficient \citep{Zhang2025}. An alternative approach to SOM for covariate shift mitigation (and without using data augmentation) is represented by stratification by the propensity score (defined as the probability of a covariate vector to be admitted as part of the training set) of both training and target data. Within each propensity score group, supervised photo-$z$ can proceed with any method of choice. This \textit{StratLearn} approach is theoretically guaranteed (under some conditions) to cancel covariate shift~\citep{Autenrieth_2023}. It has demonstrated state-of-the-art performance in the PLAsTiCC SNIa classification challenge, a factor of $\sim 2$ improvement in photo-$z$ calibration from the cosmic shear KiDS+VIKING-450 dataset~\citep{Autenrieth_2024} and a reduced fraction of catastrophic errors and one order of magnitude improvement of the bias for simulated photo-$z$ reconstruction~\citep{Moretti_2025}.

\paragraph{Hybrid Template-Based Estimators}
In contrast to empirical photo-$z$ estimators, a broad class of ``template-based'' photo-$z$ estimators attempt to circumvent the problem of covariate shift using physical models of galaxy SEDs \citep{eazy,lephare}.
These estimators trade the problem of covariate shift for the problem of model misspecification.
Hybrid methods, however, attempt to combine the strengths of empirical and template-based estimators by deriving SED templates in a physics-informed, data-driven manner \citep{budavari2000,Csabai2000}.
These models have been shown to deliver higher-quality photo-$z$ estimates than traditional template-based estimators while suffering less from covariate shift than pure empirical methods \citep{crenshaw2020,li2025}.
They do not perform as well in-distribution as pure empirical methods, however, and still rely on spectroscopic calibration sets.
It may be possible to remedy these defects by implementing hybrid, physics-informed models in deep learning frameworks to enable self-supervised learning without reliance on spectroscopic data sets \citep{2021Boone_ParSNIP}. 



\paragraph{Population-Level Hierarchical Forward Modeling} Traditional photo-$z$ workflows estimate individual galaxy redshifts and aggregate these posteriors to derive population-level quantities such as ensemble redshift distributions $n(z)$ -- a computationally expensive bottom-up approach prone to biases when combining noisy individual posteriors \citep{Leistedt2016, Malz2021, Malz2022, Alsing2023}. Population-level inference inverts this paradigm by directly targeting the population distribution $P(\boldsymbol{\theta})$ over redshift and physical galaxy parameters (stellar mass, star formation rate, metallicity) as the primary inference objective, leveraging the collective constraining power of the entire photometric dataset while naturally incorporating physical priors on galaxy evolution. These methods rely on forward modeling: generating synthetic photometry from physical parameters via stellar population synthesis (SPS; for a review, see, e.g., \citealp{Conroy:2013, Iyer:2026}) models and comparing the distribution of model photometry to observed data. Classical SPS calculations (as implemented by, e.g., the FSPS and Prospector ecosystem; \citealp{Conroy:2009, Conroy:2010, ConroyGunn:2010, Leja:2017, Johnson:2021, Wang:2023}) are computationally prohibitive for large samples, motivating neural network emulators like \href{https://github.com/justinalsing/speculator/tree/master}{Speculator} \citep{SPECULATOR} that achieve $\sim10^{3}$--$10^{4}\times$ speedups with negligible accuracy loss. The \href{https://github.com/Cosmo-Pop/pop-cosmos}{pop-cosmos} framework \citep{Alsing:2024, Thorp:2024, Thorp:2025, Deger:2025} exemplifies this approach: it defines a probability distribution over a 16-dimensional SPS parameterization using a score-based diffusion model calibrated on $\sim$420,000 galaxies from COSMOS2020 \citep{Weaver:2022} with 26-band photometry spanning deep UV to mid-IR. 
%The \href{pop-cosmos}{https://github.com/Cosmo-Pop/pop-cosmos} 
This model enables direct estimation of tomographic redshift distributions, and, when used as a data-driven prior in SED fitting, highly accurate individual galaxy redshift inference.

\paragraph{Benchmarking and Evaluation Frameworks}
As AI methods become increasingly central to cosmological analyses, it is critical to develop robust frameworks for testing and validation that ensure reproducibility and enable systematic comparison of different approaches.
For this purpose, the DESC has developed RAIL (\textit{Redshift Assessment Infrastructure Layers}), an open-source, Python-based framework to support large-scale photometric-redshift workflows for the Vera C. Rubin Observatory's Legacy Survey of Space and Time.
Although RAIL is not itself an AI algorithm, it provides a comprehensive infrastructure that
(i) supplies a unified API and modular pipeline stages to train, apply, and compare a broad range of redshift estimators (catalog-based, image-based, probabilistic),
(ii) embeds evaluation modules and metrics for both individual-galaxy redshift and ensemble PDFs ~\citep{RAIL_2025}, and
(iii) enables data challenges to test the robustness of photo-z estimators to a wide array of systmatic errors.
This standardized framework facilitates reproducible results and fair benchmarking across different methods, which is essential for validating AI techniques in preparation for LSST data.


\subsubsection{Strong Lensing}
% \note{Text written by Stefan Schuldt, with help from Clecio de Bom, Sydney Erickson, Martin Millon and Padma Venkatraman. Comments are welcome!}
\label{sec:strong_lensing}
\begin{ThemeBoxA}[]
\themebullet \themekey{Methodology.} \meth{cnn}, \meth{transformer}, \meth{sbi}, \meth{anomaly-detection}  \\
\themebullet \themekey{Challenges.} \challenge{data-sparsity}, \challenge{systematics-modeling} \\
\themebullet \themekey{Opportunities.} Multi-survey cross-matching (Roman+LSST+Euclid), population-level inference, automated discovery, subhalo constraints from anomalous flux ratios
\end{ThemeBoxA}

Strong gravitational lensing is a rare astrophysical phenomenon where the light of a distant object, the source, is deflected by the gravity of an intervening structure, the lens, forming multiple images of the background source. In galaxy-galaxy strong lensing, both the source and the lens are individual galaxies, while on larger scales, the lens could range from a group to an entire galaxy cluster. Despite their rarity, which is a result of the required near-perfect alignment of the source, lens, and observer, lensed systems are powerful cosmological probes. They offer unique opportunities to study the dark matter fraction and distribution on sub-galactic scales within the lens. Furthermore, the lensing magnification enables the study of high-redshift sources, offering insights into phenomena such as early galaxy evolution.

One particularly powerful application is time-delay cosmography, which uses strongly lensed transients to measure cosmological parameters such as the Hubble constant ($H_0$)~\citep[e.g.,][]{tdcosmo2025}. LSST is poised to revolutionize this field \citep{erickson_de_2025}. This approach yields a geometrical measurement and provides an independent method for resolving the Hubble tension, that is, the significant discrepancy between measurements of the Hubble constant from supernovae \citep[e.g., SH0ES;][]{riess22} and early-universe measurements from the Cosmic Microwave Background \citep[Planck;][]{planck20}. ML/AI models have been proposed for time-delay estimation from light curves~\cite{cuevas06,cuevas2010uncovering,otaibi16}, particularly kernel-based methods, which are the core of Support Vector Machines (SVMs). Moreover, a combination of time-delay lenses and large samples of static lenses from LSST has been demonstrated to enable a competitive dark energy measurement \citep{shajib_SL_2025}. LSST's impact on time-delay cosmography will be transformative, providing time-domain coverage for $\sim100$ more systems than current surveys \citep{abe_2025}. Beyond this, it will also serve as a unique discovery opportunity for new strong lenses, with forecasts of $170,000$ systems \citep{collett15}, two orders of magnitude more than are currently known. This large sample will not only be crucial for building the statistical power required by various science applications, but also for detecting a significant number of currently rare systems, such as double-source-plane lenses, lensed supernovae, and larger-mass-scale lenses containing multiple background sources. Finally, LSST will deliver high-quality ground-based images in six filters, enabling the analysis in multi-band imaging. 

\paragraph{Supervised Detection in the Low Data Regime} Given the rarity of strong lensing events, identifying them among billions of cutouts is inherently challenging for visual inspection in wide-field surveys. While early automated detection approaches relied on curvature-based features \citep{estrada2007systematic} and arc-characterizing descriptors \citep{de2012metodo, bom2017neural}, the advent of convolutional neural networks (CNNs) led to state-of-the-art performance in lens finding \citep{2018A&A...611A...2S, 2019MNRAS.482..807P, 2018MNRAS.473.3895L}. Building on this progress, \citet{metcalf19} launched a lens-finding challenge using Bologna Lens Factory simulations based on the Millennium data \citep{Lemson2006}, showing that LSST-like ground-based multi-band images are well suited for this task. A subsequent Euclid-like challenge produced a winning algorithm combining multi-resolution CNNs \citep{bom2022developing}, later validated on real data by \citet{melo25} using Legacy Survey and HST images to mimic LSST–Euclid synergy. Leveraging multiple networks classifications as ensemble lens classifiers showed improved results over a single network classifier \citep[e.g.][]{andika23, schuldt23b, gonzalez25}, while \cite{holloway_2024} incorporated citizen science annotations to enhance the performance. Because too few real lenses exist for supervised training, realistic mock datasets are essential. \citet{schuldt21} proposed simulating only the lensing effect on real galaxy images, now a standard practice also adopted in the ongoing LSST DESC and Strong Lensing Science Collaboration challenge (Bom et al., in prep.). In preparation for LSST, HSC data \citep{aihara18} -- with similar filters and pixel scale -- have been used to develop and compare models \citep[see e.g.,][]{shu22, andika23, canameras24, jaelani24, more24}. While early efforts focused on simple CNN or ResNet architectures, foundation models such as Zoobot \citep{ZoobotRelease2023} have recently achieved strong results on Euclid imaging \citep{walmsley25, lines25}, and will be adopted for LSST (see Sect.~\ref{sec:foundation_models}). Finally, ML methods are now expanding beyond galaxy-scale lenses to systems involving entire clusters \citep[][Bazzanini et al., in prep.]{schuldt25} and galaxy–galaxy lenses within clusters \citep{angora23}.

\paragraph{Simulation-Based Inference (SBI)} Beyond lens finding, \citet{hezaveh17} pioneered the use machine learning models to predict characteristics of strong lensing systems. Specifically, \citet{hezaveh17} showed that simple CNNs can be used to predict parameters of the lens (Einstein radius, position, ellipticity and angle) from images from the Hubble Space Telescope with a precision comparable to that of traditional methods. \citet{Perreault:2017} proposed using approximate Bayesian Neural Networks to obtain calibrated estimations of the marginal posterior of these lens parameters, an approach applied to ALMA observations in~\citet{Morningstar2018}. \citet{Legin2021, Legin2023} compared this approach to NLEs, demonstrating potential for better calibration in 2-stage SBI methods. The following years saw significant progress using HSC images to prepare for LSST \citep[e.g.,][]{pearson19, schuldt21, gentile23, schuldt23a, schuldt23b, gawade25}. Following earlier work by \citet{Park2021,Wagner-Carena2021}, \citet{erickson25} applied (sequential) Neural Posterior Estimation (NPE) within a hierarchical framework to model strongly lensed quasars, testing on real systems discovered by DES and followed-up with HST high-resolution imaging, and \citet{venkatraman_2025} applied hierarchical NPE modeling to simulated LSST i-band coadds.  Ongoing DESC work examines how modeling an uncertain sample of static galaxy-galaxy lenses with ML enables new cosmological constraints (Holloway et al. in prep.), leveraging the method demonstrated by \citep{li_gg_SL}.
%Furthermore, \citet{schuldt23b} made a first direct comparison between the model predicted quantities obtained in the classical way without machine learning using real HSC images. The classical procedure requires dedicated software that fits typically profiles to the observed image by minimizing the difference though Monte Carlo Markov Chain sampling, or similar sampling techniques. This makes it computationally very expensive and requires also significant human input time, such that it is not scalable to the full sample expected by LSST. 
Finally, \citet{filipp25} applied a sequential NPE approach to detect dark-matter clumps in strong-lensing systems, exploiting the sensitivity of lensing to total mass.
 
\paragraph{High Dimensional Inverse Problem for Lens Modeling}  The task of lens modelling, that is, predicting surface brightness of background sources and density maps of foreground lenses is, in its simplest form, a non-linear inverse problem involving a handful of parameters ($\sim 10-20$). However, as the quality and resolution of data increases, such parametric description of lensed object becomes too simplistic, and more complex parametrization become necessary to avoid biases. An example of such a parametrization that is particularly well-adapted to ML applications are pixelated images of sources surface brightness and projected densities of lenses. Traditionally, it has been difficult to characterize appropriate priors analytically on such high-dimensional spaces~\citep[see, e.g.,][]{Suyu2006, WarrenDye2005, Birrer2015, Delaunay2009, Nightingale2018}, however, recent advances in high-dimensional inference with deep learning has made progress on this front possible. \\
An initial attempt at solving the source reconstruction problem was presented in~\cite{Morningstar2019}, and extended in ~\cite{AlexAdam2023} to enable joint modeling of generalized pixellated lens densities and sources surface brightness. However, while these models provided high-fidelity MAP estimates, they lacked the crucial ability to quantify uncertainties. Approaches based on variation inference have also been proposed in~\citet{Chianese2020, Karchev2022GP, Mishra-Sharma2022}.\\
Advances of, e.g., \cite{Song_2019, Ho_2020, Song:2020, Song_2021, 2022Yang_Diffusion}, have shown that generative models used as expressive, data-driven priors are a promising alternative to address this problem. \cite{adam2022posterior, Karchev2022Diffusion} used score-based models (SBMs) as flexible priors in an explicit inference framework to produce posterior samples of background galaxy sources. In \cite{Barco2025blindinversion}, this method was extended to allow joint sampling the source and lens parameters for smooth, parametric lenses. Such methods have been shown to alleviate known biases in lens parameters induced by misspecified traditional priors, and methods have been proposed to empirically adapt initially biased SBM priors to correct for, e.g., population-level evolution of galaxy morphologies~\cite{Barco_2025}, and to empirically extend misspecified physical models~\cite{Payot2025}. More recently, \cite{RonanSLACLenses} has shown that SBM priors can be leveraged in a Gibbs sampling scheme to reanalyze HST data from SLACS lenses. Ongoing challenges include increasing the sampling efficiency of these methods to allow modelling a large fraction of the strong lenses expected with LSST.

\paragraph{Leveraging LSST's time-series data} LSST will generate an overwhelming number of transient alerts, making the discovery and characterization of strongly lensed short-lived transients (e.g., supernovae) both difficult and time-critical. Kernel-based methods and probabilistic machine learning models such as Gaussian processes will likely play a major role in time-delay inference for lensed quasars \citep[e.g.][]{cuevas06, cuevas2010uncovering, hojjati14, otaibi16, tak17} and supernovae \citep[e.g.][]{hayes24, hayes25} discovered by LSST. Temporal deep learning models will also be essential in this area. For instance, \citet{bag24} developed a model using unresolved light-curve data from difference imaging, while \citet{bag25} extended this to full multi-band time series with a 2D convolutional LSTM. \citet{huber24} instead used an LSTM to predict time delays between such transients directly from their light curves, whilst \citet{goncalves25} used an ensemble of CNNs to directly estimate $H_0$ from time series of lensed supernova images and \citet{Campeau2023} demonstrated the potential of NREs to infer $H_0$ from time delays and lens models. Beyond short-lived transients, \citet{jimenez25} modeled microlensing in lensed quasar light curves, and \citet{fagin_light_curves} introduced a Latent SDE framework to jointly model AGN variability and transfer functions, potentially extendable to lensed AGNs for joint inference of time delays and disk parameters.

\subsubsection{Weak Lensing}
\label{sec3:wlss}
\begin{ThemeBoxA}[]
\themebullet \themekey{Methodology.} \meth{sbi},  \meth{neural-compression}, \meth{differentiable-programming} \\
\themebullet \themekey{Challenges.} \challenge{systematics-modeling}, \challenge{uq-calibration},
  \challenge{scalability}\\
\themebullet \themekey{Opportunities.} Multi-resolution joint processing (Roman+LSST), probabilistic deblending, DeepDISC instance segmentation, physics-informed priors for galaxy morphology
\end{ThemeBoxA}

As light from background galaxies travels through the Universe, its path is deflected by the gravitational potential of foreground matter, inducing subtle shape distortions of observed galaxies that can be statistically measured. This effect provides a direct probe of the total matter distribution in the Universe, making weak gravitational lensing a powerful tools for constraining cosmological parameters such as the matter density $\Omega_m$, the amplitude of matter fluctuations $\sigma_8$, and the dark energy equation of state.
With its unprecedented depth, image quality, and sky coverage, LSST will provide the most precise mapping of the large-scale structure of the Universe to date. This level of precision has two key implications:
(1) It  presents a major opportunity to refine cosmological constraints, motivating the development of advanced inference methods that can fully exploit this high-quality data; (2) It demands rigorous control of systematic uncertainties to ensure unbiased cosmological constraints.

\paragraph{Systematics modeling / mitigation}
Systematic errors such as imperfect shear calibration, photometric redshift uncertainties, spatially varying selection effects, and residuals in the point spread function need to be well characterized. To date, the precision of current cosmological surveys has permitted the use of state-of-the-art prescriptions that capture the dominant effects of these systematics (e.g.\ \citealp{Weaverdyck2021,RodriguezMonroy2022}). However, as forthcoming large-scale structure and weak-lensing data achieve substantially higher statistical precision, a more accurate and detailed characterization of these systematics will be required, at a level of complexity that renders purely analytical treatments intractable.  
ML methods offer a complementary pathway by learning complex non-linear mappings between observational features (for example, properties of individual galaxies, local image quality metrics, PSF residuals, depth maps, shape measurement parameters) and the resulting systematic bias or residual error \citep{Tewes:2018she,Rezaie2020,Pujol:2020wrk}. Neural-network or other machine-learning algorithms can be trained on simulated or calibration data for which the true systematic shifts are known, and can then be tuned to predict, flag or correct for the systematic effect when applied to real survey data \citep{Fluri:2022rvb}. In doing so these methods enable a more optimal separation of systematic contamination from the cosmological signal and thus a cleaner inferred signal with higher robustness.

\paragraph{Clean catalog construction} Systematics such as intrinsic alignments \citep[e.g.][]{Mandelbaum:2006, Mandelbaum:2011, Troxel:2015, Joachimi:2015} can be mitigated by constructing clean source and/or lens catalogs from galaxy populations that are known to be less susceptible to these effects. Scalable inference of galaxy properties that correlate with quiescent populations (such as specific star formation rate, sSFR) can enable the construction of intrinsic alignment-mitigated galaxy samples. For instance, machine-learned generative priors can be leveraged (e.g.\ \href{https://github.com/Cosmo-Pop/pop-cosmos}{pop-cosmos}; \citealp{Alsing:2024, Thorp:2024, Thorp:2025, Deger:2025}) to estimate per-galaxy sSFR and construct clean catalogs of star-forming galaxies with conservative cuts based on this parameter. This approach, especially when combined with amortized neural posterior inference, is scalable to LSST-sized datasets and is expected to be superior to color-based selections, which are impacted by contamination. Moreover, generative models of the galaxy population can be applied in  a weak lensing context to directly infer the redshift distributions of source catalogs subject to tomographic binning and sample selection criteria, provided that the color--redshift relation is realistic and robust. This provides an alternative to approaches such as SOM-based calibration.

\paragraph{Neural Compression and Simulation-Based Inference}
Traditional weak-lensing analyses follow a two-step pipeline: compress high-dimensional shear or convergence fields into summary statistics, then perform Bayesian inference on these summaries to obtain posteriors over cosmological parameters. The matter power spectrum and shear–shear correlation functions remain workhorse statistics (KiDS-1000 cosmic shear: \citealt{Asgari2021}; DES Y3 cosmic shear: \citealt{Amon_2022,Secco_2022}), but in the Rubin era significant non-Gaussian information becomes available. This has motivated the use of higher-order moments such as the bispectrum and trispectrum \citep[e.g.][]{desy3_moments}, as well as peak counts \citep[e.g.][]{Marques_2024}, persistent homology \citep{prat2025}, and Minkowski functionals \citep[e.g.][]{PhysRevD.85.103513}. While powerful, these handcrafted summaries are not guaranteed to capture all cosmological information. An alternative, enabled by machine learning, is to learn approximately sufficient statistics by optimizing a neural network to compress each weak-lensing map into a low-dimensional representation. \cite{neural_summary_lanzieri_2025} benchmark different loss functions to identify those that yield near-sufficient summaries. Because the likelihood of these learned summaries is unknown, neural density estimators such as normalizing flows are then used to approximate the posterior within a SBI framework. This strategy was first demonstrated on survey data in \citep{Jeffrey:2021} and subsequently applied to recent surveys \citep[e.g.][]{jeffrey2024darkenergysurveyyear, kramsta2025}. However, although information-theoretically superior to two-point statistics, SBI methods have not yet delivered substantial gains in cosmological constraining power in practice, largely because it remains challenging to simulate cosmological fields with sufficient realism to avoid biased posteriors from model misspecification. In addition, neural summaries are notoriously difficult to interrogate: monitoring them for covariate shifts, unmodeled systematics, and failures in specific regions of data space is challenging, which in turn complicates the construction of robust null tests and diagnostic pipelines. 
 
\paragraph{Hierarchical Bayesian Field-Level Inference} With the advent of GPU-accelerated probabilistic programming, it has become feasible to model the full weak-lensing field in a hierarchical framework that links Gaussian initial conditions of the matter density to observed shear maps through an explicit forward simulation model. Proof-of-concept studies have demonstrated this approach in simplified weak-lensing settings \citep[e.g.][]{porqueres2023fieldlevelinferencecosmicshear}, showing substantial gains in constraining power relative to power-spectrum analyses. DESC members have contributed key building blocks for such end-to-end pipelines, including differentiable lensing lightcone constructions \citep{Lanzieri_2023} and accurate, differentiable ray-tracing schemes \citep{Zhou2024}. However, scaling these methods to a full LSST analysis remains extremely challenging: the survey volume and the resolution required for the forward model place stringent demands on memory, compute, and algorithmic efficiency. Ongoing work aims at lifting this bottleneck through distributed simulations over multiple GPUs \citep{Kabalan_jaxDecomp_2025}. In parallel with full forward modeling of the large-scale structure, DESC members have also explored map-based hierarchical inference using lognormal fields \citep{boruah2022mapbasedcosmologyinferencelognormal, Zhou2024Prd}, which is far less computationally demanding but whose ultimate accuracy is constrained by the limitations of the lognormal approximation.
As an alternative approach, DESC members have proposed using diffusion models to learn the forward model of the density field implicitly from simulations, combining this learned prior with an explicit likelihood to constrain against observed shear data \citep{remy2023}, enabling fast reconstruction of high-fidelity mass maps.


%\newpage
%\subsubsection{Dark Matter}
%\begin{ThemeBoxA}[]
%\themebullet \themekey{Methodology.} TODO\\
%\themebullet \themekey{Challenges.} TODO\\
%\themebullet \themekey{Opportunities.} TODO
%\end{ThemeBoxA}

%ADW: Brainstorming by ADW
%star/galaxy classification (old-school, but still important!), simulation-based inference and graph neural networks for stellar stream modeling and measurement, emulators for accelerating SSI


% \newpage
\subsubsection{Galaxy Clusters}
\begin{ThemeBoxA}[]
\themebullet \themekey{Methodology.} \meth{sbi}, \meth{object-detection}\\
\themebullet \themekey{Challenges.} \challenge{systematics-modeling}, \challenge{covariate-shift}, \challenge{uq-calibration}\\
\themebullet \themekey{Opportunities.} Combination of imaging and catalog data, Hierarchical Modeling
\end{ThemeBoxA}

% Galaxy clusters probe the densest peaks of the universe, and formed relatively late in the history of the universe.  The evolution of their number count is therefore sensitive to dark energy.  Measurements of the galaxy cluster mass function have been the traditional approach to constrain cosmology. AI can facilitate cluster-cosmology in several stages.  
Galaxy clusters trace the most massive peaks of the matter density field and form relatively late in cosmic history, making their abundance and internal properties highly sensitive to the growth of structure and to dark energy. Cosmological constraints from clusters have traditionally relied on measurements of the cluster mass function and its redshift evolution, anchored by calibrated relations between mass and observable proxies. ML methods are now entering this pipeline at multiple stages (cluster finding, mass–observable calibration, and population-level inference) offering new ways to combine imaging, catalog, and multi-wavelength data while retaining control over systematics and uncertainties.


% The first step to performing a cluster cosmology analysis is `cluster-finding'. Several non-AI algorithms exist to find galaxy clusters, including RedMaPPer \citep{rykoff2014, rykoff2016} and WAZP \citep{aguena2021} for optically selected galaxy clusters. Several deep learning-based methods for cluster finding have emerged in the SZ \citep{bonjean2020, lin2021, hurier2021, Meshcheryakov2022}, X-ray [???], and optical \citep{chan2019, grishin2023, grishin2025, tian2025}.%[Lin 2022, Grishin+23; Grishin+25; Tian+25].  
% A benefit of deep learning-based methods is that cluster-finding can operate directly in image spaces, as opposed to only the information in processed catalogs.  These models can potentially learn features in the image that are not captured in the catalog data. In one example, the You Only Look Once (YOLO; \citealp{redmon2015, redmon2016, redmon2018}) architecture was trained to identify clusters with a training sample defined by the RedMaPPer cluster catalog, but also successfully detected false negative cases, verified as true clusters in external X-ray catalogs \citep{grishin2023}. Ongoing work within DESC \citep[e.g.][]{grishin2025} will enable the application of YOLO to LSST data and joint catalogs.
\paragraph{Cluster Finding from Images and Catalogs}
The first step in cluster cosmology is robust identification of cluster candidates. Non-ML algorithms such as RedMaPPer \citep{rykoff2014, rykoff2016} and WAZP \citep{aguena2021} have been widely used to find optically selected clusters from galaxy catalogs. In parallel, deep-learning–based cluster finders have emerged in the SZ \citep{bonjean2020, lin2021, hurier2021, Meshcheryakov2022} and optical domains \citep{chan2019, grishin2023, grishin2025, tian2025}. A key advantage of these approaches is that they can operate directly on images, rather than on pre-processed catalogs, and thus potentially exploit features (e.g.\ diffuse emission, subtle color–magnitude structure, environment) that are not captured in standard catalog-level summaries. For example, a You Only Look Once (YOLO; \citealp{redmon2015, redmon2016, redmon2018}) architecture trained on RedMaPPer clusters was shown to recover not only the training sample but also previously missed systems that were later confirmed in external X-ray catalogs \citep{grishin2023}. Ongoing DESC work \citep[e.g.][]{grishin2025} aims to deploy such models on LSST imaging and joint catalogs, with particular attention to domain adaptation and calibration across surveys.


% Another stage of cluster-cosmology relies on our ability to constrain galaxy cluster masses from observables, often referred to as the ``mass-observable'' relation.  This task has also seen an emergence of deep neural network based approaches, with methodology inferring masses from X-ray signatures \citep{ntampaka2019, krippendorf2024, iqbal2025}, dynamics of cluster member galaxies \citep{ho2019, ho2021, ho2022, wangthiele2025}, and the SZ signature \citep{deandres2022}.  
% A particular contribution of photometric galaxy data to cluster-based cosmology is in the weak lensing mass estimates, which is used to anchor the mass-observable relation.  The DESC tool, Cluster Mass Modeling \citep[CLMM;][]{aguena2021clmm}, currently uses traditional inference methods with MCMC to infer weak lensing masses from radial profiles that assume an underlying mass distribution model, such as the NFW profile.  Ongoing in DESC are the development and testing of likelihood implicit approaches to infer weak lensing masses, specifically an approach called Simulation Based Inference (SBI). 
\paragraph{Mass-Observable Relations and Weak-Lensing Mass Calibration}
Cosmological analyses require accurate and precise relations between cluster mass and observables (richness, SZ signal, X-ray luminosity/temperature, velocity dispersion). Deep neural networks are being explored as flexible mass estimators across multiple wavebands, including X-ray signatures \citep{ntampaka2019, krippendorf2024, iqbal2025}, the dynamics of member galaxies \citep{ho2019, ho2021, ho2022, wangthiele2025}, and SZ measurements \citep{deandres2022}. For Rubin and DESC, photometric galaxy data contribute primarily through weak-lensing mass estimates that anchor mass–observable relations. The DESC Cluster Mass Modeling tool (CLMM; \citealp{aguena2021clmm}) currently infers weak-lensing masses from radial shear profiles using traditional likelihoods and MCMC, assuming parametric mass models such as NFW. Ongoing work within DESC explores alternative SBI approaches at this level, which can in principle incorporate more realistic shear profiles, complex noise, and selection effects without requiring an explicit closed-form likelihood.

% Simulation-based inference enables a computationally efficient approach to derive posteriors for relevant parameters, such as the cluster mass.  Computational efficiency becomes particularly important for analyses where we might want to simultaneously account for models of individual galaxy clusters and the broader galaxy cluster population (e.g.\ Bayesian Hierarchical Modeling).  Recent numerical experiments within indicate consistency between MCMC and SBI posteriors, provided that the SBI does not suffer from model misspecification more than MCMC (Gill et al., in prep.).
% SBI has also been recently used for direct cosmology parameter estimation from observed data vectors, such as from the CMB \citep{cole2022, lemos2023cmb}, shear two-point statistics \citep{kramsta2025, modi2025}, void lensing analysis \citep{su2025}, topological summaries \citep{prat2025}, and galaxy cluster number counts \citep{reza2022, reza2024}.  The latter of these is now being applied to DESC simulations, with a goal of constructing alternative approaches to cluster-based cosmology that can be incorporated into the DESC Cluster Cosmology Pipeline.  The DESC cluster cosmology analysis will need to incorporate more astrophysical effects and probe a larger parameter space than stage-III experiments, which is increasingly inefficient to perform with a Likelihood-based method. The SBI approach will bring improved computation flexibilities and in some cases, efficiencies, for a DESC analysis.
\paragraph{Simulation-Based Inference for Cluster Cosmology}
SBI provides a computationally efficient route to deriving posteriors for cluster-level and population-level parameters directly from simulated data vectors. This is particularly attractive for analyses that must jointly model individual clusters and the cluster population via hierarchical frameworks, where traditional likelihood-based methods become increasingly costly and brittle as the parameter space and model complexity grow. Ongoing work within DESC indicate that SBI can recover cluster weak-lensing mass posteriors consistent with those from MCMC, provided that model misspecification is not worse than in the explicit-likelihood case (Gill et al., in prep.). In addition, SBI has also been demonstrated to produce relevant constraints directly from cluster counts \citep{reza2022, reza2024}, and this approach is now being developed on DESC simulations as alternative, simulation-native pathways to cluster cosmology that can be integrated into the DESC Cluster Cosmology Pipeline. Compared to Stage-III experiments, DESC cluster analyses will need to incorporate richer astrophysical modeling and explore larger parameter spaces; SBI offers the algorithmic flexibility and, in many regimes, the computational efficiency required to meet these demands.

% \newpage
\subsubsection{Supernova cosmology and transients}
\begin{ThemeBoxA}[]
\themebullet \themekey{Methodology.} \meth{rnn}, \meth{transformer}, \meth{active-learning}, \meth{ensembles}\\
\themebullet \themekey{Challenges.} \challenge{covariate-shift}, \challenge{model-misspecification}, \challenge{data-sparsity}, \challenge{scalability}\\
\themebullet \themekey{Opportunities.} PLAsTiCC/ELAsTiCC simulation infrastructure, DESC leadership in alert broker integration (ALERCE, Fink, ANTARES)
\end{ThemeBoxA}
%%%%%%%%%%%%%%%%%%%%%%%%%%%%%%%%
% - methodology
% - challenges
% - unique opportunities
%%%%%%%%%%%%%%%%%%%%%%%%%%%%%%%%%%

%%%% BROAD MOTIVATION/SCIENCE %%%% Photometric classification of SNe Ia necessary to do cosmological inference with Rubin data. LSST will drive discovery rates, population-level studies of rare transients. Methods for large-scale inference are sorely needed. Triaging is still necessary for follow-up classification/science.

%%%% TECHNIQUES %%%%
% - Partial-phase classification and active learning (e.g., RESSPECT) can help for real-time triaging.
% RNNs/transformers 
% ensemble methods for UQ
%   - Challenges for photometric SN~Ia cosmology with AI: distribution shifts/selection bias, use of simulations, uq

%%%% opportunities %%%%

% - DESC's leadership in PLAsTiCC/ELAsTiCC drove the infrastructure to model these observations, now we have good simulations we can take advantage of.

% - Many alert brokers in various stages of AI integration; they're very involved in DESC, and so DESC can uniquely connect data streams to algorithms. 

The Vera C. Rubin Observatory's Legacy Survey of Space and Time (LSST) is expected to detect $\sim$10 million transient and variable objects each night, a thousand-fold increase over current surveys. The sheer volume and cadence of detections renders traditional spectroscopic classification infeasible for most events. This presents a major bottleneck to the identification of pure Type Ia supernova (SN~Ia) samples for cosmological distance measurements, and to constraining the explosion physics of populations of rare and novel phenomena for the first time. To achieve reliable cosmological constraints and meet DESC goals of reducing the systematic uncertainties from light curve modeling below 3\% of existing models, analysis techniques demand \textit{well-calibrated uncertainty quantification}, \textit{adaptive and scalable performance}, and \textit{robustness to covariate shifts and data corruption}.


\paragraph{Spectrophotometric Modeling} Type Ia supernovae (SNe Ia) are broadly homogeneous and viable standard candles, but diversity in their spectro-temporal properties and persistent host-dependent effects \citep{2006ApJ...648..868S,2010MNRAS.406..782S,2010ApJ...722..566L,2010ApJ...715..743K,2013ApJ...764..191H} still limit standardization precision. Modern ML approaches to standardization now focus on data-driven, differentiable, and multi-modal models rather than hand-engineered linear corrections. Already, progress in these directions can be seen in modeling using Probabilistic Auto-encoders~\citep{2022ApJ...935....5S} or Variational Autoencoders (parSNIP)~\citep{2021Boone_ParSNIP}, which predicts time evolving spectra (SEDs) from light curves, and uses a differentiable forward model to compare in observation space. High quality, well-calibrated data has been instrumental in these endeavours, which can be augmented by the LSST/Rubin samples; however, strategies that account for the shifts generated by calibration errors must still be investigated. 

\paragraph{Photometric Classification} The methodological evolution of photometric classifiers from feature-based approaches to end-to-end learning has been driven by the challenge of processing irregular, heteroskedastic observations: \href{https://github.com/LSSTDESC/snmachine}{SNmachine} \citep{SNmachine2016} leveraged a range of feature sets, from physics-based through to non-parametric approaches, coupled with a variety of traditional machine learning techniques to achieve high classification accuracy; SCONE's Gaussian process interpolation \citep{SCONE:2021} creates regular 2D representations from sparse observations; while transformer architectures \citep{2023Pimentel_Attention,2024Allam_Attention,ATAT2024} leverage self-attention mechanisms to handle missing data naturally. Hybrid, physics-informed approaches have also been explored to extract latent features from light curves using generative modeling, which are then used for classification \citep{2021Boone_ParSNIP}. Classification tools have already proven their utility in LSST survey optimization for supernova metrics, with realistic LSST survey cadences (e.g., \href{https://github.com/LSSTDESC/snmachine}{SNMachine}, \citealt{Alves2022, Alves2023}).   

The Dark Energy Survey (DES) pioneered the use of recurrent neural networks for photometrically classifying SNe~Ia for cosmological analysis \citep{Vincenzi:2023,SCONE:2021,Moller:2020,Moller:2022,DESSN:2024}. Propagating the prediction uncertainties from these models through to cosmological constraints remains an open problem. SuperNNova \citep{Moller:2020} addresses the former with an RNN providing calibrated probabilities essential for contamination modeling in dark energy constraints \citep{Vincenzi:2023}. In DES, BNNs and ensemble methods were also tested; for DESC, we can advance fully Bayesian approaches to photometric classification for LSST. %DESC's proximity to survey operations makes it uniquely positioned to advance fully Bayesian approaches to photometric classification for LSST.

\paragraph{Forward Modeling of the Time-Domain Landscape}
Observational modeling led by DESC has catalyzed neural approaches to photometric classification. The Photometric LSST Astronomical Time-Series Classification Challenge \citep[PLAsTiCC][]{2023Hlozek_PLasTicc,knop2023} provided 3.49M test light curves across 18 transient classes using SNANA simulations with LSST OpSim \citep{2014Delgado_OpSim} cadences and realistic observing conditions. The challenge established weighted logarithmic loss metrics prioritizing Type Ia supernovae and kilonovae for DESC science goals \citep{2019Malz_Metric}, with winning solutions employing gradient boosting and ensemble neural networks requiring engineered features \citep{2019Boone_avocado,2023Hlozek_PLasTicc}. The dataset remains a foundational benchmark for time-series representation learning in astrophysics years after its development \citep{Fraga:2024,Masson:2024,CadizLeyton:2025, CadizLeyton:2025:MoE,ROMAE2025}. Building on PLAsTiCC, ELAsTiCC \citep{2023AAS_ELAsTiCC} stress-tested end-to-end broker infrastructure with $\sim$50M alerts streamed in real-time to seven community brokers from September 2022-January 2023. 

\paragraph{Online Learning for Spectroscopic Optimization}
While archival photometric classification will suffice for the bulk of LSST's cosmological needs, inference over partially-obtained data remains crucial for prioritizing spectroscopic targeting before an event has ended. GRU-based recurrent and convolutional neural networks have successfully classified partial-phase synthetic light curves \citep{2019Muthukrishna_RAPID,SCONE:2021,2023Gagliano_FirstImpressions, shah2025b_oracle}, but performance on observed data remains modest. Transformer-based methods are being increasingly used \citep{ATAT2024}, with synthetic pre-training playing a growing role in bridging the simulation gap \citep{2025Gupta_Sims}. Within the DESC, host-galaxy correlations have been shown to improve early classification \citep{2021Gagliano_GHOST}; this has driven data-driven modeling of host-galaxy correlations for the ELAsTiCC challenge \citep{2023Lokken_SCOTCH}, although spurious host-galaxy associations and the small postage stamps of the field contained within the LSST alert packets may limit DESC's capacity for real-time inference using these data.

Beyond real-time classification, active learning faces unique astronomical challenges: objects must be selected for spectroscopic follow-up before informative light curve data are obtained, the untargeted population is substantially dimmer than the spectroscopically-confirmed sample used in training, and labeling costs vary dramatically with object brightness and sky position. RESSPECT, an initial approach to active learning for transient science, implements uncertainty sampling with Random Forest classifiers on Bazin \citep{2011Bazin_LCModel} parametric features, but requires minimum 5 observations per filter, limiting early-time selection \citep{2020Kennamer_RESSPECT}. More recent implementations have refined active learning for early-time Type Ia supernova identification, demonstrating effective follow-up optimization with simulations \citep{Ishida:2019}, real-data a posteriori \citep{Leoni:2022} and real-time observational campaigns \citep{Moller:2025}, the latter revealing the need for training sets containing events beyond supernovae. \href{https://github.com/MichelleLochner/astronomaly}{Astronomaly} \citep{2021Lochner_Astronomaly} introduces personalized anomaly detection by combining isolation forests with human relevance scoring, addressing the fundamental subjectivity of an anomaly label. The approach has been shown to double the rate of anomaly discovery in radio transients \citep{2025Andersson_astronomaly}. However, active learning remains fundamentally limited by the lack of an informative initial training set, such that early random sampling can produce biased or unrepresentative data that propagates through subsequent iterations of learning. This is a fundamental challenge for novelty detection in LSST data.

\paragraph{Prompt Processing with the Transient Alert Brokers}
The seven Rubin Community Brokers implement diverse classification pipelines. ALERCE employs a convolutional neural network for top-level classification from alert postage stamps \citep{2021Carrasco_Stamp}, and a hierarchical Random Forest applied to photometric features for classification along a 15-class taxonomy \citep{2021Sanchez_AlerceLC}. AMPEL uses a four-tier system with predominantly gradient-boosted random forests \citep{2025Nordin_AMPEL}, and Fink deploys multiple classifiers for early and late-time classification \citep{Fraga:2024,Leoni:2022, Moller:2020, Moller:2025, fink}. ANTARES employs multi-stage filtering with community-contributed Python classes for tagging sources \citep{2018Narayan_ANTARES}, while LASAIR integrates a boosted decision tree classifier from host galaxy properties \citep{2020Smith_ATLAS} with MOC-based watchmaps for coordination with 4MOST/TiDES, which will providing 35,000 transient spectra for SN~Ia cosmology \citep{2024Williams_Lasair}. These brokers have successfully demonstrated minute-scale latencies in processing millions of alerts during the ELAsTiCC campaign, with classification probabilities reported through standardized Avro schemas that enable systematic evaluation of heterogeneous ML architectures. DESC members are involved in all seven Rubin Community Brokers, offering substantial potential for shared software infrastructure for processing the LSST alert stream.

%The Rubin Community Brokers are \texttt{ALERCE} \citep{alerce}, \texttt{AMPEL} \citep{ampel}, \texttt{ANTARES} \citep{antares}, \texttt{Babamul}, \texttt{Fink} \citep{fink}, \texttt{LASAIR} \citep{lasair} and \texttt{Pitt-Google} and will be crucial for both endeavors, receiving, processing and filtering the 10 million nightly detections from LSST every night. They are crucial for both creation of samples for scientific analysis and the coordination of follow-up observations. Brokers have a strong AI/ML component with multiple classifiers targeting various supernovae types, AGNs, and other transients. Citations needed of ML algorithms developed in DESC in these brokers.

\paragraph{Cosmological Inference using Type Ia Supernovae}

Hierarchical Bayesian models have been applied to type Ia supernovae for various goals, including cosmological inference \citep[e.g.][]{2011March_BAHAMAS, 2016Shariff_BAHAMAS, 2015Rubin_Unity, 2025Rubin_Unity, 2018Feeney_H0}, constructing empirical SED models \citep[\href{https://github.com/bayesn/bayesn}{BayeSN}; ][]{2009Mandel_BayeSN, 2011Mandel_BayeSN, 2022Mandel_BayeSN, 2021Thorp_BayeSN, 2023Ward_BayeSN, 2024Grayling_BayeSN, Uzsoy_2024}, modeling intrinsic colors and dust extinction \citep{2017Mandel_SimpleBayeSN, 2022Thorp_BayeSN, 2024Thorp_BayeSN}, handling uncertain photometric classifications \citep{2007Kunz_BEAMS, 2012Hlozek_BEAMS} and redshifts \citep{2017Roberts_zBEAMS}, and modeling the spectrophotometric standards used in photometric calibration \citep{2025Boyd_DA, 2025Popovic_Dovekie}. However, in the LSST era such models will have to incorporate complex effects that cannot easily be treated analytically; for example, selection effects, photometric classification and photometric redshifts. Work is ongoing to enhance our statistical models using simulation-based inference (SBI), leveraging the flexibility of neural networks to capture these complex effects \citep[e.g.][]{2024_Boyd_Flows, 2024Karchev_SIDEreal, 2025Karchev_CIGARS}. SBI will enable scalable and principled statistical inference of cosmological parameters with LSST. 

\subsubsection{Theory and Modeling}
\begin{ThemeBoxA}[]
\themebullet \themekey{Methodology.} \meth{emulator}, \meth{gaussian-process}, \meth{neural-surrogate}, \meth{differentiable-programming}, \meth{sbi} \\
\themebullet \themekey{Challenges.} \challenge{scalability}, \challenge{model-misspecification} \\
\themebullet \themekey{Opportunities.} Flexible emulation frameworks, model selection and hypothesis testing, efficient construction of realistic mock datasets, gradient-based sampling.
% I edited this, feel free to change - Dani Leonard
\end{ThemeBoxA}

The role of theory and modeling within DESC is to provide the essential bridge between cosmological parameters and the statistical observables derived from LSST data. Accurate theoretical models are required to translate measured galaxy shapes, positions, and fluxes into constraints on dark energy, dark matter, and gravity. This entails constructing predictive models of large-scale structure, galaxy bias, baryonic physics, and lensing observables that can be robustly compared with data while marginalizing over astrophysical and observational systematics. As the scale and precision of LSST data demand modeling at unprecedented accuracy and speed, machine learning-based emulators, differentiable theory libraries, and simulation-based inference approaches are increasingly central to this effort, enabling fast and robust connections between data and theory.


\paragraph{Fast Surrogates for Cosmological Likelihoods} 

Emulation and related methods for creating fast-surrogate models using ML and AI will be of crucial importance in accelerating inference pipelines for DESC’s cosmological analyses.  Emulation is an indispensable tool for integrating aspects of modelling which by nature require simulation too slow to ever consider incorporating directly in sampling (e.g. those requiring N-body or hydrodynamical simulations). At the same time, even for aspects of modelling which are more moderate in evaluation cost (seconds rather than many-hours), emulation allows us to dramatically accelerate individual likelihood evaluations. This is one of the key ways we can make computationally feasible the sampling in high-dimensional parameter spaces which will be required for DESC analyses.

Work on these emulation techniques within DESC has included directly building emulation tools, particularly outside of wCDM models \citep{ramachandra2021matter}, as well as using emulators to efficiently evaluate modelling choices for LSST data \citep{boruah2024machine}. DESC's theoretical modelling package {\tt pyCCL} \citep{chisari2019core} natively supports key matter-power-spectrum emulation tools {\tt baccoemu} \citep{arico2021bacco} and {\tt cosmicemu} \citep{moran2023mira}. However, developing frameworks for DESC to analyze models outside wCDM in the nonlinear regime of LSST data remain a challenge that need to be addressed \citep{Ishak:2019BwCDM} and where AI can play a major role.  

Lightweight, highly accurate neural network models can take on the heavy lifting associated with speeding up massively-parallel computations that are inherently faster compared to the above examples (seconds per computation) but where the use cases require millions or billions of evaluations, leading to major computational bottlenecks. Examples include the construction of mock galaxy catalogues with realistic photometry, and inference of SPS parameters via SED-fitting to photometry for very large galaxy catalogues. An example use-case within DESC is \href{https://github.com/justinalsing/speculator/tree/master}{Speculator} \citep{SPECULATOR}, which is used within the \href{https://github.com/Cosmo-Pop/pop-cosmos}{pop-cosmos} generative model \citep{Alsing:2024, Thorp:2024, Thorp:2025, Deger:2025}. Speculator accelerates SPS calculations in pop-cosmos by factors of 10,000x on GPU hardware, relative to Flexible Stellar Population Synthesis \citep{Conroy:2009, Conroy:2010, ConroyGunn:2010, Johnson:2021}  on CPUs.


Looking to the future, DESC would benefit from developing mechanisms to enable emulation which has more flexibility with respect to modeling components. Our current methods of building a single emulator from scratch per modeling-set-up is high-cost (computationally and in terms of person-power). This does not scale well for enabling adaptation to new systematics modeling or inference in models outside of wCDM. Considering approaches which use meta-learning (e.g. \citealt{macmahon2025meta}) or which are philosophically aligned with foundation models would be of value. 

Another area of growth where fast emulators can make major impact is in model selection and hypothesis testing of beyond wCDM models. Instead of being limited by the cost of theoretical model evaluations, future analyses will be constrained by how efficiently inference pipelines can navigate and compare competing cosmological models. Integrating these emulators within agentic AI systems (\autoref{sec:llm_agentic})—which can autonomously refine training data, adapt inference strategies, leverage tools for evidence computation and simulation-based inference, and even propose new model extensions—will further accelerate discovery. For LSST-DESC, this synergy will transform the capacity to test gravity, dark energy, and dark-sector interactions, turning high-quality data into a powerful engine for identifying new physics. 





\paragraph{Differentiable Programming for Accelerated Sampling}

The jax-cosmo library \citep{JAX-COSMO} enables differentiable and hardware-accelerated cosmological computations. Together with a NumPy-compatible API and close integration with tools such as NumPyro \citep{phan2019composable} and JAXopt \citep{jaxopt_implicit_diff}, this makes jax-cosmo a practical framework for building scalable and fully differentiable cosmological models. Since its first release, jax-cosmo has been used in several contexts. \cite{10.1093/mnras/stae2138} employed it to build a differentiable forward model for testing gradient-based samplers such as Hamiltonian Monte Carlo (HMC) and No-U-Turn Sampler (NUTS) \citep{HoffmanGelman2014}, finding improved efficiency compared with standard Metropolis–Hastings methods. \cite{CosmoPower} combined jax-cosmo with neural-network emulators in CosmoPower-JAX, enabling high-dimensional Bayesian inference through automatic differentiation and GPU acceleration. The halox package \citep{halox} uses jax-cosmo for cosmological calculations such as power spectra and distance measures in its modeling of dark-matter halo statistics. \cite{2025arXiv250707833S} used jax-cosmo as a benchmark for testing a differentiable Fisher-information approach based on score matching. Recently too, \cite{bartlett2025symbolic} have developed a symbolic emulator that leverages genetic programming-based symbolic regression to derive compact, analytic expressions for cosmological observables, including the radial comoving distance, linear growth factor, and nonlinear matter power spectrum. When integrated into the jax-cosmo framework, these symbolic approximations replace computationally expensive numerical evaluations—such as CAMB \citep{Lewis_2000} or HALOFIT (eg. \citealp{Takahashi_2012})—with differentiable, GPU-accelerated formulas, achieving sub-percent accuracy while reducing runtime by over an order of magnitude. This seamless integration enables efficient, gradient-based inference pipelines, such as those using Hamiltonian Monte Carlo, without sacrificing precision or interpretability.


\subsubsection{Cosmological and Survey Simulations}
\begin{ThemeBoxA}[]
\themebullet \themekey{Methodology.} \meth{emulator}, \meth{diffusion-model}, \meth{differentiable-programming}, \meth{sbi}\\
\themebullet \themekey{Challenges.} \challenge{covariate-shift}, \challenge{systematics-modeling} \\
\themebullet \themekey{Opportunities.} Joint modeling of galaxies and environments, inference at catalog and field level, Modular components for DESC simulators, survey-scale generative models, survey design, systematics mitigation, pipeline stress tests.\end{ThemeBoxA}

Cosmological simulations are a fundamental tool to not only validate analysis pipelines, but increasingly to provide “theory” samples for SBI frameworks. Producing mock survey data at the scale and accuracy required for LSST science remains a major challenge, which machine learning can help address by emulating expensive numerical predictions and by providing data-driven models of otherwise poorly constrained aspects of the galaxy population. Neural emulators have long been used as fast surrogates for non-differentiable components of cosmological forward models (e.g.\ summary statistics of N-body simulations as in CosmicEmu \citep{CosmicEmu}, Aemulus \citep{2019ApJ...875...69D}, CosmoPower \citep{CosmoPower}, or 21cmEMU \citep{21CMEMU}) and more recently for accelerating stellar population synthesis (SPS) calculations via models such as \textsc{Speculator} and \textsc{ProMage} \citep{SPECULATOR,ProMage}. Extending these approaches to additional components of the simulation pipeline holds the promise of greatly increasing the dynamical range, realism, and flexibility of LSST mock catalogs at manageable computational cost.

\paragraph{Population-Level Generative Models for Realistic Galaxy Catalogs}
Diffusion-based generative models operating in the space of physical galaxy parameters provide a powerful route to building realistic mock catalogs that remain anchored in deep-field observations. The \href{https://github.com/Cosmo-Pop/pop-cosmos}{\textsc{pop-cosmos}} framework \citep{Alsing:2024,Thorp:2025,Deger:2025} defines a score-based diffusion model over a high-dimensional SPS parameterization (star-formation histories, metallicities, dust, nebular emission, etc.), calibrated on $\sim$420,000 galaxies from COSMOS2020 \citep{Weaver:2022} spanning 26 bands from UV to mid-IR. Rather than directly emulating observed fluxes, \textsc{pop-cosmos} learns a data-driven prior $p(\boldsymbol{\theta}_{\mathrm{SPS}},z)$ over physical parameters and redshift that reproduces the joint distribution of observed photometry. This model encodes realistic priors on star-formation histories over cosmic time, and learn the evolution of the star-forming sequence. When coupled to survey-specific selection functions and noise models, \textsc{pop-cosmos} can therefore generate realistic mock galaxy catalogs that inherit both empirical constraints from deep multi-wavelength data and the flexibility of generative modeling. Within DESC, such population-level priors are particularly valuable as building blocks for photometric-redshift calibration, for testing SED-fitting pipelines, and for providing consistent galaxy populations to survey simulators such as Diffsky \citep{OpenUniverse2024}.

\paragraph{Differentiable Empirical Galaxy–Halo Forward Modeling}
Complementary to purely catalog-level generative approaches, differentiable galaxy–halo forward models seek to describe galaxy populations as conditional generative processes on top of dark-matter structure. The \textsc{Diffsky} framework \citep{OpenUniverse2024} rebuilds the traditional “halo $\rightarrow$ SFH $\rightarrow$ SED’’ chain using differentiable, physically interpretable blocks. \textsc{Diffmah} \citep{Diffmah} provides a JAX-based, few-parameter model of halo mass assembly $M_{\rm halo}(t)$, replacing noisy merger trees with smooth, analytic, differentiable growth histories. On top of this, \textsc{Diffstar \citep{Diffstar}} models in-situ star formation histories with a small set of parameters (e.g.\ star-formation efficiency, gas-consumption timescale, quenching time), while \textsc{DiffstarPop} \citep{Diffstar} lifts this to the population level by learning the statistical link between SFH parameters and halo assembly across suites of reference simulations. Finally, DSPS \citep{DSPS} maps these SFHs and associated metallicity/dust parameters to SEDs and photometry entirely within JAX. Together, \textsc{Diffmah} + \textsc{Diffstar}/\textsc{DiffstarPop} + DSPS replace merger trees, non-differentiable semi-analytic recipes, and black-box SPS calls with a modular, probabilistic, fully differentiable stack whose low-dimensional, physically meaningful parameters can be calibrated and explored with gradient-based methods and SBI, while still generating large, realistic synthetic catalogs.


\paragraph{Differentiable Cosmological N-body solvers} Recent years have seen the emergence of particle–mesh $N$-body solvers implemented in modern, GPU-accelerated, deep-learning frameworks that support automatic differentiation \citep[e.g.][]{FlowPM, pmwd, DISCO-DJ, JAXPM}. Automatic differentiation enables hierarchical Bayesian inference directly over forward simulations of large-scale structure, opening a path toward full-field inference and near-optimal extraction of cosmological information. The main obstacles to deploying these methods at LSST scale are the computational and engineering demands of simulating survey-sized volumes in a differentiable way. Computing derivatives through the simulation implies non-trivial memory costs that are difficult to satisfy under the constraints of GPU accelerators. Several complementary strategies are being developed to address this challenge, including multi-node domain decomposition for distributed simulations \citep{Kabalan_jaxDecomp_2025}, techniques to reduce the memory cost of gradient evaluation \citep{Li__2024}, and improved time integrators that achieve a given accuracy with fewer time steps \citep{Rampf2025}. Beyond enabling full-field inference, differentiable simulations also make it possible to combine physics-based solvers with learned components, yielding hybrid schemes that can improve the speed and accuracy of particle–mesh simulations \citep{Lanzieri_2023, Payot2023}.

\paragraph{Hydrodynamical-Simulation-Based Mappings of Galaxy Properties}
A complementary strategy is to treat state-of-the-art hydrodynamical simulations as high-fidelity “teachers’’ and use machine learning to distill their complex, small-scale physics into fast, effective models defined directly on dark-matter fields. Rather than specifying parametric galaxy–halo or SPS models, these approaches learn mappings from halo or large-scale-structure descriptors to galaxy properties as realized in the simulations. Recent work on intrinsic alignments illustrates this paradigm. A traditional approach, followed by \citep{VanAlfen2024} is to develop an empirical IA model constrained by hydrodynamical simulations within a flexible HOD-like framework. A more ML-oriented approach is to learn an end-to-end emulator of galaxy properties as demonstrated in \citep{Jagvaral2025} introduce a geometric deep-learning approach in which galaxy shapes and orientations from IllustrisTNG \citep{IllustrisTNG} are modeled using E(3)-equivariant graph neural networks defined on the cosmic web. In the latter, galaxy orientations are treated as elements of the Lie group $\mathrm{SO}(3)$, and a diffusion-based generative model on $\mathrm{SO}(3)\times\mathbb{R}^n$ learns the conditional distribution of shapes and orientations given halo mass, environment, and tidal field. This yields a fast, simulation-calibrated surrogate capable of reproducing intrinsic-alignment statistics at the percent level, providing a route to embedding hydro-level realism into DESC mock catalogs without rerunning expensive hydrodynamical simulations.


\subsection{Impact at the Technical Level}

\subsubsection{Deblending}
\begin{ThemeBoxA}[]
\themebullet \themekey{Methodology.} \meth{vae}, \meth{npe}, \meth{instance-segmentation}, \meth{diffusion-model}, \meth{normalizing-flow} \\
\themebullet \themekey{Challenges.} \challenge{blending-crowding}, \challenge{metrics} \\
\themebullet \themekey{Opportunities.} Multi-survey training, Data-driven Priors, Generative Models 
\end{ThemeBoxA}

Turning pixels into objects is a fundamental problem in astronomical survey pipelines. Object detection and deblending of LSST data is a crucial step in producing catalogs useful for DESC science.  Given its unprecedented depth for a ground-based survey, LSST will face new challenges in its detection pipelines compared to previous legacy surveys \citep{Melchior21blending}.  Blending, or the overlapping of source light profiles, is an imaging systematic that affects all downstream analysis, as it becomes difficult (in reality intractable) to disentangle photons from a given source in a blend.  This problem is exacerbated with increased observing depth, as more light is collected from sources that are overlapping due to line-of-sight projections or physical interactions. Traditional object detection pipelines for wide-field surveys use a maximum likelihood estimator method \citep{Bosch18hsc} to identify peaks in intensity corresponding to sources in an image.  This method, while statistically justified, is still subject to failure modes, wherein AI can provide alternative and complimentary methods.  Similarly, traditional deblending algorithms typically rely on models and assumptions about source light profiles that may not provide sufficient flexibility for the billions of sources LSST will observe \citep{Melchior18scarlet}.  DESC has been exploring and developing AI methods to aid in these challenging problems, which are crucial to understand and mitigate.

\paragraph{Catalog-level Blend Identification}  While a majority of sources observed by LSST are expected to have some level of blending, a particularly pernicious case is that of unrecognized blends.  These are sources in a blended scene that are indeed distinct (determined from high-resolution space-based observations), but are only recognized as a single source by cataloging and deblending pipelines.  Unrecognized blends impact measured properties such as galaxy shapes \citep{Dawson16ublends}, photometric redshifts \citep{Liang25catblend}, and more.  Estimates of the level of unrecognized blends in LSST typically lie from $\sim15-30\%$, with analysis of early LSST data compared to catalogs from HST CANDELS yielding an unrecognized blend rate of 18\% \citep{sitcom128}.  Blends that remain at the catalog level are definitionally unrecognized blends but may still be detectable as outliers in the multi-band photometry and their shapes. The use of Random Forests and Self-Organizing Maps along with various anomaly detection algorithms were tested in \cite{Liang25catblend} who showed that it is possible to detect unrecognized blends at a cost to the sample size. These algorithms were used to assign an unrecognized blend probability, however improvements can be made for specific science cases. For example, using the blend entropy [Ramel et al. in prep, Project 284] can improve cluster cosmology by removing the most problematic unrecognized blends for cluster analysis. Designing better blending metrics like blend entropy and incorporating that into machine learning algorithms like gNNs is the main goal of friendly [Project 295]. 

\paragraph{Image-level Deblending Using Deep Learning} Deep learning algorithms designed for object detection and deblending provide an alternative method to traditional pipelines that may help improve catalog completeness and downstream source property measurements.  For instance, the Bayesian Light Source Separator (BLISS) framework utilizes neural posterior estimation alongside training a supervised convolutional neural network to produce probabilistic catalogs directly from simulated LSST images.  This method outperforms the standard LSST pipeline in source detection, as well as downstream flux measurement, star/galaxy classification, and galaxy shape measurement, \citep{Duan25NPE}.  The DeepDISC instance segmentation \citep{Merz23DeepDISC} framework produces object catalogs and segmentation masks from image data, and is being tested with joint Roman-Rubin data to incorporate multimodal information for downstream detection and deblending improvement. Both DebVader \citep{Arcelin21debvader} and MADNESS \citep{Biswas25mad} use Variational Autoencoders to handle blending. They use self-supervised training to learn the structure of isolated galaxies. Through additional training of a deblending encoder they learn to isolate a galaxy from a blend. A specialized decoder can directly measure the characteristics of the galaxy (shape, photo-z) without reconstructing explicitly the image of the isolated galaxy. MADNESS adds to the architecture a normalizing flow to improve the performances by modelling the latent space distribution of galaxies that provides an explicit likelihood for posterior optimization. Ongoing work uses a multimodal VAE to learn both from imaging and spectroscopy, adding more information in the latent space to improve the galaxy characteristics measurement, especially photo-z. The very principles of deblending VAEs alleviate the impact of unrecognized blends, and ongoing work on the use of probabilistic catalogs where the number of detected galaxies is itself non-deterministic will reduce it even more.  Even if blended sources appear in such close proximity that LSST imaging will not be able to recognize the overlap and detect the blended group as one source, it is feasible to model the detected sources first, compute the residuals from the fit, and run detection again on the residuals. Because the residuals can have complicated structure, it is beneficial to perform the detection on multi-band residuals, where unrecognized sources appear as colored, localized over- or underdensities. Recognizing them, as well as their likely centers is possible and fairly effective with computer vision architectures like YOLO \citep{sowmya_kamath_2020_3721438}.

\paragraph{Data Driven Priors for Deblending with Explicit Likelihoods}

Generative models like normalizing flows and diffusion models can be trained on unblended galaxies (potentially with limited amounts for space-based data) and then serve as data-driven priors for galaxy morphologies \citep{Lanusse19gen}. Posterior optimization and sampling becomes possible for inverse problems with explicit likelihood functions (such as inpainting, deconvolution, and deblending). This is particularly effective in low SNR cases, where the deblender scarlet \citep{Melchior18scarlet} which is the default deblending method in the Rubin pipeline, is outperformed by a new, prior-augmented version \citep{Sampson24scarlet2}. The same approach can also perform transient photometry in the presence of a host galaxy without the need for difference imaging \citep{Ward25scarlet2}. Additionally, posterior optimization in latent space by the MADNESS deblender \citep{Biswas25mad}, using data-driven priors, also outperformed scarlet.


% Skipping for now...
% \newpage
% \subsubsection{Instrumental Response Modeling}
% \begin{ThemeBoxA}[]
% \themebullet \themekey{Methodology.} TBD \\
% \themebullet \themekey{Challenges.} TBD \\
% \themebullet \themekey{Opportunities.} TBD
% \end{ThemeBoxA}

\subsubsection{Shape Measurement}
\begin{ThemeBoxA}[]
\themebullet \themekey{Methodology.} \meth{differentiable-programming}, \meth{deep-network}, \meth{sbi} \\
\themebullet \themekey{Challenges.} \challenge{covariate-shift}, \challenge{blending-crowding}, \challenge{uq-calibration}, \challenge{systematics-modeling} \\
\themebullet \themekey{Opportunities.} Joint optimization of detection, deblending, and shear, hybrid analytic–neural estimators,  realistic multi-instrument and multi-epoch scene modeling, active learning, unified shear–photo-z modeling
\end{ThemeBoxA}

Weak-lensing shape measurement is one of the most sensitive components of the DESC analysis pipeline: small percent-level biases in ensemble shear propagate directly into the cosmological parameters targeted by LSST. Meeting DESC requirements therefore demands methods that are simultaneously accurate enough to control multiplicative and additive shear biases, computationally efficient enough to process billions of galaxies, and amenable to calibration. A unifying theme of recent work is to exploit differentiability and GPU acceleration, both in explicit shear-calibration schemes and in forward models.

\paragraph{Analytic Calibration and Differentiable Shear Estimators}
The Analytic Calibration (AnaCal) framework  \citep{Li2023,Li2025_bias} demonstrates how differentiability can be used to obtain high-precision shear responses and noise-bias corrections without relying on large external simulation campaigns. By expressing galaxy properties and pixelized images in terms of differentiable basis functions, AnaCal delivers analytic shear responses for detection, selection, and shape measurement, achieving LSST-grade accuracy with sub-millisecond CPU inference per galaxy. At present, however, AnaCal is implemented as a CPU-based shape-measurement engine that assumes pre-detected sources. Extending this paradigm to GPU-based implementations and integrating it more tightly with detection and scene modeling would unlock orders-of-magnitude throughput gains. More broadly, metacalibration-style schemes also stand to benefit from differentiable image models and measurement operators: with gradients available throughout the pipeline, shear response and noise-bias corrections can be computed faster and more robustly.

\paragraph{Deep Learning–Based Shape Estimators}
Modern deep-learning architectures provide a natural path to end-to-end differentiable shear estimator that can simultaneously integrate detection, deblending, shear estimation, and robustness to image defects. Neural networks are inherently GPU-accelerated, highly parallel, and differentiable, making them well suited to high-throughput shear inference at LSST scale. Such an approach was originally demonstrated in \citep{Ribli2019} and can are being explored in a DESC context using the DeepDISC architecture \citep{Merz23DeepDISC} which was originally designed as a general purpose architecture for detection and segmentation and which can estimate gravitational shears within a single GPU-resident and differentiable model. Being automatically differentiable, this estimator can be calibrated using the schemes mentioned above.

\paragraph{Hierarchical Forward Modeling with Differentiable Image Simulators}
A complementary strategy frames shape measurement as a hierarchical forward-modeling problem, in which cosmological parameters, population-level distributions of galaxy properties, and individual galaxy shapes are inferred jointly from the pixel data \citep{Schneider2015}. In this view, a forward model generates simulated images given a set of hierarchical parameters, and inference proceeds by comparing these simulations to the observed images. Such approaches are made practical thanks to the JAX-GalSim effort \citep{jaxgalsim} which re-implements key GalSim \citep{ROWE2015121} functionalities in JAX, making this forward model fully differentiable and GPU-accelerated while supporting vectorized batch simulations of thousands of galaxies at once. In ongoing DESC efforts, JAX-GalSim is used to implement the hierarchical shear-inference framework of \citet{Schneider2015}, but instead of traditional MCMC, gradient-based samplers (NUTS), GPU-acceleration, and batching can be used to yield roughly an order-of-magnitude speedup while keeping multiplicative shear biases within LSST requirements. While the aforementioned approach relies on analytic surface brightness profiles to model galaxies, more realism can be achieved through projects like Scarlet2 \citep{Sampson24scarlet2} which extends the modeling to non-parametric morphologies and blended scenes observed with multiple instruments, providing a JAX-based, differentiable scene-modeling framework in which gradients of the likelihood with respect to source parameters and hyperparameters are readily available.

\newpage
\section{Methodological Research Priorities to Advance ML for Precision Cosmology}
\label{sec4:aiml_research}

Extracting robust cosmological constraints from \acrshort{lsst} requires not only advanced algorithms but also a coherent methodological foundation that bridges simulation, data processing, and inference. Each of these pillars must meet unprecedented demands in scale, accuracy, and interpretability, demands that challenge the limits of both physical modeling and \acrshort{ml}. Beyond simply applying existing \acrshort{ai} techniques, \acrshort{desc} must develop methods tailored to the structure of astronomical data, the physics of observables, and the statistical rigor required for precision cosmology.

The scientific ambitions of LSST thus motivate AI/ML research in several key areas.
First, Bayesian inference and \acrshort{acr:uq} must evolve to handle the high-dimensional, hierarchical models that describe cosmic fields and galaxy populations, while maintaining interpretability and calibration across vast data volumes.
Second, \acrshort{acr:sbi} and related implicit-likelihood methods must confront the challenge of model misspecification and covariate shifts, ensuring that learned posteriors remain valid when simulations imperfectly represent real observations.
Third, physics-informed modeling, through differentiable programming and hybrid generative–physical architectures, offers a path toward interpretable and physically consistent deep learning, capable of representing both known and unknown components of the Universe.
Fourth, discovery and anomaly detection are essential to LSST’s potential for unexpected science, requiring representation learning and active human-AI collaboration to identify rare and previously unmodeled phenomena.

This section examines these research directions in detail, outlining both recent progress and outstanding challenges. We emphasize not only algorithmic innovation but also the validation, calibration, and interpretability principles required to integrate AI into cosmological analysis pipelines.

The fundamental question is: \textit{What would convince us of a cosmological result obtained with AI?}
Answering this question defines the research agenda for AI/ML in DESC and ensures that ML becomes not merely a computational shortcut, but a scientifically trustworthy component of cosmological inference.


\subsection{Bayesian Inference and Uncertainty Quantification}

\acrshort{acr:uq} represents perhaps the most critical challenge for deploying deep learning in precision cosmology. It is an area where \acrshort{ai} in the sciences demands solutions that differ from those in many commercial settings. Robust UQ must distinguish between aleatoric uncertainties (irreducible measurement noise) and epistemic uncertainties (model limitations and incomplete knowledge), as these have fundamentally different implications for cosmological inference and systematic error budgets.
\acrshort{ml} is poised to be revolutionary for inference, but only if current challenges are satisfactorily addressed. 

\subsubsection{Explicit Likelihood-Based Bayesian Inference}
\label{sec4:Bayes}

\begin{ThemeBoxA}
\themebullet \themekey{Related Methodologies.} \meth{hierarchical-bayes}, \meth{variational-inference}, \meth{gaussian-process} \\
\themebullet \themekey{Addresses.} \challenge{uq}, \challenge{scalability}
\end{ThemeBoxA}
    The inferential paradigm in astrophysical and cosmological data analysis has been for the past two decades primarily Bayesian, as this offers conceptual, methodological, and computational benefits~\citep{Trotta_2008}. The massive increase in data size and complexity afforded by \acrshort{lsst} will require a new step forward in inferential methodology, as LSST data will challenge the computational feasibility of current inferential engines. We cover in this section the case of \textit{explicit inference}, where we have the ability to directly evaluate the log-likelihood of our probabilistic models, and potentially its gradients. 

\paragraph{Accelerating Posterior Inference} \acrshort{mcmc} methods have been the workhorses of likelihood-based Bayesian parameter inference to date. Notable examples are Gibbs sampling \citep{Casella1992} and parallelized versions of Metropolis--Hastings (e.g., the affine-invariant sampler by ~\citealp{goodman2010ensemble, Foreman_Mackey_2013}). For cases where multimodality and/or strong parameter degeneracies are important, nested sampling~\citep{RN599,Ashton_2022} in its many variants (e.g., \texttt{MultiNest}~\citealp{Feroz:2007kg, Feroz:2009}; \texttt{PolyChord}~\citealp{Handley:2015,Handley:2015fda}; \texttt{dynesty}~\citealp{RN1654}; \texttt{DNEst4}~\citealp{Brewer:2016scw}), recently improved with gradients~\citep{lemosGGNS}, accelerated with neural emulators~\citep{Lovick:2025wdj}, or normalizing flows~\citep{RN1653} have been key to ensuring reliable inference.  

However, as analyses become increasingly complex, involving large numbers of nuisance parameters and expensive likelihood evaluations, the cost of running cosmological inference with conventional techniques becomes prohibitive. 

One avenue to speed up inference is to leverage access to the gradients of the log-posterior. When such gradients are available, a number of inference methods can benefit from them, including the well-established \acrshort{hmc} \citep{neal2011hmc}, as well as more modern generalizations such as \acrshort{nuts} \citep{HoffmanGelman2014}. Remarkably, physical ideas continue to lead the development of gradient-based inference techniques. \acrlong{rhmc} \citep[\acrshort{rhmc}][]{RHMC} simulates trajectories in arbitrary geometries by following the geodesics of the likelihood manifold. This makes RHMC schemes extremely robust sampling algorithms for high-dimensional inference in the face of severely non-gaussian posteriors. However, RHMC has not seen a wide spread adoption due to instability of second-order auto-differentiation needed to compute the curvature of the likelihood. In a similar vein, relativity-inspired HMC schemes \citep{RelativtyHMC} have recently been proposed that particularly target Minkowski geometries. This effectively introduces a maximum speed for the simulated particle, slowing it down in the areas of most challenging geometry, achieving most of the goals of RMHC without many of its hurdles. The recent Ray-tracing sampler \citep{2025arXiv251025824B} take a related approach, using light refraction as the guiding analogy to steer samples toward high-likelihood regions while providing a unifying framework in which HMC, LMC, and related methods emerge as special cases. As an alternative to Hamiltonian dynamics, \acrlong{lmc} (\acrshort{lmc}; \citealp{LangevineMC}) is based on a Langevin diffusion (an \acrshort{sde} whose invariant distribution is the target posterior) and in practice simulates a discretization of this process to generate approximate posterior samples. In this framework, the Metropolis adjustment that ensures the target distribution is sampled can be replaced with a bias requirement on the solution of the SDE, leading to significant speed ups \citep{GrumittLangevin}. A final avenue of improvement is to revise the assumed partition function of the particles simulated by the inference algorithm. Traditional HMC schemes assume that the distribution of particles being simulated follows a canonical partition function. However, more efficient sampling schemes can be constructed by exploring other partition functions. Microcanonical or energy-conserving HMC \citep[\acrshort{mchmc},][]{Robnik2022MicrocanonicalHM} explores the posterior distribution using a single energy shell. In a way similar to relativistic schemes, this is achieved by modifying the momentum of the particle to slow down at the regions of high-posterior density \citep{SteegHMC}, leading to a far more efficient sampling. \Acrlong{mclmc} \citep[\acrshort{mclmc};][]{MCLMC} is a sampling algorithm that combines all the ideas described above to great success. MCLMC already has been deployed to perform inference on physics problems such as lattice field theory simulations \citep{MCLMC} and even for cosmology where it has been shown to speed up field-level inference by an order of magnitude \citep{2023arXiv230709504B,simononfroy2025benchmark}. This makes MCLMC the cutting edge of gradient-based inference schemes and a promising tool to speed up analyses within \acrshort{desc}.   

Alternatively, one can replace a complicated posterior distribution with a more tractable one. \acrshort{vi} aims to find an approximation to the posterior distribution $p$ by a ``surrogate" parametrized distribution $q_\phi$, whose parameters $\phi$ are trained to minimize the Kullback-Leibler divergence between $q$ and $p$ (see, e.g., \citealp{Uzsoy_2024, JAX-COSMO}, using the JAX-powered NumPyro framework, \citealp{phan2019composable}). Here gradients are only needed for $q$, not for $p$.

Another area of research focuses on \textit{neural sampling methods}, which leverage in various ways neural networks to accelerate sampling while attempting to preserve asymptotic correctness guarantees. For example, normalizing flows have been used to re-parameterize the sampling space and cure complex geometries \citep{2022PNAS..11909420G}. Another recent line of research also leverages ideas from diffusion models and uses a neural score model to accelerate sampling \citep{2025arXiv250411713H}. 

As shown above, such inference strategies depend on differentiable components and will benefit greatly when likelihood codes are rewritten in frameworks that support automatic differentiation. An additional advantage of making probabilistic models compatible with such frameworks is that they usually support \acrshort{gpu} acceleration and vectorization, which opens up yet another avenue for acceleration---e.g.,the Numpyro \citep{phan2019composable, bingham2019pyro} and BlackJax \citep{cabezas2024blackjax} libraries written in JAX. Affine-invariant samplers \citep[see][]{Foreman_Mackey_2013} are particularly suited to vectorization on GPU hardware, as has been demonstrated in astronomy contexts (e.g.,\citealp{Thorp:2024, Thorp:2025}, using the \href{https://github.com/justinalsing/affine}{\tt affine} sampling code). 

\paragraph{Bayesian model comparison} Estimation of the Bayesian evidence, the central quantity for model comparison, remains challenging when the models being compared are very high dimensional. Nested sampling has been established as one of the main methods for Bayesian evidence computation, but in its original formulation it suffers from the curse of dimensionality: the efficiency of the constrained sampling step decreases rapidly as the dimensionality of parameter space increases. This has been somewhat mitigated by recent developments such as \texttt{PolyChord}~\citep{Handley:2015}, which can be used in a few hundreds of dimensions; \texttt{dynesty}~\citep{RN1654}, which uses dynamical allocation of live points \citep[see also][]{higson19}; and \acrlong{ggns}~\citep[\acrshort{ggns};][]{lemosGGNS}, which exploits gradients, generative flows and differentiable programming to achieve better efficiency and accuracy in up to $\sim 200$ dimensions.
A suite of other methods for the evaluation of the high-dimensional average of the likelihood over the prior are also being explored, sometimes combining density estimation with neural techniques  \citep[e.g.,][]{RN590,mcewen2023,Srinivasan_2024}. However, they remain confined to moderately low-dimensional parameter spaces, of order a few tens of dimensions. 

The frontier represented by evidence estimation in very large dimensional (of order $10^3$ or more) parameter spaces from real data remains largely untouched outside of synthetic demonstration examples where the ground truth is known. \acrshort{nre} shows promise in this respect, in that evidence estimation can be obtained from an NRE architecture by adding a suitable inferential head that is trained only on model labels, thus implicitly marginalizing over all parameters in the model. Such an approach naturally also generalizes to performing Bayesian model averaging. An example of this method is~\cite{SimSIMS}, where six models for empirical corrections for \acrshort{snia} data are compared from \acrfull{csp} observations \citep{krisciunas17} within a Bayesian hierarchical model setting with $\sim 4,000$ latent variables. 

\paragraph{Hierarchical Bayesian Models in Extremely High Dimensions} The manyfold increase in data size requires in many cases a more sophisticated model to capture previously unimportant effects; this in turn increases the dimensionality of the parameter space (especially in hierarchical models, where the latent space dimensionality scales with the number of objects within the model); the likelihood might become intractable, or previously used approximations, such as approximate Gaussianity or linear propagation of errors~\citep{Karchev2022}, neglecting of Eddington bias~\citep{Karchev_STARNRE}, might break down. 

 \autoref{sec3:wlss} introduced so-called full-field inference for cosmological surveys, in which not only cosmological parameters are inferred but also the initial conditions that seed the evolution of the large-scale structure of the Universe~\citep{porqueres2023fieldlevelinferencecosmicshear}. The fidelity of the forward simulation directly controls the accuracy of posterior constraints on cosmological parameters; consequently, this approach requires exploring extremely high-dimensional parameter spaces (millions to billions of parameters). Sampling such spaces is intractable for traditional MCMC and instead calls for gradient-based methods, as noted above. This, in turn, demands a forward simulation that is both fast and differentiable (see \autoref{sec3:sims}) to make full-field inference at LSST scale attainable.

It is worth noting that such hierarchical full-field inference models are substantially more computationally expensive than alternative \acrshort{acr:sbi} methods (see next section~\ref{sec4:sbi}), but offer several advantages. First, analyzing statistical errors directly in data space is more interpretable than working with the compressed summary statistics typical of SBI workflows; even with optimal compression, signals can mix and model misspecification becomes difficult to detect. Here, systematic contamination can be treated as additional parameters to be sampled \citep{2019porqueres}, becoming a machine-aided report of contaminations that have a characteristic pattern on the sky. Second, hierarchical Bayesian inference is designed for extensible, modular models in which new physics can be added—e.g., augmenting the simulation with a baryonification model—whereas SBI would require retraining neural density estimators from scratch. Taken together, these properties make hierarchical Bayesian inference well suited to joint inference of cosmology, systematics, and redshift-distribution uncertainties—capabilities that are considerably more difficult with implicit approaches. Additionally, hierarchical inference provides a digital twin of the Universe, which has multiple scientific applications but also provides a unique way of testing the results by cross-validating with independent data \citep{stopyra24}.

\subsubsection{Implicit Likelihood Bayesian Posterior Inference}
\label{sec4:sbi}

\begin{ThemeBoxA}
\themebullet \themekey{Related Methodologies.} \meth{sbi}, \meth{npe}, \meth{normalizing-flow} \\
\themebullet \themekey{Addresses.} \challenge{uq}, \challenge{scalability}
\end{ThemeBoxA}

The other paradigm is implicit inference, in which we do not assume direct access to the likelihood function, but only have access to samples from the joint distribution $p(x, \theta)$ of data samples $x$ and parameters of interest $\theta$. It should be noted that this situation covers both the case of \acrshort{acr:sbi}, and the case where $x$ and $\theta$ are available from a training sample of real observations (the canonical example being \acrshort{photoz} estimation from a set of spectroscopic observations). 

In particular, SBI is rapidly emerging as a powerful alternative to traditional fitting techniques for Bayesian models. The key idea is to replace an explicit likelihood function by forward simulating (under the model) parameter-data pairs, which are then used to train a neural network to perform inference \citep[e.g.,][]{Alsing2018, AlsingWandelt2018, AlsingWandelt2019,savchenko2024,lyu2025}. The advantages are that the (potentially intractable) likelihood can, in principle, incorporate physical effects of arbitrary complexity, which would otherwise be difficult to model (including, e.g., selection effects and complex parameter dependencies). In some variants~\citep{Miller2021} the 1- or 2-dimensional {\em marginal} distribution for the parameters of interest is targeted directly, thus circumventing the need to evaluate the high-dimensional joint posterior over all parameters in the model; such approaches are naturally suited to Bayesian evidence estimation~\citep{SimSIMS}. 

Additionally, inference can be {\em amortized} within a certain prior range, meaning that once trained the network can deliver almost instantaneous posteriors for a wide range of parameter values, a critical benefit when dealing with billions of galaxies \citep{2022ApJ...938...11H}.  This speed-up also permits posterior calibration methods (e.g., guaranteed coverage), which are computationally unfeasible with traditional posterior evaluation methods.

\paragraph{Neural Density Estimation (NDE) methods} The fundamental building blocks of these methods is \acrshort{nde}, where a neural network is used to estimate a distribution, or a ratio of distributions. Various kinds of methods exist: \acrshort{nle} (e.g., \citealp{Papamakarios2019, Lueckmann2019, Alsing:2019}), \acrshort{acr:npe} (e.g., \citealp{PapamakariosMurray2016, Lueckmann2017}) and \acrshort{nre} are among the most popular (for an overview see~\citealp{Alsing:2019, Cranmer2020, Lueckmann2021}). Implementations of NLE and NPE both learn a density based on simulated parameter-data pairs (see, e.g., \citealp{Alsing:2019}), with a variety of different approaches used for learning the multivariate joint or conditional density. Approaches to this include Gaussian mixtures \citep[e.g.,][]{Alsing2018}, mixture density networks \citep[e.g.,][]{PapamakariosMurray2016}, and normalizing flows (e.g., \citealp{Papamakarios2017, Papamakarios2019, Alsing:2019, Jeffrey:2021, 2022ApJ...938...11H}; for a review see \citealp{Kobyzev2021, Papamakarios2021}). More sophisticated density estimators used in generative modeling -- such as continuous normalizing flows \citep{Grathwohl2018, Chen2018}, score-based diffusion models \citep{Song:2020}, flow-matching models \citep{Lipman:2022}, and transformers \citep{Transformers2017} -- are also well suited to NLE and NPE tasks \citep[e.g.,][]{DiazRiveroDvorkin2020, Geffner:2022, Wildberger:2023, Gloeckler:2024} alongside the generative modelling tasks they are commonly used for \citep[e.g.,][]{Alsing:2024, Cuesta:2024, Thorp:2025}. 

\paragraph{Optimal Neural Summarization} 
To ensure the robustness of implicit inference, the process is usually divided into two main steps enabling each neural network to focus on a specific task: (1) compression of high-dimensional data into informative summary statistics, and (2) performing Bayesian inference using neural density estimation methods on this low dimensional but highly informative statistic. To maximize information extraction and improve constraints on cosmological parameters, the community has increasingly adopted neural network–based summarization techniques. While any neural network can be trained on the regression task of inferring parameters given data \citep[e.g.,][]{Gupta2018MAE, kacprzak2022deeplss, lu2023cosmological}, it is unclear how much of the information contained in the data is extracted. In particular, \cite{neural_summary_lanzieri_2025} demonstrate that standard regression loss functions do not guarantee the systematic construction of sufficient statistics. \Acrlongpl{imnn} (\acrshortpl{imnn}; \citealp{2018PhRvD..97h3004C}) directly address this problem by learning summary functions that maximize the Fisher information. They can produce nearly exact posteriors and are thus approximately sufficient statistics of the data. Another approach is to derive a loss function directly from the definition of sufficiency, i.e., by maximizing the mutual information between the summary statistics and parameters of interest \citep{Jeffrey:2021, chen2021neural}.


\paragraph{Controlling Epistemic Errors in Inference Results} One fundamental limitation of NDE methods is that their reliance on a neural network to model at some level the likelihood of the data is inherently imperfect. In the asymptotic regime of infinite data and flexible neural network, the approximation to the target posterior will converge, but in practice we are never guaranteed to find ourselves in this regime, and must therefore take into account and mitigate \textit{epistemic errors}. Several strategies have been developed over the years to quantify and mitigate this epistemic uncertainty on inference results.  \acrshort{mcmc} sampling over network parameters provides gold-standard uncertainty estimates but at usually prohibitive computational cost \citep[e.g.][]{2025arXiv251025824B}. In addition, detecting convergence of the chain remains difficult, usually necessitating drawing more samples than ultimately needed. Because such approach is extremely expensive, other approaches have been developed. \textit{\acrshortpl{bnn}} approximate the posterior distribution over network parameters through \acrshort{vi}, providing principled uncertainty estimates at reduced computational cost. However, the tradeoff between approximation quality and speed remains concerning, especially in the highly multi-modal loss landscapes of deep neural networks. In such settings, scalable variational methods often collapse to a single mode of the posterior rather than exploring the full diversity of solutions \citep{fort2020losslandscape}, which limits the quality of their uncertainty estimates. One of the most used VI methods in astronomy is Monte Carlo dropout which utilizes the dropout layer commonly introduced in deep neural networks to prevent correlated activation as one of the computationally cheapest approximations to Bayesian inference; however, its theoretical justification and empirical accuracy are questionable \citep{LeFolg:2021}. Nonetheless, comparisons with other methods have shown promise for astronomical applications: strong lensing \citep{Perreault:2017}, supernova time-series classification \citep{Moller:2020,Moller:2022ICML}, and star time-series classification \citep{Astromer2023, CadizLeyton:2025, CadizLeyton:2025:MoE}. Other VI methods such as Bayes by Backprop and SWAG have been sparsely used for time-series classification and regression with mixed results \citep{Moller:2020,Cranmer:2021}. Another strategy is Deep
Ensembles in which multiple networks are trained from different random initializations to provide uncertainty estimates via the variance of their predictions \citep{Makinen:2021,Moller:2022,Moller:2024}. Unlike variational methods, ensembles capture uncertainty by effectively sampling from different modes of the loss landscape, resulting in more robust and better-calibrated uncertainty estimates. However, they are more computationally demanding than MC dropout, and do not mitigate errors arising from model misspecification. Comparative studies exploring the trade-offs between these \acrshort{acr:uq} methodologies for achieving superior uncertainty evaluation are a growing focus in the field \citep{CadizLeyton:2025}.


\subsubsection{Model Mispecification and Covariate Shifts}
\label{sec4:model-misspec}

\begin{ThemeBoxA}
\themebullet \themekey{Related Methodologies.} \meth{sbi} \\
\themebullet \themekey{Addresses.} \challenge{covariate-shift}
\end{ThemeBoxA}

From a technical standpoint, \acrshort{acr:sbi} has achieved impressive results: \acrshort{acr:npe} with normalizing flows performs well in low-dimensional regimes~\citep[e.g.,][]{srinivasan2025}, \acrshort{nle} scales satisfactorily to higher dimensions, and marginal \acrshort{nre} has shown success across diverse applications~\citep[e.g.,][]{alvey2024,list2023,Franco_Abell_n_2024,Saxena_2024}. Yet, these demonstrations rely primarily on simulated data and therefore represent best-case scenarios; direct validations of SBI on real data remain scarce~\citep{2024Karchev_SIDEreal,Lueber2025}.

The robustness of SBI depends critically on the realism and completeness of the simulations that underpin it. Simulations must reproduce all relevant aspects of the observations, including astrophysical, instrumental, and observational effects. Any unmodeled process leads to domain shift or model misspecification, which can severely bias inference. This is particularly problematic for NRE, which relies on accurate joint modeling of data and parameters~\citep{filipp25}, while NLE is comparatively more interpretable since it operates directly in data space. Even when the theoretical model is sound, the observational and noise models must be equally faithful—a condition often unmet given the traditional divide between theoretical and observational cosmology. Bridging this gap is essential for SBI to succeed. Efforts are underway to diagnose and quantify model misspecification through simulation-based calibration and related approaches~\citep{2018arXiv180406788T,2023PMLR..20219256L,CoLT,montel2025,kelly2025simulation}. While model misspecification is a key vulnerability of SBI, it is not fundamentally different (if more difficult to diagnose and cure) than the similar risk incurred when using explicit, likelihood-based models: missing components of the model w.r.t. the true data-generating process will lead to potentially severe bias in the resulting inference. SBI is a relatively new technique, and therefore appropriate diagnostic tools are still being developed to ensure its robustness and reliability. 

\paragraph{Training Set Representativity}
A fundamental challenge across deep learning applications in astronomy is the representativity of training data. Models trained on simulations may fail to generalize to real observations, while those trained on current surveys may struggle with the deeper, higher-resolution \acrshort{lsst} data. Techniques such as fake source injection—embedding simulated objects into real images—can mitigate these gaps~\citep{2016MNRAS.457..786S,2018PASJ...70S...6H}, though their success depends on how realistic the injected sources are. The problem is particularly acute for rare phenomena, where training examples are intrinsically limited. To improve generalization, domain adaptation, transfer learning, and hierarchical Bayesian methods are being explored. E.g., \cite{Swierz2024} use domain adaptation to obtain more robust data summaries that can generalize well between simulated data and mock observations, enabling more accurate neural density estimation. Principled approaches such as stratified learning (discussed in Section~\ref{sec3:photo-z}) can mitigate covariate shift with little modification in the learning procedure, but other methods often require substantial experimentation and modifications of training procedures~\citep{2025arXiv251019168K}. Such corrections must themselves be treated as part of the inference pipeline and undergo rigorous calibration and uncertainty quantification. 


\paragraph{Physics Hardening}
When available datasets are incomplete or non-representative, physics-informed augmentation can enhance robustness. For example, the \acrshort{desc} \acrshort{elasticc} challenge~\citep{knop2023} injected transients simulated using semi-analytic models (e.g., \acrshort{sne}, \acrshort{kne}) to make classifiers more resilient to underrepresented classes, and \cite{Moskowitz2024} augmented spectroscopically-incomplete training samples with simulated photometry to improve photometric redshift estimation. Latent representations derived from \acrshort{sps} models can also be used to generate synthetic photometry for missing or incomplete observations~\citep[e.g., \href{https://github.com/Cosmo-Pop/pop-cosmos}{\tt pop-cosmos}][]{Alsing:2024,Thorp:2025,Deger:2025}, and facilitate comparisons with hydrodynamical simulations without observational systematics. However, these methods inherit the assumptions and uncertainties of the underlying theoretical models—such as uncertain nebular emission strengths in SPS~\citep{Byler2017_Nebular,Li2025_cue,Morisset2025_Nebular,2025arXiv250103133N}—which can themselves introduce model misspecification~\citep{Leistedt2023,Jespersen2025_opticalIR}. Addressing these limitations requires deeper astrophysical modeling of galaxy formation and evolution, as well as diagnostic tools for identifying misspecification in high-dimensional generative models~\citep[e.g.,][]{Thorp:2025:QQ}. Because physics-informed generative models (e.g., those that capture information within SPS parameterizations) can be used to synthesize observables that the model has not been trained on, such models can be validated not only against unseen data from a test set but also on new types of observations and other surveys \citep[e.g.,][]{Alsing2023, Alsing:2024, Thorp:2024, Thorp:2025, Deger:2025}. 


\subsubsection{Validating Inference Results}
\label{sec4:validation}

\begin{ThemeBoxA}
\themebullet \themekey{Addresses.} \challenge{uq}, \challenge{metrics}
\end{ThemeBoxA}

While the quality of neural posteriors can be tested \citep{2018arXiv180406788T, 2023PMLR..20219256L}, and while statistical tests can be performed to determine the probability that the distribution learned by a generative model is identical to that of the training data \citep{PQMass}, an open issue is the determination and procurement of a sufficient volume of training data for those tests to be sufficiently sensitive and for the learned distribution to be accurate.

Models trained on the same data but with different algorithms exhibit distinct probability calibration characteristics that must be evaluated and corrected. Similarly, identical algorithms trained on different training sets require independent calibration assessment. Common diagnostics include reliability diagrams used in time-series classification \citep{Moller:2020} or non-conformity scores from conformal inference techniques \citep{Xie:2025}, both of which compare predicted probabilities against observed frequencies. Detection of anomalies, i.e., the classification whether a signal is anomalous enough to be reported, is particularly vulnerable because it probes the tails of a learned distribution.  For regression tasks, calibration ensures that predicted uncertainties accurately capture the true error distribution. Poorly calibrated uncertainties can introduce systematic biases in downstream cosmological analyses, leading to incorrect parameter constraints. Recalibration methods for lens modeling are presented by, e.g.,  \citet{Perreault:2017}, \citet{Karchev2022GP}, and \citet{gentile23}. Regarding generative models, \cite{Campagne_2025} propose a “two-models” framework to evaluate their statistical consistencies trained on independent subsets of galaxy images. The results emphasize the need for large-enough datasets to enable calibration and validation strategies specific to each generative architecture (e.g., generative adversarial networks, normalizing flows and score-based diffusion), since apparent visual quality and morphological variable distributions alone do not guarantee statistical reliability.


\subsection{Physics-Informed Approaches}
\label{sec4:physics-informed}

From a high-level point of view, neural networks are never perfectly trustworthy and are often hardly interpretable. This motivates a general desire to build \textit{physics-informed} models, which can leverage as many explicit physical constraints as possible, thus limiting the potential failure modes of \acrshort{ai} components. 

\subsubsection{Hybridization of Generative Modeling and Physical Models}
\label{sec4:hybrid-gen-phys}

\begin{ThemeBoxA}
\themebullet \themekey{Related Methodologies.} \meth{normalizing-flow}, \meth{diffusion-model}, \meth{neural-surrogate}, \meth{differentiable-programming} \\
\themebullet \themekey{Addresses.} \challenge{covariate-shift}, \challenge{scalability}
\end{ThemeBoxA}

A promising direction for scientific inference is the hybridization of explicit, physics-based models with generative components that flexibly represent unknown or intractable distributions. In this framework, flow-based or diffusion models serve as probabilistic priors over complex latent variables, such as galaxy morphology or small-scale baryonic processes, while the rest of the model remains physically interpretable and simulation-driven. These generative priors have already proven powerful in astronomical inference, e.g., in the estimation of galaxy properties and photometric redshifts at scale (e.g.,  \href{https://github.com/Cosmo-Pop/pop-cosmos}{\tt pop-cosmos}; \citealp{Alsing:2024, Thorp:2024, Thorp:2025, Deger:2025}), and in the generation of high-fidelity, field-level H\,{\sc i} maps from dark matter simulations~\citep{Mishra:2025}. More broadly, generative models naturally support amortized inference frameworks, where neural posterior estimators are trained on samples from the generative process, enabling accurate Bayesian inference without \acrshort{mcmc} sampling but at the cost of greater upfront training effort.

\paragraph{Differentiable Programming} To fully integrate such probabilistic components with physical simulations, modern astrophysical codes are increasingly being reimplemented in automatic-differentiation libraries. Differentiable simulators eliminate the approximation errors of emulators and provide exact gradient information for optimization and uncertainty quantification. Examples include differentiable particle-mesh cosmological solvers~\citep{FlowPM,DISCO-DJ}, theoretical cosmology computations in \href{https://github.com/DifferentiableUniverseInitiative/jax_cosmo}{\texttt{jax-cosmo}}~\citep{JAX-COSMO}, galaxy–halo connection models in \href{https://github.com/ArgonneCPAC/diffsky}{\texttt{Diffsky}}/\href{https://github.com/ArgonneCPAC/diffstar/}{\texttt{Diffstar}}~\citep{Diffstar}, stellar population synthesis in \href{https://github.com/ArgonneCPAC/dsps/}{\texttt{DSPS}}~\citep{DSPS}, halo-model calculations in \href{https://github.com/fkeruzore/halox}{\texttt{halox}}~\citep{halox}, and differentiable image simulations in \href{https://github.com/GalSim-developers/JAX-GalSim}{\texttt{JAX-GalSim}}~\citep{jaxgalsim}. 

While the widely used \href{https://github.com/GalSim-developers/GalSim}{\texttt{GalSim}} library~\citep{ROWE2015121} produces realistic galaxy images, its non-differentiable design limits efficient gradient-based inference. In contrast, the emerging \href{https://github.com/GalSim-developers/JAX-GalSim}{\texttt{JAX-GalSim}} library reimplements core \texttt{GalSim} functionalities in JAX, yielding \acrshort{gpu}-accelerated, fully differentiable forward models that enable direct gradient computation for both population-level and individual-level parameters. Similarly, \href{https://github.com/pmelchior/scarlet2}{\tt scarlet2} offers a JAX-based, differentiable framework for non-parametric source morphologies and blended scenes observed by multiple instruments. Both libraries support vectorized batch simulations, crucial for large-scale hierarchical inference, and allow gradients of differentiable likelihoods to be computed automatically for maximum-likelihood estimation, variational inference, or gradient-based MCMC.


\paragraph{High-Dimensional Inverse Problems with Explicit Likelihood and Data-Driven Priors} 
High-dimensional inverse problems are central to cosmology, from galaxy deblending and strong-lensing source reconstruction to recovering the dark matter field from noisy data. In these settings, the forward process, such as instrumental response, noise model, or lensing distortion, is well understood and can be encoded in an explicit likelihood. The underlying components, however, such as galaxy morphologies or the non-Gaussian dark matter structure, lack closed-form descriptions and require expressive statistical models. Generative models, such as diffusion models, can learn realistic priors from high-dimensional observations or simulations. Combining such data-driven priors with explicit likelihoods yields a principled framework: the prior enforces realistic structure, while the likelihood anchors inference to the data, even under low \acrshort{snr} conditions where fully amortized approaches may drift. Recent works have demonstrated this hybrid approach for galaxy source reconstruction, strong lens modeling, and superresolution \citep{adam2022posterior,Barco2025blindinversion,2025Adam_SBProfiles} and dark matter field inference \citep{remy2023}. Moreover, the presence of an explicit likelihood enables learning data-driven priors directly from observations, through iterative refinement using posterior samples \citep{rozet2024learning, Barco_2025}, and can even allow for the correction of model misspecification \citep{Payot2025}. A remaining challenge is efficient posterior sampling, as inference with diffusion priors entails solving \acrfullpl{ode}, which is computationally demanding, although it can be performed practically at scale \citep[see, e.g.,][]{Thorp:2024, Thorp:2025}.



\subsubsection{Imposing Consistency with Physical Equations and Symmetries}
\label{sec4:physics-constraints}

\begin{ThemeBoxA}
\themebullet \themekey{Related Methodologies.} \meth{physics-informed} \\
\themebullet \themekey{Addresses.} \challenge{covariate-shift}, \challenge{metrics}
\end{ThemeBoxA}
%\note{This section can be fleshed out further, it's a bit thin right now.}

For trustworthy results, we additionally demand that the \acrshort{ai}/\acrshort{ml} outputs at various stages of the pipeline satisfy null tests (e.g., B-modes in gravitational lensing or rho-statistics for \acrshort{psf} modeling, \citealp{10.1111/j.1365-2966.2010.16277.x}) or obey the laws of physics of the corresponding analysis component rather than merely report final results with high fidelity. The modularity of the pipelines and multi-scale nature of the phenomena asks for validations at every analysis stage. Crucially, our knowledge of physical relations (in the universe, in the atmosphere, in the instrument) permits a form of validation that is typically omitted or impossible in industry applications of AI/ML. This motivates research in areas such as invariant/equivariant representation learning and geometric learning, with possible interdisciplinary implications beyond the scope of \acrshort{desc} \citep{2025arXiv250902661F}. Furthermore, the use of symmetry-aware \acrfullpl{enn} could help with the extraction of more robustness features from the data and, in combination with domain adaptation, enable easier mitigation of covariate shifts~\citep{Pandya2025}. E.g., with \acrfullpl{pinn} one can, in principle, find solutions for explicitly specified differential equations if their optimization could be made more robust \citep{2024arXiv240201868R}. 

\subsection{Novelty Detection and Discovery}
\label{sec:discovery}

\begin{ThemeBoxA}
\themebullet \themekey{Related Methodologies.} \meth{anomaly-detection}, \meth{self-supervised}, \meth{active-learning} \\
\themebullet \themekey{Addresses.} \challenge{covariate-shift}, \challenge{data-sparsity}
\end{ThemeBoxA}
A central scientific promise of \acrshort{lsst} lies in its potential for unexpected discovery. Many now-fundamental astrophysical phenomena, such as strong lenses, fast radio bursts, and pulsars, as well as singular systems like the Bullet Cluster, were first identified as anomalies. With an anticipated catalog exceeding 20 billion galaxies and roughly 10 million alerts per night, however, detecting novel phenomena in LSST data represents an unprecedented challenge.

Generative models offer a powerful framework for unsupervised discovery by learning the ensemble properties of galaxy images, spectra, photometry, and time-domain behavior, and by enabling the detection of statistically anomalous signals without labeled training data~\citep{2023AJ....166...75L}. In time-series and high-energy astronomy, representation learning has already proven effective for discovery-oriented analyses~\citep{Dillmann2025,Song2025}. However, a common failure mode of generative models, where atypical signals appear highly typical \citep{2018arXiv181009136N}, will mean that true outliers may not be recognized, a loss if we seek to find them (e.g., rare SN types such as pair instability \acrshort{sne}) and a problem if they contaminate carefully selected samples used in high-precision cosmology (e.g.,unrecognized blends in shape catalogs, \citealp{Dawson16ublends}; or catastrophic outliers in \acrshort{photoz} estimates). 
More specifically, standard unsupervised methods often struggle in the dense and homogeneous latent spaces produced by deep representations~\citep[e.g.,][]{Baron2025,StarEmbed2025}. E.g., while \citet{AstronomalyDecals2024} successfully combined Zoobot (a foundation model discussed in \autoref{sec:foundation_models}) features with the \href{https://github.com/MichelleLochner/astronomaly}{\tt Astronomaly} framework~\citep{2021Lochner_Astronomaly} to identify new sources, \citet{ZoobotApplications2022} found that tailored anomaly-detection techniques were necessary even within the well-studied Galaxy Zoo dataset. This line of work culminated in \texttt{Astronomaly} \texttt{Protege}~\citep{Protege2025}, a general-purpose, active anomaly detection system optimized for exploration in deep latent spaces.

The emergence of deep learning and foundation models (\autoref{sec:foundation_models}) further elevates the importance of active learning, the tight integration of human expertise and machine-driven pattern recognition~\citep{Protege2025}. Self-supervised methods and foundation models promise the ability to generate rich, general-purpose representations, but expert oversight remains essential to interpret their outputs and assess scientific relevance. Automated systems may flag outliers or cluster data effectively, yet human judgment is required to determine which patterns constitute genuine discovery. As \acrshort{ai} systems evolve toward agentic operation (\autoref{sec:llm_agentic}), the collaboration between human and machine will become increasingly intertwined. Embedding active-learning capabilities directly within AI infrastructures will therefore be critical to enable rapid, scalable, and participatory scientific discovery—bridging expert analysis and citizen science within the LSST era.
\newpage
\section{Emerging Techniques}
\label{sec5:emerging_tech}


\subsection{Data Foundation Models}
\label{sec:foundation_models}
% \coordinator{Michelle Lochner}
% General intro FMs
% Motivation for FM for DESC
%    - Pre-training and reusability
%    - Opportunities offered by multimodality
% Technical considerations
%    - Neural Architectures 
%    - Training objectives 
% Evaluation 

Foundation models, AI systems trained on massive datasets to perform a broad spectrum of tasks \citep{Bommasani2021FoundationModels}, have not only revolutionized AI research but are also rapidly reshaping modern life. Vision foundation models are now enabling breakthroughs in robotics \citep{RobotApplication2024}, medical diagnostics \citep{MedicalApplication2024}, and remote sensing \citep{RemoteSensing2024}. Beyond these applications, their adoption in scientific disciplines like genetics \citep{Evo2025} and heliophysics \citep{Surya2025} has highlighted their powerful predictive capabilities and their capacity to uncover fundamental relationships in complex data. The burgeoning field of foundation models presents a significant opportunity to enable and accelerate cutting-edge astrophysics.

Zoobot \citep{Zoobot2022} can be considered the first vision foundation model in optical astronomy. Having been trained on labels from the Galaxy Zoo citizen science project \citep{GalaxyZoo2008, GalaxyZoo2011, GalaxyZoo2013}, it has demonstrated versatility on a range of downstream tasks. These include broad morphological classification problems critical for studies of galaxy evolution, such as identifying merging galaxies \citep[e.g.,][]{MergerChallenge2024, MergerSims2025, MergersHSC2023}, and anomaly detection for finding rare phenomena like strong lenses \citep{walmsley25, lines25}. Formally released in \citet{ZoobotRelease2023}, Zoobot has been adapted to imaging data from multiple surveys, including DESI \citep{DECALS2019}, Euclid \citep{ZoobotEuclid2024}, HST \citep{ZoobotHST2023}, and JWST simulations \citep{ZoobotJWST2024}, among others \citep{ZoobotGAMA2024, ZoobotUNIONS2025, ZoobotHSC2025}.

Foundation models have also been designed for astronomical time-series datasets, spurred by the need for automated photometric classification of Galactic and extragalactic transients in the Rubin era. Some examples of these models include Astromer \citep{Astromer2023, Astromer2025}, ASTROCO \citep{Astroco2025}, ATAT \citep{ATAT2024}, FALCO \citep{FALCO2025}, and ROMAE \citep{ROMAE2025}. While the primary requirement for these models is high classification accuracy, they also enable the discovery of new classes of transients through anomaly detection, and, in some cases, provide lightcurve interpolation for further downstream analysis.

Despite early successes in modality-specific foundation-models, many astrophysical questions can only be answered by fusing different data types. This is a task where traditional methods, which rely on reducing data to summary statistics, are quickly being outpaced by powerful multimodal foundation models. Though the ideal neural architectures and training objectives for these multi-modal models are areas of active research, models have now been developed for galaxies \citep{Astroclip2024}, supernovae \citep{2024Zhang_Maven}, variable and non-variable stars \citep{2024Leung_stellarfm,AstroM2025}, and cosmological simulations \citep{MOSAIC2025}. 

\subsubsection{Foundation Models for DESC Science}
\paragraph{Pre-training and reusability} 
Foundation models offer a significant advantage over existing techniques by reducing heterogeneous data types into a unified and simplified numerical representation known as a latent space, representation or feature space. Instead of reprocessing a full dataset for each analysis, a single, powerful foundation model can generate rich data representations once. These representations can then be used directly or rapidly fine-tuned for numerous specific science cases, resulting in significant savings of computing resources. 

This shared representational basis also changes how DESC can approach cosmological inference. Traditional analyses reduce complex image data to limited summary statistics (e.g., moments, colors, or flux ratios), inevitably discarding information. Deep learning enables direct inference from raw observations, but training bespoke networks for each task across LSST’s petabyte-scale archive is infeasible. A general-purpose foundation model, trained once at scale, can thus act as a reusable ``backbone'' for all DESC pipelines, propagating consistent representations across tasks. 

For time-domain astronomy, the representations produced by foundation models are especially powerful. Because models learn features directly from the data, they are not constrained by the a priori assumptions of human-engineered features. This makes them ideal for identifying new or unexpected classes of objects via anomaly and novelty detection. Furthermore, these representations could be highly effective for standard tasks like early transient classification, which is critical for triggering spectroscopic follow-up.

For simulation-based inference (\autoref{sec4:sbi}), foundation model latent spaces can act as highly flexible encoders, providing a more powerful data compression than traditional statistical summaries. In terms of data handling, multimodal models can naturally accommodate missing data when fusing datasets. Moreover, the ability to pre-train on vast unlabeled datasets allows foundation models to help solve the representativity problem (dataset shift) often encountered between training and test sets in supervised learning.

\paragraph{Opportunities offered by multimodality}
 In astronomy, multi-wavelength models learn a shared latent space from multiple observational wavelengths (e.g., optical and infrared) of the same object. Multimodal models extend this concept by integrating fundamentally different data types into a shared latent space, often through self-supervised learning techniques. While the specific AI architectures for combining heterogeneous datasets remain in active development, all approaches fundamentally treat different data types as complementary views of the same underlying astrophysical system. The resulting multimodal representations provide a computationally efficient and powerful framework for a broad spectrum of scientific analyses. Further, multimodal representation learning naturally supports the need in DESC for self-calibrated systematics: once the relationships among observables are learned jointly, systematics can be explicitly modeled and corrected for survey-wide.

In the time domain, joint modeling of photometric light curves and spectra provides a direct path to improving supernova cosmology and our understanding of explosion physics. A multimodal FM trained to predict the spectral properties of a type Ia supernova from irregular photometric sequences \citep[as has been demonstrated on synthetic data;][]{2025Shen_DitSNe,2025Shen_MMVAE} may recover physically meaningful features (line velocities or continuum temperatures) that are otherwise inaccessible from broadband imaging data alone. These inferred spectra can serve as additional standardization parameters for type Ia supernovae, potentially reducing residuals in the Hubble diagram by incorporating information linked to, e.g., progenitor diversity \citep{2025Son_AgeBias} or host metallicity \citep{2013Childress_Metallicity}.

Beyond improving standardization, the same cross-modal embeddings enable proactive discovery. By comparing inferred to obtained spectra, outliers can be flagged for long-term monitoring. When combined with host-galaxy properties across the transient samples discovered by Rubin LSST, these models can provide a probabilistic mapping between host galaxy and transient properties, useful for exploring population-level correlations potentially linked to supernova physics and for obtaining sub-populations of highly-standardizable type Ia supernovae.

\paragraph{Challenges}
Despite rapid progress, technical and practical challenges must be addressed before multimodal FMs can be fully integrated into DESC pipelines. Propagating observational uncertainties to all studies conducted downstream of a DESC-wide FM is critical, and the diversity of applications across spatial and temporal scales may not be well matched to a single architecture's inductive biases. Multiple application areas also require accounting for modality imbalances and missingness in extant samples, which may cause a model to over-weight existing data and under-represent rare but informative modalities. Further, domain expertise may be necessary for ensuring embeddings are not dominated by observational systematics (e.g., tracking issues, bright sky backgrounds). Some architectures have attempted to factorize instrumental and astrophysical contributions to observed data explicitly \citep[e.g.,][]{2025Audenaert_CausalFMs}, but more work is needed on this front. 

\subsubsection{Training Objectives}
Training objectives determine whether foundation models learn astrophysically meaningful structure or merely reproduce observational correlations. For DESC, these objectives must explicitly promote representations that encode physical invariants (e.g., morphology–redshift relations, color–temperature gradients) while remaining robust to the observational systematics and covariate shifts defined as calibratable in the SRD. Self-supervised learning (SSL) offers the most practical and scalable route toward this goal, as it enables the extraction of representations from vast unlabeled datasets without compromising the requirement for bias quantification and calibration in DESC.

\paragraph{Self-supervised learning}
SSL encompasses a range of approaches: reconstructive methods, such as autoencoders \citep{Autoencoders1993, Autoencoders2006}, learn to compress data into a low-dimensional bottleneck and then reconstruct the original input; contrastive learning \citep{Contrastive2020, NonContrastive2021} trains models to produce invariant representations for augmented versions of the same data point (e.g., zoomed or rotated); and predictive methods, often implemented via transformer architectures \citep{Transformers2017}, learn by predicting masked or omitted sections of data based on their surrounding context. The principal advantage of SSL is its scalability, allowing models to be trained on vast datasets without costly manual annotation. 

\paragraph{Generative approaches}
Beyond these paradigms, generative and diffusion-based models are also being adapted for self-supervised representation learning in the astronomical domain. Although originally designed for data synthesis, diffusion objectives \citep{2022Yang_Diffusion} can act as powerful denoisers and uncertainty estimators,  aligning with the DESC requirement that calibratable systematic errors remain subdominant to statistical uncertainties. The study of how to manage a good generative model in the context of galaxy image synthesis has been investigated for instance using this denoising capability of score-based diffusion models \citep{Campagne_2025}. Hybrid diffusion autoencoders \citep{2021Preechakul_DiffAEs} combine reconstructive and generative losses, yielding latent spaces that capture astrophysical variation while marginalizing over observational noise. Methods such as these will be critical for DESC to achieve unbiased shear and photometric-redshift inference with early Rubin data.

\paragraph{Unsupervised learning considerations}
As highlighted in \autoref{sec:discovery}, the vast LSST dataset will hold immense potential for scientific discoveries. Given this sheer scale, foundation models will be critical for creating powerful representations that enable anomaly detection, clustering, and similarity searches. However, a significant challenge remains: FMs are typically designed and optimized for supervised tasks, while their use for unsupervised applications is often an afterthought. Recent work by \citet{ZoobotApplications2022} and \citet{Protege2025} demonstrates this gap. They found that traditional anomaly detection methods fail when applied to the deep latent features learned by both supervised and self-supervised methods. This indicates a clear need for new research: new unsupervised methods compatible with these features must be developed (such as Astronomaly: Protege) and FMs must be optimized specifically for unsupervised discovery.

\subsubsection{Architectural Innovations}
Realizing the capabilities of foundation models for DESC science requires architectural and training advancements. Astronomical data presents unique barriers to large-scale training, including wide dynamic ranges from faint to bright objects, Poisson noise properties, multi-wavelength observations, and irregular sampling. These characteristics, combined with the science priorities of DESC, necessitate architectures that not only achieve optimal representational power on LSST data but also permit robust uncertainty propagation.

% \paragraph{Neural architectures}
%%Continuous neural fields (NeRFs, SIREN) offer a paradigm shift for representing astronomical observations as continuous functions rather than discrete pixels. For weak lensing, this could enable resolution-independent shape measurements by learning continuous representations of galaxy light profiles. DESC could leverage these architectures to naturally handle PSF convolution in function space rather than pixel space, potentially eliminating pixelization systematics entirely.

%State-space models (S4, Mamba) provide linear-time sequence modeling with unbounded context, ideal for LSST's decade-long light curves. In contrast to transformers’ quadratic attention cost, SSMs maintain long-range temporal memory while preserving physical interpretability via explicit time constants and impulse responses—critical for capturing slow-evolving variability or secular evolution across the full LSST survey baseline.

%%% Old version
% Efficient attention mechanisms have matured significantly beyond standard transformers toward architectures that can scale to LSST-level data volumes. Flash Attention and its variants reduce memory footprint by 10-100x through tiling and kernel fusion, making CCD- or raft-scale attention suddenly feasible. Sparse attention patterns (Longformer's sliding window + global tokens, BigBird's random + window + global) could naturally map to astronomical hierarchies, such as by imposing local attention for nearby galaxies and global tokens for cluster properties. These designs directly support DESC’s need to jointly model small- and large-scale correlations that may affect shear, clustering, and supernova systematics.

% Hierarchical Vision Transformers (e.g., Swin Transformer V2) represent an especially promising class for DESC imaging applications. Their multi-resolution attention windows mirror the multi-scale nature of cosmological information, from pixel-level PSF modeling to large-scale galaxy clustering. Such architectures could unify weak-lensing and large-scale-structure analyses under a shared image encoder, enabling end-to-end uncertainty propagation across spatial scales as a direct step toward DESC’s requirement for cross-probe consistency in systematic control.

% Finally, architectures supporting multimodal data fusion are essential for LSST-scale inference. Early-fusion models \citep[e.g., Chameleon][]{Chameleon2024} integrate multiple data types during training, while late-fusion approaches merge specialized encoders post hoc \citep{2023Pereira_LateFusion}. Architectures like the Perceiver family \citep{2021Jaegle_Perceiver,2021Jaegle_PerceiverIO} generalize these ideas further by learning compressed latent arrays that can flexibly accommodate new data modalities. This is an operational advantage for DESC, where evolving survey conditions and ancillary datasets (DESI, Roman, 4MOST/TiDES) demand continuous integration into the shared analysis space.

\paragraph{Attention}
Efficient attention mechanisms, which allow models to weigh the importance of different parts of the input data, have evolved significantly beyond standard transformers, offering new architectures that can scale to LSST-level data volumes. Methods like Flash Attention \citep{FlashAttention2022} use techniques such as tiling and kernel fusion to reduce the memory footprint by 10-–100$\times$. This significant reduction for the first time makes practical application of attention mechanisms at the scale of individual CCDs or rafts. 

\paragraph{Hierarchical Approaches}
Astronomical data is inherently hierarchical, with structures ranging from individual galaxies to massive clusters. Sparse attention models like Longformer \citep{Longformer2020} and BigBird \citep{BigBird2020} are well-suited to this, as their architecture directly mirrors this physical structure. They use local attention (e.g., sliding windows) to model interactions between nearby objects, while global tokens aggregate information about the entire system. These designs are critical for DESC, as they enable the joint modeling of small- and large-scale correlations essential for controlling systematics in shear, clustering, and supernova analyses. This hierarchical approach is well-developed in vision models, with Hierarchical Vision Transformers \citep[e.g., Swin Transformer V2,][]{Swin2022} being especially promising for DESC imaging. Unlike the token patterns in sparse models, these architectures use multi-resolution attention windows that mirror the multi-scale nature of cosmological information, enabling pixel-level PSF modeling up to large-scale galaxy clustering. This design could allow for the unification of weak lensing and large-scale structure analyses under a shared image encoder. Such a model would facilitate end-to-end uncertainty propagation across spatial scales, directly addressing the DESC requirement for cross-probe consistency in systematic control.

\paragraph{Mixture-of-Experts}
The principle of a shared, unifying backbone extends to Mixture-of-Experts (MoE) architectures, which offer natural alignment with DESC objectives. Rather than training independent networks for each object class or redshift regime, sparse MoE layers -- such as the Switch Transformer \citep{Switch2022} and Mixtral \citep{Mixtral2024} -- can dynamically route inputs through specialized sub-networks while preserving a common latent backbone. This paradigm mirrors the DESC software model itself: probe-specific inference modules built atop a common analysis infrastructure. For example, expert sub-networks could specialize in quiescent versus star-forming galaxies or early- versus late-time transients, while shared latent features will ensure cross-consistency in calibration and selection functions across working groups.

\paragraph{Data Fusion}
A final architectural requirement is multimodal data fusion, driven by the operational need to combine LSST data with a continuous stream of ancillary datasets (e.g., DESI, Roman, 4MOST/TiDES) as well as managing evolving observing conditions. This challenge can be addressed with early-fusion models, which integrate data types during training \citep[e.g., Chameleon][]{Chameleon2024}, or late-fusion approaches that merge specialized encoders post hoc \citep{2023Pereira_LateFusion}. More generalized architectures, like the Perceiver family \citep{2021Jaegle_Perceiver, 2021Jaegle_PerceiverIO}, are particularly advantageous. By learning a single, compressed latent array, they are explicitly designed to flexibly accommodate new data modalities. This provides a clear operational path for DESC to continuously update and expand its shared analysis space.

Together, these architectural innovations suggest a coherent design philosophy for DESC foundation models: hierarchical representations that preserve cosmological structure, modular routing across scientific workflows, and learned compression layers capable of cross-probe alignment. Each of these desiderata drive toward reproducible, uncertainty-aware analyses within a unified DESC software framework.



\subsubsection{Evaluation}
FMs promise a significant leap in efficiency, enabling generalization across tasks and reduction of the research time and computational resources required for bespoke models. However, this versatility carries a critical risk: FMs can inherit and propagate biases from their training data. This risk is particularly important in cosmology, where rigorous evaluation of bias and uncertainty is non-negotiable. Therefore, establishing a common framework of benchmarks to validate FMs for sensitive scientific applications must be a research priority, as this is not standard practice in many industry-developed methods.

\paragraph{Benchmarks}
Any DESC methodology employing a FM should be accompanied by a comprehensive suite of benchmarks designed to evaluate:
\begin{itemize}
\item Predictive performance across representative science tasks;
\item Calibration of uncertainties and probabilistic outputs;
\item Robustness of algorithmic behavior to survey systematics;
\item Sensitivity to biases introduced by training data or simulation assumptions
\end{itemize}

Beyond standard metrics, DESC evaluation metrics should be compositional, assessing performance hierarchically across the scientific workflow. Benchmarks should test low-level operations such as deblending and denoising, intermediate analyses including morphology or photometry, high-level physical inference such as redshift or stellar mass estimation, and final cosmological parameter recovery. This multi-level structure would identify precisely which stages contribute the largest uncertainty or bias to the dark energy equation-of-state constraint and guide model retraining or correction.

\paragraph{Interpretability}
Mechanistic interpretability will provide a complementary route to validation. Techniques such as attention visualization, activation clustering, or sparse dictionary learning can be adapted to determine whether internal model representations recover known astrophysical relations including the color–magnitude diagram, the fundamental plane, or the Tully–Fisher relation. Developing astronomy-specific interpretability tools would enable DESC to quantify whether FMs encode physically meaningful structure or merely reproduce empirical correlations.

\paragraph{Distribution shift}
Evaluation under distribution shift is also essential for robustness. DESC models must maintain reliability under temporal drift across survey years, spatial variation in observing conditions, and transfer to external datasets such as Euclid or Roman. Dedicated stress tests should replace the assumption of IID validation, using importance-weighted calibration errors and worst-group accuracy measures to reveal biases that emerge only under covariate change. These tests will be crucial for ensuring that DESC FMs remain stable as LSST transitions from early to full survey operations.

\paragraph{Long-term impact}
Finally, the deployment of large-scale models must consider sustainability and community governance. Training and fine-tuning FMs require substantial computational and environmental resources, underscoring the need for shared development, standardized documentation, and reproducible training pipelines. Strategic coordination within DESC and with external collaborations will ensure that these models serve as transparent, scientifically verifiable, and environmentally responsible assets for the next generation of cosmological analysis.

\subsection{Large Language Models \& Agentic AI}
% \coordinator{Clecio R. Bom}
\label{sec:llm_agentic}
%1. Large Language Models for DESC

 %  1.1 Overview
 %      - Foundation model capabilities and relevance to DESC workflows
   
  % 1.2 Desiderata for successful AI/LLM integration in DESC:
   
%       - Facilitated access to data and analysis / Education and onboarding
%         • Natural language interfaces to databases (e.g., ChatGaia)
 %        • Infrastructure training and documentation
 %        • Dedicated RAG for all software and technical documentation
       
%       - Proactive agents for data inspection and knowledge synthesis
 %        • Automated data quality assessment
%         • Cross-dataset analysis and anomaly detection
       
%       - Literature search and research assistance
%         • Paper synthesis and knowledge management
       
 %      - Automated scientist workflows
 %        • Domain-specific agents (e.g., Denario, CMBAgent)
 %        • Integration of specialized knowledge with data and compute access
       
 %      - Efficient optimization of limited expert feedback
 %        • Human-in-the-loop strategies for collaboration-wide learning
 %        • Connection to foundation model development
   
 %  1.3 Implementation Considerations
 %      - Risk-benefit analysis (accuracy, privacy, computational costs)
 %      - DESC-specific benchmarks leveraging unique data challenges

Large language models (LLMs) and multi-agent systems (MAS) represent a new class of foundation models capable of performing high-level reasoning, hypothesis generation, and workflow orchestration across scientific domains. In astronomy, these systems extend beyond perception and representation to enable dynamic reasoning over code, data, and documentation. Their emergence aligns directly with DESC’s long-term objective to establish reproducible and efficient analysis frameworks that minimize human feedback while preserving scientific validity.

\subsubsection{Overview}

Recent works suggest  that LLMs can coordinate end-to-end scientific workflows. \citet{laverick2024mas} implemented an LLM-driven multi-agent system for cosmological parameter estimation from the Atacama Cosmology Telescope (ACT), integrating retrieval-augmented generation (RAG) with local code execution. Distinct agents that serve as, e.g., ``manager,'' ``coder,'' ``experiment-RAG,'' and ``software-RAG'', collaboratively retrieved literature, executed Monte Carlo Markov Chain pipelines, and synthesized posterior constraints, converting what was once a labor-intensive workflow into an auditable, agent-mediated loop of experimentation and inference.

This capacity for orchestration reflects a broader transition from model-centric AI to \textit{agentic science}, where autonomous systems plan, reason, and execute experiments with minimal supervision \citep{zhou2025autonomous, xu2025opensourceplanning, CMBAGENT_2025}. However, automation introduces epistemic risks. As emphasized by \citet{ilievski2025aligning}, human scientific generalization relies on abstraction, causal reasoning, and conceptual transfer, whereas current AI systems often depend on statistical interpolation and distributional heuristics. For DESC—where analyses frequently extrapolate beyond training regimes, from rare transient classification to high-redshift inference—agents that merely interpolate can produce superficially plausible yet scientifically invalid results. The design of agentic frameworks must therefore prioritize conceptual alignment, emphasizing reasoning grounded in physical principles, provenance tracking, and transparent uncertainty propagation \citep{wu2023autogen, ilievski2025aligning}.

\bigskip
\noindent{\bf Scientific Potential.}
By integrating such systems within DESC’s data pipelines, agents can automatically test calibration procedures, generate human-readable explanations for results, and suggest retraining strategies. These agents have the potential to act as intelligent research assistants: executing, validating, and interpreting analyses under human oversight. Their long-term promise lies not only in efficiency processing of existing workflows but in augmenting scientific reasoning, translating between natural language, code, and physical abstraction.

\subsubsection{Desiderata for Successful AI/LLM Integration in DESC}

\paragraph{Facilitated access to data and analysis.}
Natural-language interfaces built atop LLMs can dramatically reduce the barrier to accessing DESC’s simulation and analysis infrastructure. RAG-based systems \citep{lewis2020rag,fan2024survey} allow LLMs to remain grounded in up-to-date, domain-specific knowledge while  mitigating %(JEC) : \sout{avoiding}, by experience, hallucination remains a pb in RAG and one should build hallucination detection system in paralell which is an endless. If you do not agree keep "avoiding" word....
hallucinated results. Systems such as ChatGaia demonstrate that conversational querying of stellar databases and catalog crossmatches is feasible at scale. For DESC, an analogous interface could provide seamless access to Rubin/LSST data products, DESC documentation, and simulation catalogs, enabling new members to interact with complex pipelines through natural language. This directly supports collaboration-wide education, onboarding, and reproducibility.

\paragraph{Proactive agents for data inspection and knowledge synthesis.}
Multi-agent Systems frameworks permit agents to monitor and diagnose data streams in real time. Using local embeddings of image, catalog, and telemetry data, agents could autonomously flag photometric outliers, cross-dataset inconsistencies, or nightly calibration drifts. Comparable architectures are now emerging in other fields: e.g., agentic systems for real-time climate monitoring and autonomous laboratory instrumentation \citep{pierre2023ai, mandal2025evaluating}. In DESC, similar agents could manage nightly data-quality summaries, report unexpected deviations to experts, and generate RAG-enhanced explanations tied to the survey’s provenance logs.  

In parallel, LLMs are enabling large-scale knowledge synthesis. One example of this in astrophysics is Pathfinder \citep{iyer2024pathfinder}, which applies semantic retrieval and citation-aware summarization across $\sim$350,000 astrophysical papers in ADS, allowing users to move from keyword-centric searches to concept-level exploration. Embedded within DESC’s software ecosystem, such systems could dynamically summarize instrument performance trends, algorithm changes, or cross-probe findings, enabling fast, collaborative knowledge transfer.

\paragraph{Literature search and research assistance.}
Domain-specialized LLMs, such as AstroSage-Llama-3.1-8B \citep{dehaan2025astrosage}, demonstrate that compact, astronomy-tuned models can rival much larger general-purpose LLMs at a fraction of the computational cost. Integrating these models into DESC’s internal infrastructure could enable contextual question-answering and literature synthesis directly within analysis notebooks. For example, an LLM agent could parse DESC pipeline documentation, identify related publications, and draft code examples linked to verified repositories. By unifying human-readable documentation, code, and literature into a retrieval-augmented knowledge graph, DESC could maintain a continuously evolving, auditable record of methodological provenance that would not be possible from human effort alone.

\paragraph{Automated scientific workflows.}
MAS architectures such as Mephisto \citep{mephisto} and Denario \citep{Denario_2025} illustrate how LLM-driven agents can coordinate complex, multi-step analyses. In Mephisto, agents interpret multi-band galaxy observations, call the CIGALE astrophysical SED-fitting code \citep{cigale}, and iteratively refine stellar population parameters through feedback loops. Denario extends this paradigm to a full research lifecycle, spanning idea generation, data selection, modeling, interpretation, and manuscript drafting integrating reasoning, execution, and documentation within a closed agentic loop.  

For DESC, such frameworks could orchestrate photometric redshift validation, transient classification, or simulation–real comparisons. Each agent can perform specific tasks (data ingestion, model training, diagnostic visualization) under a shared governance layer to ensure that each action is logged, unit-tested, and reversible. The result is an ecosystem of automation without loss of traceability.

\paragraph{Efficient optimization of limited expert feedback.}
DESC’s scale makes manual supervision of all AI systems infeasible. Human-in-the-loop frameworks address this by using expert feedback to refine agentic models efficiently. Active-learning strategies allow LLMs to request clarification only for ambiguous cases, optimizing annotation throughput \citep{settles2009active, christiano2017feedback}. Such loops could guide automated anomaly detection or simulation-based inference pipelines, ensuring that model refinements are scientifically meaningful. Recent work evaluating RAG agents for astrophysical QA tasks \citep{hyk2025queries} suggests that combining expert oversight with structured retrieval logs improves reliability and transparency. Connecting these feedback mechanisms to DESC’s foundation-model development ensures that lessons from one analysis (e.g., transient classification) propagate across the collaboration’s full machine-learning stack.

\subsubsection{Implementation Considerations}

Integrating LLMs and MAS into DESC’s data ecosystem introduces both opportunity and risk. Benefits include improved reproducibility, faster analysis cycles, and democratized access to complex pipelines. Risks include propagation of biases, code-generation errors, data-privacy violations, and environmental cost.  
A risk–benefit analysis must quantify not only computational cost against the anticipated scientific yield but also epistemic reliability — how confidently can a model’s recommendation be traced, reproduced, and validated? Frameworks such as the \textit{ScienceBoard} benchmark \citep{scienceboard2025} now measure LLM scientific reasoning under domain shifts, providing a template for DESC-specific evaluations.

\paragraph{Evaluation and benchmarking.}
LLMs must be evaluated beyond text-generation metrics. DESC benchmarks should measure: (1) factual accuracy on domain literature; (2) code validity and runtime safety; (3) reproducibility under random seed variation; and (4) robustness to distribution shift and systematic missingness (early vs late LSST years, cross-survey transfers). Existing efforts in autonomous science evaluation, such as \textit{DiscoveryWorld} \citep{discoveryworld} and the AI Scientist benchmark \citep{xu_ai_scientist}, demonstrate compositional assessments of reasoning, hypothesis testing, and execution fidelity. Complementary to these general frameworks, \citet{ye2025replicationbench} introduce \textit{ReplicationBench}, an astrophysics-specific benchmark that evaluates whether AI agents can faithfully reproduce published research papers. By decomposing studies into author-validated tasks that test methodological adherence and quantitative accuracy, \textit{ReplicationBench} exposes persistent gaps in scientific reliability—critical context for assessing agent performance in DESC applications. DESC could adopt similar multi-level benchmarks to assess agentic performance across deblending, photometry, and cosmological inference tasks.

\paragraph{Governance and reproducibility.}
Governance mechanisms must accompany agentic integration. Each automated workflow should maintain full *provenance metadata*—including prompts, model version, retrieved sources, and execution logs. Sandboxed environments can enforce deterministic behavior and ensure that code emitted by agents is executed within secure, auditable contexts \citep{zhou2025autonomous}. Verifiable autonomy requires that every AI-derived result includes uncertainty estimates, validation metadata, and reproducibility tokens linking outputs to inputs. These measures echo the emerging norm of “executable transparency” in computational astrophysics.

\paragraph{Infrastructure integration.}
Finally, the physical deployment of LLMs within DESC must respect both scale and sustainability. Compact open-weight models fine-tuned on DESC’s public datasets could balance capability with energy efficiency. A dedicated RAG layer connecting DESC documentation, simulation archives, and survey data products would form the backbone of an agentic knowledge graph. By coupling this infrastructure to controlled APIs and authenticated data services, DESC can safely experiment with LLM-assisted reasoning under real survey conditions, transforming static documentation into an interactive scientific interface.

\bigskip
\noindent In summary, LLMs and MAS have the potential to become trusted collaborators in DESC’s scientific ecosystem. Their integration promises not merely faster pipelines but qualitatively new modes of discovery—where agents amplify, rather than replace, human scientific insight.
\newpage
\section{Infrastructure Requirements to Support AI/ML in DESC}
\label{sec6:infra_requirements}
% \coordinator{Adam Bolton}
%\sectioninstr{Summary of the computing, data management, and infrastructure needed to deploy/scale existing and emerging AI/ML across DESC. Addresses benchmarking standards and frameworks for reproducibility. Subsections: Computing Resources, Data Management and Access, Benchmarking and Reproducibility.}

Infrastructure is the shared technology needed to realize the methods, models, and scientific opportunities outlined in the preceding sections. It may be deployed in a distributed mode with individuals or small teams using local or institutional resources, or in a coordinated mode within a shared platform or environment. This latter mode is most relevant for the largest-scale \acrshort{ai}/\acrshort{ml}-enabled model training and inference workflows at the scale of the entire Rubin-\acrshort{lsst} data set, including via core infrastructure through \acrshort{doe}-funded high performance computing facilities in the US. The following subsections review the infrastructure elements most relevant to the future of AI/ML in \acrshort{desc}.

\subsection{Software}
% What are the important points we want to surface here?
%  - There is a notion of building up a robust software stack for scientific AI.
%      - What is critical to the success of a large organization like DESC is strong 
%        Leadership on adoptingindustry standard solutions for that software stack. With a focus
%        on ensuring that our stack will survive a few years (e.g. frameworks disappearing)
%      - Notion of coordination with supporting facilities like CC/IN2P3 or NERSC on making 
%        these choices and deploying them for DESC members
%  - Then there is a notion that increasingly, DESC analysis pipelines may start to rely on
%    AI components. 
%       - Foundation models served as a service for X
%       - Dedicated models integrated in analysis workflows
%       - SBI and such 
%       - All the way to integrating and/or enabling agents to interface with our systems

Software is foundational to modern cosmology, especially for \acrshort{desc}, where scientific insight increasingly depends on sophisticated computational workflows. As \acrshort{ai} and \acrshort{ml} mature into core scientific technologies, software itself becomes a form of research infrastructure. In this context, two tightly connected priorities emerge: first, the development and long-term stewardship of a robust AI software stack that enables model development and experimentation; second, the strategic integration of AI methods into the DESC scientific pipelines, where they will ultimately shape how we reduce data, extract measurements, and perform inference.

\subsubsection{The AI Software Stack} 

DESC has long demonstrated leadership in scientific software development: from collaboration-wide development guidelines\footnote{\url{https://lsstdesc.org/assets/pdf/docs/DESC_Coding_Guidelines_latest.pdf}}
 and software-oriented publication policies, to a culture of reproducibility and sustained support for collaboration-wide software stacks. As AI becomes a first-class component of scientific analysis, extending that same discipline and strategic thinking to the AI software ecosystem is increasingly important. The goal is to define a modern, durable, and interoperable stack built on best practices for reproducibility and maintainability. This accelerates research while preserving the transparency that has always characterized DESC software.
 
To meet this goal, we recommend converging on a small set of shared practices and services that make ML development reproducible, portable to the \acrfull{cc-in2p3} and \acrfull{nersc}, and sustainable over the 10-year duration of the LSST survey. Key components include:

\begin{itemize}
\item \textbf{Frameworks for model development}, likely PyTorch for large models and JAX for differentiable physics.
\item \textbf{Experiment tracking}, capturing code/data versions, configurations, metrics, and compute environments to ensure reproducibility.
\item \textbf{Model and artifact registries} to version and archive trained models, datasets, and reports, mirroring survey data-release practices.
\item \textbf{Standardized export formats} such as \acrfull{onnx} so models integrate cleanly with Rubin/DESC pipelines and \acrfull{hpc} environments.
\item \textbf{\Acrfull{ci} and \acrfull{cd} for models} to test and validate training configurations, exported artifacts, and deployment environments.
\item \textbf{Observability} and drift monitoring so ML components used in production behave predictably and transparently.
\end{itemize}

These elements are not overhead; they are the operational foundations that allow AI to become reliable scientific machinery rather than episodic experimentation. They also reduce long-term maintenance burden by enforcing shared conventions, minimizing bespoke tooling, and allowing learned models to be audited, reused, and trusted throughout the \acrshort{lsst} decade.

In addition to these considerations for ML model development, DESC will increasingly rely on \acrshortpl{llm} as flexible interfaces to data, documentation, and tooling. LLMs are unusual compared to conventional models in that they are supplied by a rapidly changing ecosystem of commercial providers, open models, and self-hosted deployments. Versions change frequently, and some use cases may require on-premises or institutionally-hosted models---e.g., at NERSC, CC-IN2P3, the \acrfull{csd3}, or similar facilities---via serving stacks such as vLLM for data-governance or cost reasons. To remain agile in this landscape, DESC should treat LLMs as interchangeable backends behind a stable internal \acrshort{api}. Such an abstraction layer enables swapping models without rewriting pipelines and moving workloads between commercial services and collaboration-owned \acrshort{gpu} resources as needs evolve. Beyond the LLMs themselves, agentic frameworks (libraries that orchestrate multi-step LLM workflows, tool calls, and human-in-the-loop interactions) are even more volatile, with new options appearing and disappearing on timescales of months. Frameworks such as LangGraph exemplify the current direction, representing agents as graphs of tools, memory, and control logic, and providing execution and tracing engines around them. 

Framework sustainability and openness should play an important role in guiding tooling choices. Solutions with strong governance and broad adoption (e.g., PyTorch under the Linux Foundation, ONNX, MLflow or self-hostable experiment-tracking systems) provide long-term stability and avoid dependence on proprietary silos. Finally, coordination with computing partners is essential to ensure portability of the software stack and deployment on more HPC-aligned facilities.

\subsubsection{Integration of AI/ML within Analysis Pipelines}
% PRevious content
% The space of “AI for software” is extensive and rapidly evolving. A non-exhaustive selection of topics of interest to DESC science is:
% \begin{itemize}
% \item Integrated AI support within notebook environments enabling rapid development and iteration of scripting software for data exploration and analysis.
% \item New approaches to data-reduction pipelines that optimize AI models directly against observational data rather than operating sequentially in a classical ``raw-to-reduced'' data pipeline workflow.
% \item Foundation Model-based implementations of X-as-a-service, where X is, for example, photometry, astrometry, cross-matching, classification, or anomaly detection.
% \item AI-based ``digital twin'' models for experimental apparatus that enable more precise and accurate calibration and control of systematics than traditional approaches.
% \item AI/ML-based cosmological inference frameworks (e.g., FMs, simulation-based inference, field-level inference)
% \item Use of specialized agents to complement DESC’s analyses by querying and exploring the heterogeneous landscape of astronomical data sets available across different repositories.
% \end{itemize}


As AI components mature, DESC pipelines may evolve from purely sequential “raw-to-reduced” workflows to systems where learned models are first-class, production-grade elements. The emphasis is on embedding AI in ways that enhance measurement fidelity, calibration control, and inference efficiency, while preserving transparency, reproducibility, and smooth integration with existing practices.

\paragraph{Data reduction pipelines} AI/ML is already present in DESC workflows (e.g., \acrshort{photoz} estimation within \acrshort{rail}). Over time, more components are likely to incorporate learned models at defined stages such as deblending, \acrshort{psf} estimation, artifact rejection, and photometric calibration, tightly coupled to Rubin/DESC data structures and HPC execution. In parallel, DESC has interest in end-to-end approaches that optimize models directly against observational data and simulators, potentially replacing brittle stage boundaries while maintaining provenance through experiment tracking and model registries.

\paragraph{Foundation Models as Services} With the advent of the foundation model paradigm, we envision the possibility of providing “X-as-a-service” (photometry, astrometry, cross-matching, classification, anomaly detection) behind stable APIs, so models can evolve without churn in downstream code.

\paragraph{AI/ML-based cosmological inference} AI-based methods are shifting inference from sampler-centric workflows toward models that learn mappings from data to posteriors or summaries. Foundation-style surrogates can amortize computation and reduce \acrshort{mcmc}-heavy workloads, while simulation-based inference trains directly on forward models and field-level methods operate on minimally reduced data. Active-learning loops may trigger simulations on the fly to refine surrogates and concentrate the HPC budget where it most reduces uncertainty. For pipeline use, these elements remain tied to provenance systems and model registries.

\paragraph{Shared infrastructure for emulators and surrogate models} Consistency across emulator efforts can lower reuse costs. A lightweight scaffold could include common interfaces (inputs, units, cosmology/observational context, stochastic controls, uncertainty outputs), a minimal dataset schema for training/evaluation, embedded metadata for provenance and validity domains, and containerized artifacts with dual exports (native framework and ONNX where feasible). A small validation suite (accuracy, coverage, calibration under shift, throughput) running in CI on facility images would help with portability to DESC computing facilities, and hooks for active learning can keep surrogates co-evolving with forward models.

\paragraph{LLMs and agents} Large language models and specialized agents can contribute at multiple levels of DESC work: in notebook environments as integrated assistants that accelerate iteration on analysis code, diagnostics, and documentation/query synthesis; in data discovery and curation by searching heterogeneous archives and DESC repositories, proposing cross-matches, and flagging anomalies; and at facility scale by assembling templated workflows, submitting authenticated jobs, and recording outcomes into experiment tracking and model registries. Adoption pairs capability with governance: standardized interfaces for serving models, authenticated access, audit logs of prompts and actions, and human-in-the-loop checkpoints for any decision with scientific impact.


Likely, investment in general-purpose AI tooling across industry, open-source, and public efforts will continue to exceed resources available for cosmology-specific software. When such tools meet DESC’s scientific and operational needs, adopting them can leverage broader community advances while allowing collaboration effort to focus on domain-specific modeling and validation. Finally, AI will also influence our approach to engaging with computing in the future. Computational infrastructure will increasingly be harnessed with the assistance of LLMs and agentic frameworks. Depending on how this transition proceeds, it may enable a larger community of researchers within DESC to engage directly with large-scale scientific computing.


\subsection{Computing}

\subsubsection{Workflows and Scales}

Estimating the full scale of resources needed to support the range of \acrshort{desc} \acrshort{ai}/\acrshort{ml} use cases is beyond the scope of this white paper, and any estimates will evolve with time as new methodologies and science applications are developed. Major resource categories include \acrshort{gpu} and \acrshort{cpu} time, short- and long-term storage, and network bandwidth both between and within analysis facilities. AI-oriented workloads will range from small-scale \acrfull{rd} to training and serving \acrshortpl{fm}, serving tokens for agents/\acrshortpl{llm}, running on-the-fly simulations for active learning loops for \acrshort{acr:sbi}, up to potentially running simulations on the fly as part of explicit inference loops. All of these will bring their own requirements for computing cycles, data locality and throughput, interactivity, and orchestration.

At the low end of requirements, resource-augmented instances of the commercial cloud-based \acrfull{rsp} would provide an accessible route for individuals and small teams to scale up AI/ML workflows that require integration with the Rubin data and software stack. For cost efficiency, these resources would likely need to be managed through either a batch-processing system or through an elastic data-science workflow system. Allocating GPU-based interactive servers (virtual or physical) in the cloud-based RSP context is unlikely to be viable at scale for many users, given the typical idle time in interactive sessions and workflows. Larger AI workflows could also be accommodated through individual or DESC-wide allocations of CPU, GPU, and working storage at \acrshort{nersc} or through proposal-driven allocations under the 10\% of computing time reserved for Rubin science users at the Rubin \acrfull{usdf} at \acrfull{slac}.

On the high end, significant computing resources  may be needed for new simulations to train simulation-based inference approaches to large-scale cosmology analyses. This includes not only the computing needed for simulation but also the storage to save and share large simulation outputs. Another major driver of resource needs would be training of data-oriented FMs at the scale of the full Rubin data set. Again, an accurate estimate of the resources needed for this would require further study of an appropriate reference architecture and training strategy. A rough guide to the scale can be found from the (CPU-based) compute sizing model for Rubin--\acrshort{lsst} Data Release Production\footnote{\url{https://dmtn-135.lsst.io}} since it is operating on roughly the same scale of data as a full-scale LSST-based FM would train on. The operations compute model estimates a need for about 50M core-hours in year 1 of the survey, rising linearly year over year (assuming annual data releases) to roughly 10x this amount in year 10. Storage needs are estimated to increase from 30PB in year 1 to over 800PB in year 10 although not all of the associated data products would necessarily be needed for FM training.

In addition to ``offline'' computing needs, a number of time-critical AI/ML-driven DESC workflows will be driven by the nightly alert stream, including ML-driven algorithms for classification, anomaly detection, and intelligent follow-up observations. Many of these will be implemented within the broker and marshal systems that receive the LSST alert stream; depending on their level of resource-intensiveness, broker/marshal computing capacity may need to be further augmented, and/or integrated with elastic or preemptive compute allocations within research computing facilities or in commercial cloud.

\subsubsection{Computing Resource Providers}

In the US, DESC computing needs are primarily supported by \acrshort{doe} through its network of computing facilities under the \acrfull{ascr} program within the DOE Office of Science. This includes NERSC as the primary user facility for DESC science analysis, \acrfull{esnet} for advanced wide-area networking and data movement, and the \acrshort{alcf} and \acrshort{olcf} for larger-scale computational work. Recognizing the need for a facility dedicated to data-intensive computing, ASCR is also supporting the early stage of development for a future \acrshort{hpdf}. DOE is also advancing the development of an \acrfull{iri} to support flexible, powerful, and accessible implementation of scientific workflows across all these facilities.

Anticipating an increasingly prominent role for AI in the scientific exploitation of data created by Rubin and other DOE-funded facilities across all disciplines, the US Congress has funded the creation of \acrshort{amsc}\footnote{\url{https://science.osti.gov/-/media/grants/pdf/lab-announcements/2025/LAB-25-3555.pdf}} to develop and deploy the next generation of AI-oriented capabilities on the foundation of the DOE computing platforms noted above. AmSC development is underway now, with funding distributed across an AmSC infrastructure component, a core AI model-development consortium, pilot funding for discipline-specific data curation and science benchmarking activities, and seed funding for discipline-specific AI model teams. While it is still in an early ramp-up phase, computing and AI model development infrastructure within the AmSC represent a significant opportunity to address ambitious DESC AI/ML goals.

The joint \acrshort{nsf}--DOE nature of Rubin Observatory opens the possibility of leveraging significant NSF-supported computing facility resources, especially if pursued in coordination with other Science Collaborations working in areas typically supported by NSF. These resources include the \acrfull{lccf} being constructed at the Texas Advanced Computing Center. DESC members also have connections to both astrophysics-oriented National Artificial Intelligence Research Institutes, funded jointly by NSF and the Simons Foundation (\acrshort{skai}, led by Northwestern University; and CosmicAI, led by the University of Texas at Austin), and to their associated computing resource allocations. Considering interests in Rubin--\textit{Roman}--\textit{Euclid} joint analysis, \acrfull{nasa}-funded computing resources may also be a viable option.

More broadly, many DESC members have access to significant campus-level computing at their institutions. Members outside the US have access to their own networks of national resources, including national and regional initiatives prioritizing computing for AI in science. Through the in-kind contribution program that supports LSST data rights for scientists outside of the US and Chile, a network of \acrshortpl{idac} is being deployed, with some sites bringing additional CPU and GPU capabilities that DESC would be well positioned to make use of. The UK will host an IDAC, sized to satisfy the resource needs of 20\% of the global LSST community during survey operations. The UK IDAC will be connected to UK national research computing facilities, both traditional (simulation and modeling) supercomputing services and the coming generation of AI-focused Digital Research infrastructure currently being specified and prototyped through the \acrshort{ascend} process\footnote{\acrlong{ascend}; see \url{https://engagementhub.ukri.org/ukri-infrastructure/ascend-process/}}.

Hyperscale commercial enterprises may also offer a viable path for certain novel and ambitious Rubin-LSST AI/ML applications, provided that their resources can be engaged through partnership or at significant discount. Potential partners include major cloud providers such as Google, Amazon, and Microsoft; major AI players such as OpenAI, Anthropic, and (again) Google; and GPU manufacturers such as NVIDIA and AMD. Additional effort would be required to develop private-sector partnerships that are mutually beneficial and compatible with the proprietary and non-commercial requirements of the Rubin--LSST data policy\footnote{\url{http://ls.st/RDO-013}}. On the positive side, timescales in industry are typically much shorter than in academia: work with an engaged hyperscale commercial partner could potentially deliver large-scale results quickly.

% environmental impact text moved to section 8 -- risks & challenges

\subsection{Data}

The primary data products relevant for \acrshort{desc} \acrshort{ai}/\acrshort{ml} work fall into several categories:
\begin{enumerate}
\item \acrshort{lsst} data products delivered by Rubin Observatory
\item Derived data products produced through DESC collaboration efforts
\item Data from other major surveys that enhance and expand DESC AI/ML science
\item Simulation data
\item Model weights and biases from AI models trained on the above
\end{enumerate}

Rubin-LSST data products are organized into three categories distinguished by the timescale of their delivery. The most immediate data are the world-public alert packets that will be distributed within minutes of shutter-close, including difference-image detections of transient and variable objects along with associated postage stamps and (for repeat detections) a 1-year time series. ``Prompt products'' will be released to the LSST data-rights community after 24 hours (for catalogs) and 80 hours (for full focal plane images). On a longer cadence, uniform reprocessing and coaddition across all epochs will deliver annual data releases of catalogs and images. The alert stream will be distributed via the network of LSST Community Brokers, while the prompt products and annual data releases will be accessible via the \acrshort{rsp} and also available for bulk download to DESC via the Rubin \acrshort{usdf} at \acrshort{slac}. A workflow based on the Rucio data management software has been implemented to mirror LSST data to \acrshort{nersc} from the USDF, and could be employed for staging data at \acrshort{alcf} and \acrshort{olcf} as well.

Other major data sets of interest for cross-match, co-analysis, and multi-modal FM training in conjunction with Rubin data include: space-based surveys such as \textit{Roman}, \textit{Euclid}, the \acrfull{spherex}, and the \acrfull{wise}; spectroscopic surveys such as the \acrfull{sdss}, \acrshort{desi}, and \acrshort{4most}; precursor imaging and time-domain surveys such as \acrshort{des}, the \acrfull{decals}, \acrfull{pan-starrs}, and \acrfull{ztf}; and \acrshort{cmb} data from facilities such as Planck, \acrshort{act}, the \acrfull{spt}, and \acrfull{so}.

Given the multi-petabyte size of the LSST data (and of simulations and other survey data sets on similar scales), both network transfer and disk storage will be limiting factors. DESC would likely benefit from strategic coordination with other LSST science collaborations, \acrshort{lincc} (see \autoref{sec7:broader_coordination}), and \acrshortpl{idac} regarding which LSST data products are mirrored where, for how long, at what quality of service, in conjunction with which \acrshort{cpu} and \acrshort{gpu} allocations, and for which analysis purposes.

The current Rubin Data Management system was not primarily designed for large-scale AI/ML work. Hence the data will need to be fitted with additional data interfaces, APIs, and standards that enable efficient use in this new context, such as the following examples:
\begin{itemize}
\item Adoption of tools like the Hyrax framework\footnote{\url{https://hyrax.readthedocs.io/en/stable/}} that provides modular components for a full AI/ML workflow tailored to astrophysical data.
\item Large-scale cross-match capabilities such as the \acrfull{hats}\footnote{\url{https://hats.readthedocs.io/en/latest/}} and Fink Xmatch\footnote{\url{https://fink-portal.org/xmatch}} that are critical to multi-modal dataset construction.
\item Performant and scalable services for streaming large batches of image cutouts into AI/ML training and inference workflows.
\item Data tokenization and embedding strategies that are well matched to AI/ML model architectures and downstream science tasks.
\item Active learning frameworks for alerts and images that maximize the value of limited human expert labeling time with respect to relevant modeling objectives.
\end{itemize}
In some cases, standards developed by the International Virtual Observatory Alliance\footnote{\url{https://www.ivoa.net}} may be fit for these purposes in DESC although they are typically conceived around classical astronomy use cases that do not map onto survey-scale AI/ML.

\subsection{Benchmarking and Reproducibility}

Challenges of reproducibility will only increase as \acrshort{ai}/\acrshort{ml}-based analyses become more common. Full computational reproducibility requires infrastructure for versioned retention of all input data, any pre-trained models used for inference, all analysis code and frameworks, and all software and environment dependencies. Solutions that allow for some or all of the above elements to be mutable over time fall short of true reproducibility but may be acceptable (or even preferable) to the extent that significant scientific conclusions remain replicable.

Additional challenges posed by the increasing adoption of AI/ML methods include:
\begin{itemize}
\item Defining the role and framework of blind analysis.
\item Maintaining independence of different experiments in the context of multi-modal FM-based analyses.
\item Defining training, validation, and test data sets when we ultimately want to use the full data set for cosmological inference.
\item Ensuring that any agentic software used or developed to generate \acrshort{desc} science records the provenance of the operations done within its task in ways that are compatible with scientific standards of reproducibility (explainable AI).
\end{itemize}

While the above are primarily questions of methodology rather than infrastructure, their solutions will have implications for infrastructure requirements. Some of the required infrastructure elements may include
\begin{itemize}
\item Persistent, accessible storage for testbed datasets, SBI training simulations, and pretrained model weights.
\item Hosted frameworks for deploying and running against science benchmarks.
\item Minified production environments that facilitate use of small-scale development instances to develop methods and benchmarks.
\item Standardized \acrshortpl{api} and architectures for reproducible large-scale model training and deployment.
\item Agentic frameworks for analysis reproduction \citep[e.g.,][]{ye2025replicationbench}
\item Comprehensive provenance tracking \& support for long-term co-archiving of data and analysis.
\end{itemize}

Our traditional thinking about reproducibility will be further challenged if \acrshortpl{llm} and AI agents continue to move us toward a more natural-language approach to dynamically and interactively extracting understanding from data. This trend can bring about a shift away from the traditionally sequential and siloed model of ``data, pipeline, results as papers'' and toward a more dynamic and interactive world characterized by prompts such as ``I want to regenerate the plot in Figure 1 of so-and-so's paper, except I want the magnitude cut at $g=22$ rather than $g=23$\dots'' Other disciplines will be experiencing similar transitions and DESC should look for opportunities both to lead by example and to benefit from broader trends and investment.

% Infrastructure text from 
The vast data scale of \acrshort{lsst} has necessitated a fundamental evolution in scientific methodology. This paradigm shift moves from local data processing on individual researchers' computers toward analytical tools designed to operate on shared, high-performance compute platforms. This centralization, combined with the anticipated widespread adoption of general-purpose foundation models, compels the establishment of a common implementation framework to ensure scientific analyses are reproducible, consistent, and interoperable.

While the \acrshort{rsp} serves as the primary portal for LSST data access, it was not designed for resource-intensive, large-scale AI applications. To bridge this gap, the Hyrax framework provides a  unified platform for exploring the latent space of foundation models. However, their operational deployment remains a significant undertaking that requires dedicated computational resources and a community-governed process for selection and validation.

The rapid pace of innovation in ML means that \acrshortpl{fm} cannot be considered static, long-term solutions. They must be constructed for a dynamic life cycle that includes systematic processes for review, evaluation, and replacement. This agile strategy is critical for all modalities, especially for time-series models deployed on real-time alert brokers, which must reliably process transient astronomical events to enable rapid discovery.

%
% To ensure a robust and sustainable AI ecosystem for Rubin science, future development must adhere to three core principles:
%  \begin{enumerate}
%      \item Deployment Compatibility: New foundation models must be explicitly designed for deployment within Hyrax and/or relevant real-time alert brokers.

%      \item Reproducibility: The model, its training data, and all relevant code must be versioned and archived to permit the complete replication of any scientific analysis (***point to where this is discussed in infrastructure***).
     
%      \item \note{What is the third core principle ?}
% \end{enumerate}



% This sub-section is commented out for now, can be added back later with more content
%
% \subsection{Human Support}
% \note{please add content here if you have thoughts on this aspect, we might remove this subsection if not enough content}
% \textit{This can include both the human efforts needed to maintain the other infrastructure (software, computing, data, reproducibility), and also the human oversight that is needed for vetting AI uses and policy development etc.}


%\subsection{Hyrax Text-Dump}
%\label{sec:hyrax}
%Wider community-level adoption of advanced ML techniques in astronomy is often hampered by fragmented, project-specific infrastructure where each group develops bespoke pipelines, duplicating effort and limiting reproducibility. For foundation models and unsupervised learning specifically, standardized frameworks are needed for systematic comparison across architectures and for exploring the high-dimensional latent spaces they produce.
%The Hyrax framework\footnote{\url{https://hyrax.readthedocs.io/en/stable/}} (Ghosh et al., in prep) attempts to address these gaps by providing modular, astronomy-tailored components for the full ML workflow. While supporting both supervised and unsupervised learning across diverse applications, it includes specialized tools particularly relevant for foundation models: interactive 2D/3D latent space visualization linked to source data, and vector-database similarity search scaling to billions of objects. Users can adopt community-produced foundation models, compare performance on standardized benchmarks, and perform downstream fine-tuning - or deploy entirely custom models using the same infrastructure. Integration with MLFlow, TensorBoard and Optuna enables systematic model comparison, hyperparameter tuning, and experiment tracking. 

%For Rubin, Hyrax provides native Butler integration enabling on-the-fly image cutout generation without local storage, and can be deployed on the RSP or any platform with Butler access (e.g., USDF, IDACs). Combined with support for scaling from laptop prototypes to multi-node HPC environments, Hyrax is positioned as a key enabling technology for wider community-wide adoption of AI within DESC. However, realizing this vision will require coordinated and sustained investment in specialized software engineering talent -- embedded within LSST-DM, LINCC, and DESC-DM -- to ensure tight integration between training/visualization/exploration infrastructure and the LSST stack. Equally critical is dedicated computing infrastructure co-located with LSST data systems and focused specifically on training, deploying, and using LSST-scale ML models. These investments are essential both for developing models on petabyte-scale datasets and for lowering adoption barriers by providing researchers seamless access to models and data through familiar LSST interfaces.


\newpage
\section{Opportunities for Broader AI/ML Coordination}
\label{sec7:broader_coordination}

\acrshort{desc} does not operate in isolation. The broader scientific impact of \acrshort{ai}/\acrshort{ml}-enabled analysis will depend critically on how well we coordinate with the rest of the Rubin ecosystem, other Stage-IV surveys, AI institutes, and large-scale compute providers. Many of these connections already exist in the form of shared personnel, joint projects, or informal collaborations. Here, we outline a non-exhaustive snapshot of this landscape and highlight opportunities to deepen and systematize these links. This section should be read as a living document: the specific institutes, infrastructures, and programs will evolve over the \acrshort{lsst} decade, but the underlying goal will remain positioning DESC as both a demanding scientific user and an influential driver of AI methodologies for fundamental science.

\paragraph{Coordination across the Rubin Community}
The Rubin LSST survey provides the most immediate opportunity for DESC coordination. \acrshort{lincc}, funded and supported by Schmidt Futures, prioritizes open-source and cross-project infrastructure for faint object detection, time-series data analysis, and photometric redshift estimation. Beyond producing intermediate Rubin data products that will be used in targeted ML pipelines, LINCC will serve a key role in maintaining a software ecosystem which has the potential to substantially accelerate the development of large-scale data foundation models and language models for science. DESC efforts in the initial years of the LSST survey can benefit LINCC Frameworks in optimizing its infrastructure toward these goals: for example, by stress-testing the throughput and accuracy of newly released pipelines and providing benchmark scientific datasets that LINCC can use to develop new detection and analysis algorithms.

In parallel with DESC, other LSST Science Collaborations are advancing AI methods for science, and coordinated work would accelerate progress. The \acrshort{issc} brings 150+ scientists from both academic and industry positions to shared discussions on large-scale data analysis with LSST. The ISSC could run independent audits of DESC \acrshort{photoz}, shear, and transient pipelines and partner with DESC to explore methodological advancements in AI/ML (\autoref{sec4:aiml_research}). The \acrshort{tvs} actively develops time-series analysis tools, and could partner with DESC in stress-testing broker infrastructure and classification benchmarks to evaluate cosmological readiness using photometric LSST samples. The \acrshort{galsc} collaboration could co-develop deblending and morphology benchmarks, in order to prevent biases in cosmological inference from mischaracterized shear and clustering constraints. The \acrfull{smwlv} analysis of crowded Milky Way fields will stress test deblending and photometric calibration, while the \acrfull{sssc} identification of solar-system objects and bogus alerts should inform the use in DESC of image products in large-scale scientific models. Across these collaborations, DESC should also encourage reproducible \acrshort{rsp} notebooks, tagged releases, and quarterly cross-collaboration readiness reviews in which independent teams reproduce results and report failure modes, potentially through the Rubin Community Forum\footnote{\url{https://community.lsst.org/}}.

\paragraph{Coordination of Stage IV Experiments}
Coordination between DESC (on behalf of the Rubin LSST) and other Stage-IV cosmological experiments will create additional opportunities for mitigating cross-survey systematics and optimizing the cosmological yield of LSST data.

\acrshort{desi} Data Release 2 is expected to contain $\sim$18.7M spectra across 4,000 deg$^2$ of overlap with the LSST footprint. This spectroscopic sample can be used as a primary calibrator for LSST redshift and lensing systematics. DESC should use DESI’s public releases to improve training sets for photometric-redshift models, require uncertainty coverage checks against those sets, and deploy clustering-redshift cross-correlations to validate $n(z)$ across tomographic bins. For weak lensing, DESC could cross-correlate LSST shear measurements with DESI density fields and test magnification and selection effects with controlled changes in DESI target completeness and fiber-assignment weights. 
The \acrshort{4most} \acrshort{tides} program \citep{tides}  will provide $\sim30,000$ spectroscopic transients. These data will enable real-time validation of classification algorithms and a sub-2\% measurement of the dark energy equation of state. The TiDES survey will also produce $>$200,000 spectroscopically-confirmed transient host galaxies, providing valuable contextual information for characterizing and marginalizing over environmental differences when curating standardizable SN~Ia samples. DESC could leverage these correlations to improve its survey simulations of the extragalactic time-domain sky in successors of the \acrshort{plasticc} and \acrshort{elasticc} challenges.
Coordination within the Roman Space Telescope presents possibly the most transformative opportunity for DESC. The OpenUniverse2024 simulations ($\sim$70 deg$^2$, $\sim$400 TB publicly available; \citealp{OpenUniverse2024}) enable immediate testing of how Roman's infrared imaging and superior resolution can be used for full multi-wavelength characterization of static sources in a joint data foundation model. Roman will also reveal blends invisible in Rubin data, allowing for validation of existing deblending/image segmentation methods for Rubin. Characterization with the high-redshift \acrshort{snia} population in Roman will also provide a laboratory for exploring any redshift-dependent systematics (e.g., changes in progenitor properties across cosmic time) that should be included in DESC cosmological analysis pipelines.
In addition, Schmidt Sciences announced in January 2026 the Eric and Wendy Schmidt Observatory System, a privately funded ``system-of-observatories'' designed for open-access time-domain and multi-messenger science complementary to LSST. The system comprises four facilities: the Argus Array \citep{2022Law_ArgusArray}, a $\sim$900-telescope optical array delivering $\sim$8,000~deg$^2$ instantaneous field of view with cadences down to $\sim$1~s; the Deep Synoptic Array \citep[DSA;][]{2019Hallinan_DSA2000}, a 1650$\times$6.15~m dish radio interferometer spanning 0.7--2~GHz with real-time imaging; the Large Fiber Array Spectroscopic Telescope \citep[LFAST;][]{2024Berkson_LFAST}, a scalable fiber-fed array of 0.76~m unit telescopes targeting ELT-class collecting area for photon-starved spectroscopy and rapid follow-up; and the Lazuli Space Observatory \citep{2026Roy_Lazuli}, a 3~m rapid-response optical--NIR facility (400--1700~nm) in lunar-resonant orbit with a wide-field imager and integral-field spectrograph capable of responding to targets of opportunity in $<$4 hours. With planned operations beginning as early as 2029 and a commitment to open data and shared analysis tools, this privately funded infrastructure could provide valuable cross-wavelength and high-cadence coverage for DESC time-domain and multi-messenger science, particularly for SN~Ia cosmology and transient follow-up. As private investment in astronomical infrastructure grows, DESC should monitor these developments for coordination opportunities.

\paragraph{Coordination with AI Institutes}
Two \acrshort{nsf}-Simons AI institutes have been launched as of September 2024, with funding and explicit scientific themes targeting LSST and cosmology. The \acrshort{skai} Institute between Northwestern, University of Illinois, and University of Chicago, is developing an \acrshort{fm} for transient science that can serve as a precursor to upcoming DESC models. CosmicAI at the University of Texas at Austin is developing \acrshort{llm}-powered AI ``copilots'' for research. These efforts create opportunities for DESC members to identify DESC pipelines most amenable to automation and provide CosmicAI with datasets for beta testing of their models. Across MIT, Harvard, Northeastern, and Tufts Universities, the \acrfull{iaifi} has also explored the use of generative models for field-level inference and multi-modal foundation models for transient science, which DESC can validate with synthetic Rubin datasets such as \acrshort{cosmodc2} 
\citep{2019Korytov_cosmodc2} and PLAsTiCC/ELAsTiCC \citep{PLAsTiCC1810.00001,2023AAS_ELAsTiCC}.  Further, the focus of DESC on AI integration at multiple stages of data processing will benefit the efforts of these institutes in incorporating realistic atmospheric effects and detector systematics directly into model architectures.

A feasible path toward training generalizable FMs for cosmology is to split the work. Models could be prototyped at individual universities or AI Institutes with support from LINCC Frameworks, and scaled through the pre-training of backbones on national and European supercomputers (pooled across \acrshort{doe} and \acrshort{eurohpc} facilities), since this will likely require hundreds of \acrshortpl{gpu} and training across multiple days. The models could then be fine‑tuned and calibrated near the data on DESC computing facilities such as\acrshort{nersc}, with brokers providing smaller fine-tuned heads as software filters for targeted streaming objectives.

A primary bottleneck to achieving this widespread scientific coordination is the development of robust, well-documented, and well-maintained software infrastructure. LINCC Frameworks, supported by the Schmidt Sciences, provides this support for the Rubin Observatory LSST, but this should be equally supported across all major observatories and international collaborations such as DESC in the coming years to enable the emerging technologies outlined in \autoref{sec5:emerging_tech}.

\paragraph{Coordination with European Networks}
\acrshort{eucaif} coordinates AI infrastructure and research across European institutions, and has produced white papers on infrastructure needs \citep[e.g.,][]{caron25}, and LLMs/FMs \citep{barman25}. DESC members at European institutions can, for instance, join EuCAIF WG4 (machine learning and artificial intelligence infrastructure) to contribute cosmology-specific challenges to EuCAIF's methods repository, or WG1 (foundation models \& discovery) to coordinate the development of FMs. These connections may facilitate successful applications for EuroHPC resources where DESC could test whether cutting-edge architectures will scale to LSST volumes. EuroHPC systems (Leonardo with 240 petaflops on NVIDIA A100 GPUs, \acrshort{lumi} with 380 petaflops on AMD MI250x GPUs, or the exascale \acrshort{jupiter} with NVIDIA GH200 superchips) enable training foundation models on billions of galaxy images, computationally infeasible on current NERSC allocations. These systems are already being deployed in astrophysics as a testbed for exascale and GPU-optimized implementations of simulation codes \citep[see e.g.][]{shukla25, lacopo26}. EuroHPC access follows a staged pathway from Benchmark (testing code scaling) to Development (algorithm validation) to Extreme Scale (production runs of up to 8M GPU hours). DESC could pursue Benchmark Access to validate early algorithms before committing to larger allocations. 

\paragraph{Collaborations with Industry}
Tech partnerships can provide expertise, computational resources, and opportunities to stress-test DESC methods at scale. NVIDIA's Academic Grant Program, along with complementary access through Google Cloud and Amazon Web Services, could allow DESC to rapidly prototype architectures and objective functions for foundation models at LSST scale.

Partnerships between DESC and \acrshort{llm} providers (e.g., Anthropic, OpenAI) should also be encouraged. Research credits would allow DESC to simultaneously explore the strengths and failure modes of the current generation of models. This compute could also be used to conduct systematic benchmarking of these models (through, e.g., HuggingFace) on targeted, science-specific use-cases. Any formal arrangement for \acrshort{llm} use across DESC would need to comply with LSST data rights policies (e.g., through private networking, complete audit trails, and explicit no-train/no-retain clauses). Such an arrangement would yield reproducible evaluation suites that could serve as a case study of language model readiness for science applications.

\paragraph{Rubin Alert Brokers}
The seven full-stream Rubin alert brokers are \acrshort{alerce} \citep{alerce}, \acrshort{ampel} \citep{ampel}, \acrshort{antares} \citep{antares}, Babamul \citep{babamul}, Fink \citep{fink}, Lasair \citep{lasair}, and Pitt-Google. These systems provide the primary filtering layer between Rubin streams and science-specific transient samples, turning raw alerts into ranked candidates, host associations, and early labels that will drive spectroscopic follow up and downstream analyses. Tight coordination with these brokers will give DESC direct leverage over the quantities that contribute to cosmological systematics: the completeness and purity of SN Ia samples, characterization of selection effects, and calibration of host-galaxy priors. Characterizing the selection functions of alert brokers as part of the analysis pipeline will help align transient discovery with the DESC requirements.

Brokers should maintain rigorous provenance tracking for all derived data features, host-galaxy associations, and classification/anomaly scores so that DESC can understand the selection effects of deployed algorithms. In return, collaboration with DESC can provide the alert brokers with benchmark datasets and the targeted science objectives used to validate their infrastructure and foster additional software development. Algorithms developed in the early years of the Rubin LSST can be ported upstream to broker environments after public release, providing additional metadata (e.g., embeddings from a data foundation model or concise, text-based descriptions of a subset of high-priority alerts) and allowing the broader scientific community and all Science Collaborations to benefit from DESC efforts without violating LSST data rights policies.


\newpage 
\section{Risks, Challenges, and Mitigation for AI/ML in DESC}
\label{sec:aiml_risks}

The increasing reliance on AI/ML within DESC brings not only opportunities but also a set of technical, organizational, and cultural risks that must be managed deliberately. The aim is not to discourage the use of AI, but to ensure that methods are deployed in ways that are scientifically robust, sustainable over the LSST decade, and compatible with DESC’s standards for transparency and reproducibility. Below we highlight key challenge areas together with concrete mitigation strategies.

\paragraph{Methodological Robustness and Interpretability}
AI/ML models are vulnerable to familiar statistical pitfalls: biased or incomplete training data, overfitting, and domain mismatch between simulations and real survey data. For methods that sit close to top-level cosmological inferences, there is a justified reluctance to adopt “black-box” results as reference constraints unless they can be thoroughly validated and stress-tested. This is amplified for neural summary statistics and learned emulators, where it can be difficult to diagnose failure modes or to construct transparent null tests. There is also a subtle “ML-only” risk: some future analyses may be so data- and compute-intensive that no independent non-ML cross-check is feasible, making it even more important that AI-based pipelines be internally well understood and stress-tested.\\
From a DESC perspective, mitigation rests on \emph{explicit validation and interpretability practices}: (i) defining standardized simulation and challenge suites where AI and traditional methods are compared head-to-head; (ii) publishing diagnostic plots and ablation studies that isolate which data features drive the constraints; (iii) requiring explicit documentation of model training domains, assumptions, and known failure modes, and discouraging use outside those regimes; (iv) developing approximate surrogate models or interpretable summaries (e.g.\ response functions, feature attributions) that can be inspected by domain experts; and (v) encouraging redundancy where it matters most—for example, using different architectures, loss functions, or simulation pipelines to cross-check key inferences, even when all are “AI-based”. Any AI-based result used as a reference cosmological constraint should be accompanied by a documented validation program and, where possible, benchmarked against simpler baselines.

\paragraph{Provenance, Reproducibility, and Integration into Pipelines}
As analyses become more complex, guaranteeing full provenance (from raw data through simulations, training runs, and model selection) becomes harder but more important. If DESC cosmology results depend on opaque training pipelines, unversioned models, or undocumented hyperparameters, small bugs or biases can consume a non-negligible fraction of the “tension budget” in precision tests of $\Lambda$CDM. There is also the practical challenge of integrating AI components into mature pipelines that already rely on well-tested, algorithmic codes.\\
Mitigation here is largely infrastructural and procedural: (i) treat trained models as first-class data products, with versioning, metadata, and model cards describing training data, objectives, and known limitations; (ii) require that AI components be runnable from containerized environments and integrated into CI pipelines with regression tests; (iii) maintain “shadow” implementations (simpler, slower, or more traditional pipelines) for cross-checks where feasible; and (iv) define clear deprecation and maintenance policies so that AI dependencies do not become unmaintainable over the survey lifetime.

\paragraph{Safe Usage, Data Rights, and External Services}
Widespread availability of commercial LLMs and AI services lowers the barrier to experimentation but introduces new questions about data governance and safe usage. Uploading proprietary Rubin/DESC data or unpublished results to third-party services may raise data-rights concerns, and using off-the-shelf models without understanding their limitations can encourage application of techniques outside their domain of validity.\\
DESC can mitigate these issues by (i) establishing clear guidelines on what kinds of data and metadata may be used with external services, in coordination with Rubin, LSST-DA, and agency policies; (ii) prioritizing self-hosted or collaboration-controlled deployments (e.g.\ for LLMs and inference services) for sensitive workloads; and (iii) prioritizing services committed to long term availability of their AI models and transparent versioning of these models, for long term reproducibility.

\paragraph{Human Capital, Training, and Sustainability}
AI/ML tools (and, increasingly, agentic assistants) can accelerate research by automating repetitive tasks and lowering the barrier to entry for complex workflows. However, there is a real risk that early-career researchers learn to operate pipelines as “black boxes” or as prompt engineers, without acquiring a deep understanding of the underlying statistics, physics, and numerics. At the same time, increased use of AI for infrastructure development can create technical debt and a maintenance burden for sophisticated codebases.\\
DESC can turn this into an opportunity by (i) framing AI/ML training as an integral part of graduate and postdoctoral education, combining hands-on use of tools with explicit coverage of underlying concepts; (ii) pairing students with mentors who can help them “open the box” at least once (e.g.\ by reproducing a result from scratch or implementing a simplified version of a method); and (iii) encouraging contributions to shared, well-documented libraries rather than bespoke one-off scripts, spreading maintenance across the collaboration and ensuring that successful methods become communal assets.

\paragraph{Computing Resources and Infrastructure}
Realizing the full potential of AI/ML within DESC will place non-trivial demands on computing resources. Training large foundation models for images, catalogs, or time series, and running large-scale simulation-based inference, requires sustained access to GPU clusters at a scale that goes beyond traditional analysis workloads. Similarly, if DESC wishes to host its own LLMs or other generative models for work involving sensitive or proprietary data, these services will need reliable, secure GPU backends and operational support over many years. Without careful planning, AI workloads risk competing destructively with other science uses for scarce accelerators, or fragmenting into ad hoc deployments that are hard to maintain.\\
Mitigation here is primarily strategic: (i) aligning major AI training campaigns with DESC’s existing resource-allocation processes and external partners (e.g.\ LSST-DA, national and international HPC centers); (ii) prioritizing shared, reusable models and services over one-off experiments; (iii) investing in efficient training and inference schemes (e.g.\ parameter-efficient finetuning, mixed precision, model distillation); and (iv) treating any self-hosted LLM or foundation-model service as collaboration infrastructure, with clear policies on access, data governance, and long-term support.


Overall, the main risks associated with AI/ML in DESC are not existential but \emph{operational}: biases that are hard to diagnose, results that are difficult to reproduce, methods that are fragile under domain shift, and human capital that is either over- or under-reliant on automation. By treating AI components with the same methodological rigor as any other part of the cosmology pipeline—through validation, documentation, governance, and training—DESC can reap their benefits while keeping these risks well under control.

\newpage
\section{Summary and Conclusion}
The Vera C.\ Rubin Observatory LSST will generate heterogeneous data at a scale and complexity that strain traditional analysis pipelines. DESC’s mission is to convert these data into robust constraints on the dark sector, which demands methods that are statistically powerful, scalable, and operationally reliable. AI/ML, from neural density estimators for photometric redshifts to simulation-based inference and generative models for field-level cosmology, have \textit{already} demonstrated that they can address key bottlenecks in this program. At the same time, their utility for precision cosmology hinges on trustworthy uncertainty quantification, explicit treatment of model misspecification and covariate shift, and fully reproducible integration into DESC workflows.

Sections 3 and 4 demonstrates that DESC is at the forefront of developing cutting-edge machine learning applications in astronomy. Research into machine learning is now integral to the primary LSST cosmological probes—including strong and weak lensing, supernovae, galaxy clusters, and large-scale structure—as well as cross-cutting topics such as theory, photometric redshifts, simulations, and deblending. Across DESC working groups and the broader cosmology community, several critical themes and methodologies have crystallized:

\begin{itemize}
 \item \textbf{Simulation-Based Inference (SBI):} SBI has emerged as a powerful methodology, enabling analyses of a complexity that typically exceeds the capabilities of traditional forward modeling. This domain offers fertile ground for machine learning research, particularly in the development of emulators to accelerate pipeline components and in extending analyses beyond traditional point statistics. However, SBI remains sensitive to model misspecification—a challenging problem to solve in a machine learning paradigm.
 \item \textbf{Bayesian Methodology and Uncertainty Quantification:} While Bayesian frameworks are ubiquitous in cosmology, machine learning is increasingly being explored to accelerate inference on LSST-scale datasets that would otherwise be computationally intractable. Furthermore, the high precision required by cosmology requires accurate uncertainty estimates which is not common practice in machine learning. Building on the application of Bayesian neural networks and related methods, this an area where DESC is well-positioned to drive fundamental developments in machine learning.
 \item \textbf{Validation and Benchmarking:} For cosmology, rigorous validation is paramount. Algorithms must not only be accurate and unbiased but also capable of correctly propagating uncertainty. Given that covariate shift is inevitable in many supervised learning contexts, it must be mitigated through accurate simulations and domain-adaptation techniques. Benchmarking and validation is particularly important for algorithms used in products intended for broad usage, such as foundation models and simulations. The RAIL project, developed by DESC specifically to benchmark photometric redshift algorithms, serves as an excellent model for such validation frameworks.
 \item \textbf{Active Learning and Discovery:} Active learning has become an essential part of machine learning and will be crucial in managing LSST data. The RESSPECT project, a collaborative initiative developing an active learning pipeline for transient classification, is an example of the comprehensive infrastructure required for effective active learning. Furthermore, human-in-the-loop workflows will be vital for anomaly detection and the identification of rare phenomena within the vast LSST dataset, facilitating novel discoveries that purely automated systems might overlook.
\end{itemize}
%This white paper surveys the intersection of AI/ML with DESC science analyses, highlighting concrete roles for ML in photometric redshifts, strong lensing, weak lensing and large-scale structure, clusters, supernova cosmology and transients, theory and modeling, and survey simulations (\autoref{sec3:use_case_for_aiml}). It then identifies methodological directions where DESC science directly motivates advances in the broader AI/ML ecosystem: Bayesian and simulation-based inference, physics-informed generative models, and novelty-detection frameworks that can operate at LSST scale (\autoref{sec4:aiml_research}). Finally, we emphasize the emerging importance of data foundation models and agentic AI systems (LLMs and multi-agent systems) as cross-cutting technologies, provided their deployment is coupled to rigorous evaluation, governance, and provenance tracking (See \autoref{sec5:emerging_tech}).

Realizing this potential requires DESC to treat AI/ML not as ad hoc accelerators, but as primary components of the measurement pipeline. Sections~6--8 of this paper outline the software, computing, and data infrastructure required to support AI/ML at scale, including a shared AI software stack, containerized and RSP-compatible workflows, a DESC Data Registry for model and data products, and benchmark suites that tie model performance directly to cosmological and systematic-error budgets (see \autoref{sec6:infra_requirements}). We also discuss opportunities for broader coordination with Rubin operations, community brokers, external AI/ML institutes, and industry, as well as risks ranging from model miscalibration and covariate shift to data rights compliance, environmental cost, and the erosion of human oversight (see \autoref{sec7:broader_coordination}).

On the basis of these insights, we identify a small number of high-level actions for the DESC community and its partners:

\begin{itemize}
  \item \textbf{DESC leadership and governance:} Establish a sustained AI/ML governance structure (e.g., an AI/ML Oversight and Policy Board) with responsibility for vetting high-impact AI uses, maintaining standards for uncertainty calibration, benchmarking, and reproducibility, and coordinating cross-working-group pathfinder analyses that demonstrate cosmology-grade AI/ML pipelines end-to-end.

  \item \textbf{Infrastructure providers and external partners:} Work with Rubin operations, LINCC, national laboratories, and computing centers to secure schedulable GPU and storage allocations sized for DESC’s foundation models and simulation-based inference campaigns, integrated with broker infrastructures and survey simulators. Coordinate with full-stream alert brokers and other Science Collaborations to define shared interfaces, provenance standards, and validation datasets that connect AI/ML performance directly to DESC’s science requirements.

  \item \textbf{DESC members and working groups:} Invest in training and documentation so that AI/ML methods, including foundation models and agentic systems, are deployed by teams that understand their assumptions, limitations, and failure modes. Treat human-in-the-loop review, explicit risk registries, and transparent reporting of AI-driven selection and inference as core elements of DESC scientific practice rather than afterthoughts.
\end{itemize}

Taken together, these steps will position DESC to use AI/ML in a way that is both ambitious and disciplined. LSST-era cosmology will not be limited by a lack of algorithms, but by our ability to connect those algorithms to physical modeling, survey simulations, and governance structures that make their behavior legible and trustworthy. By adopting a coherent AI/ML strategy grounded in DESC’s science priorities and supported by robust infrastructure, the collaboration can help shape best practices for AI in fundamental physics, while ensuring that the coming wave of AI tools advances precision dark-energy science without adding complexity to an already challenging problem.

\appendix

\section{Index of AI/ML Methodologies and Challenges}

\printnoidxglossary[type=methods,title={AI/ML Methods Index}]

\printnoidxglossary[type=challenges,title={Cross-cutting Challenges Index}]

\section{Acknowledgments}
The DESC acknowledges ongoing support from the Institut National de 
Physique Nucl\'eaire et de Physique des Particules in France; the 
Science \& Technology Facilities Council in the United Kingdom; and the
Department of Energy, the National Science Foundation, and the LSST 
Discovery Alliance in the United States. DESC uses resources of the IN2P3 
Computing Center (CC-IN2P3--Lyon/Villeurbanne - France) funded by the 
Centre National de la Recherche Scientifique; the National Energy 
Research Scientific Computing Center, a DOE Office of Science User 
Facility supported by the Office of Science of the U.S.\ Department of
Energy under Contract No.\ DE-AC02-05CH11231; STFC DiRAC HPC Facilities, 
funded by UK BEIS National E-infrastructure capital grants; and the UK 
particle physics grid, supported by the GridPP Collaboration.  This 
work was performed in part under DOE Contract DE-AC02-76SF00515. AM is supported by the Australian Research Council Discovery Early Career Research Award (DE230100055). ML acknowledges support from the South African Radio Astronomy Observatory and the National Research Foundation (NRF) towards this research. Opinions expressed and conclusions arrived at, are those of the authors and are not necessarily to be attributed to the NRF. HVP and ST have been supported by funding from the European Research Council (ERC) under the European Union's Horizon 2020 research and innovation programmes (grant agreement no.\ 101018897 CosmicExplorer). HVP was additionally supported by the G\"{o}ran Gustafsson Foundation for Research in Natural Sciences and Medicine.

\newpage
\bibliography{refs}

\end{document}