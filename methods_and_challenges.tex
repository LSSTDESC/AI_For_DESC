% methods_challenges.tex
%
% Assumes in main preamble:
%   \usepackage{xcolor}
%   \usepackage[acronym]{glossaries}
%   \makeglossaries
%
% Then include this file with:
%   \input{methods_challenges}
%
% In the body, use:
%   \meth{sbi}, \challenge{uq-calibration}, ...
% and print the indices with e.g.:
%   \printglossary[type=methods,title={AI/ML Methods Index}]
%   \printglossary[type=challenges,title={Cross-cutting Challenges Index}]

% ------------------------------------------------------------------
% Glossary types
% ------------------------------------------------------------------

\newglossary[mlg]{methods}{mls}{mln}{AI/ML Methods}
\newglossary[clg]{challenges}{cls}{cln}{Cross-cutting Challenges}

% Macros to invoke entries in the text
\newcommand{\meth}[1]{\gls{#1}}
\newcommand{\challenge}[1]{\gls{#1}}

% ------------------------------------------------------------------
% Color code for METHOD families (Cool/Professional - Anchored on Rhino)
% ------------------------------------------------------------------
% F1: Probabilistic & Bayesian Inference         -> aiMethBayes (Rhino - User Anchor)
% F2: Generative Modeling                        -> aiMethGen (Muted Purple)
% F3: Deep Learning Architectures                -> aiMethDeep (Muted Cyan)
% F4: Physics-Informed & Differentiable          -> aiMethPhys (Muted Green)
% F5: Representation & Discovery                 -> aiMethRepr (Muted Teal)
% F6: Data Processing & Vision                   -> aiMethVis (Steel Blue)

\definecolor{aiMethBayes}{HTML}{263654} % Rhino (User Anchor)
\definecolor{aiMethGen}{HTML}{5D4C76}   % Muted Purple
\definecolor{aiMethDeep}{HTML}{4A7A8C}  % Muted Cyan
\definecolor{aiMethPhys}{HTML}{5C7A63}  % Muted Green
\definecolor{aiMethRepr}{HTML}{3E8E7E}  % Muted Teal
\definecolor{aiMethVis}{HTML}{5B7C99}   % Steel Blue

\newcommand{\aimethbayes}[1]{\textcolor{aiMethBayes}{\textsf{#1}}}
\newcommand{\aimethgen}[1]{\textcolor{aiMethGen}{\textsf{#1}}}
\newcommand{\aimethdeep}[1]{\textcolor{aiMethDeep}{\textsf{#1}}}
\newcommand{\aimethphys}[1]{\textcolor{aiMethPhys}{\textsf{#1}}}
\newcommand{\aimethrepr}[1]{\textcolor{aiMethRepr}{\textsf{#1}}}
\newcommand{\aimethvis}[1]{\textcolor{aiMethVis}{\textsf{#1}}}

% ------------------------------------------------------------------
% Color code for CHALLENGE families (5 categories)
% ------------------------------------------------------------------
% C1: Covariate Shifts                           -> aiChCovariate
% C2: Uncertainty Quantification                 -> aiChUQ
% C3: Scalability                                -> aiChScale
% C4: Data Sparsity & Rare Events                -> aiChSparsity
% C5: Metrics & Evaluation                       -> aiChMetrics

\definecolor{aiChCovariate}{HTML}{C2814C}  % Tussock (warm orange)
\definecolor{aiChUQ}{HTML}{92402D}         % Mule Fawn (deep red-brown)
\definecolor{aiChScale}{HTML}{7D7068}      % Dark Pumice/Taupe
\definecolor{aiChSparsity}{HTML}{5D4C76}   % Muted Purple
\definecolor{aiChMetrics}{HTML}{C06C84}    % Muted Pink

\newcommand{\aichcovariate}[1]{\textcolor{aiChCovariate}{\textsf{#1}}}
\newcommand{\aichuq}[1]{\textcolor{aiChUQ}{\textsf{#1}}}
\newcommand{\aichscale}[1]{\textcolor{aiChScale}{\textsf{#1}}}
\newcommand{\aichsparsity}[1]{\textcolor{aiChSparsity}{\textsf{#1}}}
\newcommand{\aichmetrics}[1]{\textcolor{aiChMetrics}{\textsf{#1}}}

% ------------------------------------------------------------------
% METHOD ENTRIES (IDs used with \meth{...})
% ------------------------------------------------------------------

% F1: Probabilistic & Bayesian Inference

\newglossaryentry{sbi}{%
  type=methods,
  name={\aimethbayes{Simulation-based inference (SBI)}},
  text={\aimethbayes{SBI}},
  description={Implicit-likelihood Bayesian inference using forward simulations and neural density estimators}
}

\newglossaryentry{neural-density-estimation}{%
  type=methods,
  name={\aimethbayes{Neural density estimation}},
  text={\aimethbayes{Neural density estimation}},
  description={Neural networks trained to approximate probability densities, often used within SBI}
}

\newglossaryentry{npe}{%
  type=methods,
  name={\aimethbayes{Neural posterior estimation (NPE)}},
  text={\aimethbayes{NPE}},
  description={Direct neural approximation to the posterior distribution over parameters given simulated data}
}

\newglossaryentry{hierarchical-bayes}{%
  type=methods,
  name={\aimethbayes{Bayesian hierarchical modeling}},
  text={\aimethbayes{Hierarchical Bayes}},
  description={Hierarchical probabilistic models that share information across objects or populations}
}

\newglossaryentry{variational-inference}{%
  type=methods,
  name={\aimethbayes{Variational inference (VI)}},
  text={\aimethbayes{VI}},
  description={Optimization-based approach to approximating posterior distributions}
}

\newglossaryentry{ensembles}{%
  type=methods,
  name={\aimethbayes{Deep ensembles}},
  text={\aimethbayes{Ensembles}},
  description={Collections of models combined for improved performance and uncertainty quantification}
}

\newglossaryentry{gaussian-process}{%
  type=methods,
  name={\aimethbayes{Gaussian processes}},
  text={\aimethbayes{Gaussian processes}},
  description={Non-parametric Bayesian regression and emulation for scalar or functional outputs}
}

% F2: Generative Modeling

\newglossaryentry{diffusion-model}{%
  type=methods,
  name={\aimethgen{Diffusion / score-based models}},
  text={\aimethgen{Diffusion models}},
  description={Generative models that iteratively denoise data from a noise process, often used for images and fields}
}

\newglossaryentry{normalizing-flow}{%
  type=methods,
  name={\aimethgen{Normalizing flows}},
  text={\aimethgen{Normalizing flows}},
  description={Invertible neural networks used as flexible density models in likelihoods, posteriors, or simulators}
}

\newglossaryentry{vae}{%
  type=methods,
  name={\aimethgen{Variational autoencoders (VAEs)}},
  text={\aimethgen{VAEs}},
  description={Latent-variable generative models trained via variational inference}
}

\newglossaryentry{emulator}{%
  type=methods,
  name={\aimethgen{Emulators}},
  text={\aimethgen{Emulators}},
  description={Surrogate models (often GP- or NN-based) that approximate expensive simulations or likelihoods}
}

\newglossaryentry{neural-surrogate}{%
  type=methods,
  name={\aimethgen{Neural surrogates}},
  text={\aimethgen{Neural surrogates}},
  description={Neural network models trained to mimic the input--output behaviour of complex physical simulations}
}

% F3: Deep Learning Architectures

\newglossaryentry{cnn}{%
  type=methods,
  name={\aimethdeep{Convolutional neural networks (CNNs)}},
  text={\aimethdeep{CNNs}},
  description={Convolution-based neural networks for image-like data, widely used for classification and regression}
}

\newglossaryentry{transformer}{%
  type=methods,
  name={\aimethdeep{Transformers}},
  text={\aimethdeep{Transformers}},
  description={Attention-based neural architectures for sequences, time series, or generic sets of tokens (including Vision Transformers)}
}

\newglossaryentry{rnn}{%
  type=methods,
  name={\aimethdeep{Recurrent neural networks (RNNs)}},
  text={\aimethdeep{RNNs}},
  description={Sequence models (e.g.\ LSTMs, GRUs) for time series and light curves}
}

\newglossaryentry{gnn}{%
  type=methods,
  name={\aimethdeep{Graph neural networks (GNNs)}},
  text={\aimethdeep{GNNs}},
  description={Neural networks designed to operate on graph-structured data, such as cosmic webs or halo catalogs}
}

\newglossaryentry{deep-network}{%
  type=methods,
  name={\aimethdeep{Deep neural networks}},
  text={\aimethdeep{Deep networks}},
  description={Generic deep learning architectures not otherwise categorized}
}

% F4: Physics-Informed & Differentiable

\newglossaryentry{differentiable-programming}{%
  type=methods,
  name={\aimethphys{Differentiable programming}},
  text={\aimethphys{Differentiable programming}},
  description={Formulation of simulations and models as differentiable programs amenable to gradient-based inference}
}

\newglossaryentry{physics-informed}{%
  type=methods,
  name={\aimethphys{Physics-informed ML}},
  text={\aimethphys{Physics-informed ML}},
  description={Machine learning models that incorporate physical laws or constraints (e.g., PINNs, Hamiltonian NN)}
}

\newglossaryentry{symbolic-regression}{%
  type=methods,
  name={\aimethphys{Symbolic regression}},
  text={\aimethphys{Symbolic regression}},
  description={Learning analytic mathematical expressions that describe data or physical relationships}
}

% F5: Representation & Discovery

\newglossaryentry{neural-compression}{%
  type=methods,
  name={\aimethrepr{Neural compression}},
  text={\aimethrepr{Neural compression}},
  description={Learned low-dimensional representations (summary statistics or latents) of complex data}
}

\newglossaryentry{self-supervised}{%
  type=methods,
  name={\aimethrepr{Self-supervised learning}},
  text={\aimethrepr{Self-supervised learning}},
  description={Learning representations from unlabeled data by solving pretext tasks (e.g., masking, contrastive learning)}
}

\newglossaryentry{som}{%
  type=methods,
  name={\aimethrepr{Self-Organizing Maps (SOMs)}},
  text={\aimethrepr{SOMs}},
  description={Topology-preserving maps used for exploratory analysis and domain coverage characterization}
}

\newglossaryentry{anomaly-detection}{%
  type=methods,
  name={\aimethrepr{Anomaly detection}},
  text={\aimethrepr{Anomaly detection}},
  description={Identifying rare or out-of-distribution examples in data, crucial for new physics discovery}
}

\newglossaryentry{active-learning}{%
  type=methods,
  name={\aimethrepr{Active learning}},
  text={\aimethrepr{Active learning}},
  description={Strategies that adaptively select the most informative data points for labeling or follow-up}
}

% F6: Data Processing & Vision

\newglossaryentry{instance-segmentation}{%
  type=methods,
  name={\aimethvis{Instance segmentation}},
  text={\aimethvis{Instance segmentation}},
  description={Pixel-level segmentation of individual objects, relevant for deblending crowded scenes}
}

\newglossaryentry{object-detection}{%
  type=methods,
  name={\aimethvis{Object detection}},
  text={\aimethvis{Object detection}},
  description={Identifying and localizing objects in images (e.g., YOLO)}
}

\newglossaryentry{deblending}{%
  type=methods,
  name={\aimethvis{Deblending algorithms}},
  text={\aimethvis{Deblending}},
  description={Separating overlapping light profiles from multiple sources}
}

% ------------------------------------------------------------------
% CHALLENGE ENTRIES (IDs used with \challenge{...})
% ------------------------------------------------------------------

% C1: Covariate Shifts

\newglossaryentry{covariate-shift}{%
  type=challenges,
  name={\aichcovariate{Covariate Shifts}},
  text={\aichcovariate{Covariate shifts}},
  description={Distribution mismatches between training and target data (e.g.\ spectroscopic selection bias, sim-to-real gaps, model misspecification). Addressed in \autoref{sec4:model-misspec}, \autoref{sec4:hybrid-gen-phys}, and \autoref{sec:discovery}}
}

% C2: Uncertainty Quantification

\newglossaryentry{uq}{%
  type=challenges,
  name={\aichuq{Uncertainty Quantification}},
  text={\aichuq{UQ}},
  description={Obtaining well-calibrated posteriors and propagating uncertainties to cosmological constraints. Addressed in \autoref{sec4:Bayes}, \autoref{sec4:sbi}, and \autoref{sec4:validation}}
}

% C3: Scalability

\newglossaryentry{scalability}{%
  type=challenges,
  name={\aichscale{Scalability}},
  text={\aichscale{Scalability}},
  description={Handling LSST-scale data volumes, real-time alert processing, and high-dimensional inference. Addressed in \autoref{sec4:Bayes}, \autoref{sec4:sbi}, and \autoref{sec4:hybrid-gen-phys}}
}

% C4: Data Sparsity & Rare Events

\newglossaryentry{data-sparsity}{%
  type=challenges,
  name={\aichsparsity{Data Sparsity \& Rare Events}},
  text={\aichsparsity{Data sparsity}},
  description={Limited labeled samples, rare transients, class imbalance, and challenging edge cases like blending. Addressed in \autoref{sec:discovery}}
}

% C5: Metrics & Evaluation

\newglossaryentry{metrics}{%
  type=challenges,
  name={\aichmetrics{Metrics \& Evaluation}},
  text={\aichmetrics{Metrics}},
  description={Task-relevant metrics, validation frameworks, benchmarking, and stress tests for DESC science. Addressed in \autoref{sec4:validation} and \autoref{sec4:physics-constraints}}
}
