\newpage
\section{Opportunities for Broader AI/ML Coordination}
\label{sec7:broader_coordination}

% \note{The source content of this section still lives here: \url{https://docs.google.com/document/d/1M3rf3rL0yjhIDk5GeSiyR-uv4Izag--J3lCq84i8ne8/edit?tab=t.fope24a8phrw}.}
% \sectioninstr{Discuss how DESC could benefit from interactions in the broader environment on the coordination of the development and deployment of AI/ML methods. Subsections: Broader coordination within the LSST Rubin Community, Coordination with other Stage IV Science Collaborations, Opportunities for interactions with AI Institutes and Universities, Networks such as Eucaif and Ellis, Collaborations with tech industry.}

%Overview
%- Leverage $60M NSF-Simons AI institutes, EuroHPC AI Factories, and Roman-Rubin infrastructure
%- Position DESC as both consumer and driver of AI methodologies for systematic uncertainty control

%1. Partnership Landscape
%   Academic Partnerships
%   - NSF-Simons AI Institutes ($60M, Sept 2024)
%     • SkAI (Northwestern): Trustworthy AI, SELDON foundation model
%     • CosmicAI (UT Austin): Foundation models, LLM research assistants
%     • IAIFI (MIT): Physics-informed methods, simulation-based inference
   
%   - Rubin Community
%     • LINCC Frameworks: Infrastructure for foundation model deployment (RAIL)
%     • ISSC: Methodological consultation
%     • Broker community (Fink): Active learning for resource optimization   
%   - Stage IV Experiments
%     • DESI: Multi-survey training data (18.7M spectra)
 %    • 4MOST/TiDES: Real-time transient validation (30,000 spectroscopic targets)
 %    • Roman: Multi-resolution joint processing (OpenUniverse2024, DeepDISC)
   
%   Computing and Industry
%   - EuroHPC AI Factories: European GPU access for large-scale training
%   - NVIDIA/Cloud providers: Foundation model infrastructure
%   - Intel oneAPI: Cross-platform performance portability

%2. Thematic Science Applications
%   - Multi-survey photometric inference
%     • Foundation models across DESI+LSST+Roman+Euclid
%     • Simulation-based inference for calibrated uncertainties
   
%   - Weak lensing systematics
%     • Probabilistic scene decomposition for blended sources
%     • Physics-informed deblending with diffusion models
   
%   - Time-domain optimization
%     • Foundation models for transient classification
%     • Sequential decision-making for spectroscopic follow-up
   
%   - Computational emulation and forecasting
%     • Transfer learning from large cosmological simulations
%     • Neural emulators for covariance estimation

%3. Implementation Roadmap
%   - Immediate: Individual engagement, public data analysis, infrastructure requests
%   - Near-term: Postdoc co-supervision (Oct 2025), benchmark computing access
%   - Medium-term: Joint processing infrastructure, institute workshops, coordinated proposals
%   - Long-term: Sustained partnerships, performance portability frameworks, training pipelines


% DESC does not operate in isolation: the scientific impact of AI-enabled analysis will depend critically on how well we coordinate with the broader Rubin ecosystem, other Stage-IV surveys, AI institutes, and large-scale compute providers. Many of these connections already exist in the form of shared personnel, joint projects, or informal collaborations; here we outline a non-exhaustive snapshot of this landscape and highlight opportunities to deepen and systematize these links. This section should be read as a living document: the specific institutes, infrastructures, and programs will evolve over the LSST decade, but the underlying goal remains stable: position DESC as both a demanding scientific user and an influential driver of AI methodologies for fundamental science.

% \subsection{Partnership Landscape}

% \paragraph{Academic and AI-Institute Partnerships.}  
% A growing network of AI institutes and university-based centers provides natural hubs for DESC’s AI activities. NSF–Simons AI Institutes such as SkAI (Northwestern/Chicago/Illinois), CosmicAI (UT Austin–led), and IAIFI (MIT/Harvard/Northeastern/Tufts) bring together method developers and domain experts with mandates that align closely with DESC’s needs. Similar opportunities exist in Europe through networks such as ELLIS, EUCAIF, and national AI centers. At the umbrella level, the NSF AI Institutes Virtual Organization hosts an “AI for the Natural and Physical Sciences’’ special interest group, which explicitly focuses on AI methods for fields such as astronomy, and on using domain knowledge to improve AI itself. In parallel, NASA’s new AI/ML Science and Technology Interest Group (STIG) under the Cosmic Origins Program Analysis Group (COPAG) is building community-wide AI literacy through a structured program on topics ranging from LLMs as research agents to generative models and simulation-based inference. In addition, philanthropically funded initiatives such as LINCC Frameworks and Polymathic AI are developing software infrastructure and scientific foundation models, offering natural points of contact where DESC can both contribute demanding cosmology use cases and benefit from shared tooling, training resources, and reusable AI components that extend beyond any single survey. Many DESC members are already embedded in these communities; more deliberate coordination (e.g., co-funded postdocs, shared workshops and schools, and joint white papers and proposals) would amplify DESC’s influence on the design of AI methods and training programs directly relevant to Rubin science. In particular, DESC can act as a coordination structure to assemble proposals for postdoctoral and doctoral programs (e.g. European CoFund and Doctoral Networks schemes) and provide well-defined scientific use cases, datasets, and mentoring. DESC involvement (from coordinated letters of support to joint workshops, schools, and cross-appointed fellows) would help direct design of AI methods and training programs towards Rubin science, while ensuring that DESC members and early-career researchers are tightly plugged into the broader AI-for-science ecosystem.


% \paragraph{Rubin Community and Broker Ecosystem.}  
% Within the broader Rubin community, DESC’s AI activities are naturally coupled to efforts in other Rubin LSST Science Collaborations, in particular the Transients and Variable Stars (TVS) Collaboration, and the Informatics and Statistics Science Collaboration (ISSC), but also programs like LINCC Frameworks, and Rubin operations itself. LINCC Frameworks, in particular, is developing reusable infrastructure (e.g.\ RAIL for photo-$z$, Hyrax for neural networks) and could serve as a natural home for some DESC AI components. DESC members are already engaged in Rubin in-kind commissioning activities, for example exploring language-based interfaces and notebook generation, and in Rubin Project efforts on AI for active optics and scheduler optimization \citep[e.g.][]{JFC2024}. On the time-domain side, close coordination with the broker ecosystem (e.g.\ ANTARES, Fink) will be essential to deploy and test AI models in realistic streaming environments: early-time transient classification, host-galaxy characterization, active learning loops for spectroscopic follow-up, user-defined filters, and bulk-download/cross-match services are all areas where DESC-developed models could both benefit from and help shape broker capabilities. These connections provide an on-ramp from DESC research prototypes to community-facing services.

% \paragraph{Stage-IV Surveys and Joint Pixel Processing.}  
% DESC science increasingly relies on joint analysis across multiple Stage-IV experiments. DESI and other spectroscopic programs (e.g.\ 4MOST/TiDES) provide training samples and validation sets for photometric-redshift, classification, and source-inference models, including for transients where spectra+light curves+images must be combined in real time. Euclid and Roman create additional opportunities for joint pixel-level processing: combining Rubin with space-based imaging for deblending, shape measurement, and source inference; building cross-survey foundation models for photometry and morphology; and developing shared data products and processing infrastructure (e.g.\ extending Euclid–Rubin DDP concepts to Roman). Co-location of Rubin+Roman+Euclid data and compute—whether at US centers, EuroHPC sites, or IDACs with GPU capability—would be particularly valuable for training and deploying such models at scale. Strengthening and formalizing these cross-survey links positions DESC as a key interlocutor for AI methods that must operate consistently across experiments.

% \paragraph{Compute and Industry Partnerships.}  
% Meeting DESC’s AI ambitions will require access to substantial GPU resources beyond what is available on local clusters. EuroHPC systems, PRACE centers, national supercomputing facilities, and Rubin IDACs equipped with GPUs are natural candidates for large-scale training and inference campaigns, especially for foundation models and simulation-based inference. On the US side, initiatives such as the American Science Cloud are exploring shared cloud-like resources for academic users. Industry collaborations—whether through academic grant programs (e.g.\ NVIDIA, cloud providers) or public–private partnerships—can provide hardware, credits, and software tooling, but must be navigated carefully given data-rights and governance constraints. DESC’s existing resource-allocation structures (e.g.\ the bi-annual Resource Committee and international in-kind leads) offer a mechanism to integrate these external resources: DESC can provide well-defined workloads, support, and data; partners can provide compute and tooling; and the collaboration retains ownership of pipelines and scientific products.

% \subsection{Thematic Science Opportunities}

% \paragraph{Multi-Survey Foundation Models.}  
% Coordinated use of AI across Rubin, DESI, Euclid, Roman, and other surveys will enable foundation models that ingest heterogeneous inputs (images, spectra, catalogs, time series) and support tasks such as photometric-redshift estimation, source classification, and population-level inference with calibrated uncertainties. DESC can play a central role in defining benchmark tasks, datasets, and evaluation protocols that leverage DESI spectra, Euclid/Roman imaging, and Rubin time-domain data simultaneously, turning cross-survey analysis from an ad hoc activity into a coherent AI research program.

% \textit{Time-Domain Optimization and Sequential Decision-Making.}  
% In the time domain, coordination with brokers, spectroscopic facilities, and survey schedulers opens the door to AI-powered policies for transient classification and follow-up. Foundation models trained on historical light curves, images, and host-galaxy information can provide early classifications and uncertainty estimates; agentic frameworks can then use these posteriors to drive sequential decision-making for spectroscopic targeting and follow-up cadence. DESC can help design and test such strategies in simulation, then port them to broker environments and scheduler prototypes.

% \textit{Computational Emulation and Forecasting.}  
% At the interface of cosmological simulations and observations, DESC is already developing neural emulators and differentiable forward models for weak lensing, large-scale structure, and galaxy populations. Coordination with simulation efforts across Stage-IV surveys and AI institutes—e.g.\ shared suites of simulations, common parameterizations, and foundation models trained on multi-experiment data—would enable emulators that are reusable across collaborations and directly tailored to joint analyses. Here again, DESC is well positioned to articulate scientifically motivated requirements (accuracy, coverage, uncertainty quantification) and to help steer method development in directions that maximize impact on systematic-uncertainty control.

% Taken together, these partnerships define an ecosystem in which DESC is both a demanding scientific user of AI infrastructure and a generator of methods, benchmarks, and best practices that benefit the broader community. As the AI and survey landscapes evolve, this section should be revisited regularly to document new connections and ensure that DESC remains strategically positioned within the wider network of AI-for-astronomy efforts.

DESC does not operate in isolation: the broader scientific impact of AI-enabled analysis will depend critically on how well we coordinate with the rest of the Rubin ecosystem, other Stage-IV surveys, AI institutes, and large-scale compute providers. Many of these connections already exist in the form of shared personnel, joint projects, or informal collaborations. Here, we outline a non-exhaustive snapshot of this landscape and highlight opportunities to deepen and systematize these links. This section should be read as a living document: the specific institutes, infrastructures, and programs will evolve over the LSST decade, but the underlying goal will remain positioning DESC as both a demanding scientific user and an influential driver of AI methodologies for fundamental science.

% Contemporaneous efforts in AI for Science can be leveraged by the DESC to better meet its science goals and emerge as a leader in the design and use of scientific AI. Within the Rubin LSST ecosystem, coordination with other Science Collaborations, LINCC Frameworks, and the Project itself will ensure proper use and documentation of AI techniques beyond cosmological analyses and at every level of data processing. Coordination with other Stage IV Scientific experiments will be necessary for developing the shared datasets and computational infrastructure (Section 6) on which diverse emerging technologies will rely (Section 5). The rise of Simons-NSF AI Institutes within the US and large-scale resources across Europe, as well as major industry leads in the hardware and software driving scientific AI, provide additional avenues for coordination.

\paragraph{Coordination across the Rubin Community}
The Rubin LSST survey provides the most immediate opportunity for DESC coordination. LINCC Frameworks, funded and supported by Schmidt Futures, prioritizes open-source and  cross-project infrastructure for faint object detection, time-series data analysis, and photometric redshift estimation. Beyond producing intermediate Rubin data products that will be used in targeted ML pipelines, LINCC will serve a key role in maintaining a software ecosystem which has the potential to substantially accelerate the development of large-scale data foundation models and language models for science. DESC efforts in the initial years of the LSST survey can benefit LINCC Frameworks in optimizing its infrastructure toward these goals: for example, by stress-testing the throughput and accuracy of newly released pipelines and providing benchmark scientific datasets that LINCC can use to develop new detection and analysis algorithms.

In parallel with DESC, other LSST Science Collaborations are advancing AI methods for science, and coordinated work would accelerate progress. The Informatics and Statistics Science Collaboration (ISSC) brings 150+ scientists from both academic and industry positions to shared discussions on large-scale data analysis with LSST. The ISSC could run independent audits of DESC photo-z, shear, and transient pipelines and partner with DESC to explore methodological advancements in AI/ML (\autoref{sec4:aiml_research}). The Transients and Variable Stars (TVS) collaboration actively develops time-series analysis tools, and could partner with DESC in stress-testing broker infrastructure and classification benchmarks to evaluate cosmological readiness using photometric LSST samples. The Galaxies collaboration could co-develop deblending and morphology benchmarks, in order to prevent biases in cosmological inference from mischaracterized shear and clustering. The Stars, Milky Way, and Local Volume collaboration’s analysis of crowded Milky Way fields will stress test deblending and photometric calibration, while the Solar System collaboration’s identification of solar-system objects and bogus alerts should inform the use in DESC of image products in large-scale scientific models. Across these collaborations, DESC should also encourage reproducible RSP notebooks, tagged releases, and quarterly cross-collaboration readiness reviews in which independent teams reproduce results and report failure modes, potentially through the Rubin Community Forum\footnote{\url{https://community.lsst.org/}}.

\paragraph{Coordination of Stage IV Experiments}
Coordination between DESC (on behalf of the Rubin LSST) and other Stage-IV cosmological experiments will create additional opportunities for mitigating cross-survey systematics and optimizing the cosmological yield of LSST data.
The Dark Energy Spectroscopic Instrument (DESI) Data Release 2 is expected to contain $\sim$18.7M spectra across 4,000 deg² of overlap with the LSST footprint. This spectroscopic sample can be used as a primary calibrator for LSST redshift and lensing systematics. DESC should use DESI’s public releases to improve training sets for photometric-redshift models, require uncertainty coverage checks against those sets, and deploy clustering-redshift cross-correlations to validate $n(z)$ across tomographic bins. For weak lensing, DESC could cross-correlate LSST shear measurements with DESI density fields and test magnification and selection effects with controlled changes in DESI target completeness and fiber-assignment weights. 
The Time-Domain Extragalactic Survey (4MOST/TiDES; \citealp{tides}), to be conducted on the 4-meter Multi-Object Spectroscopic Telescope, will provide $\sim$30,000 spectroscopic transients. These data will enable real-time validation of classification algorithms and a sub-2\% measurement of the dark energy equation of state. The TiDES survey will also produce $>$200,000 spectroscopically-confirmed transient host galaxies. providing valuable contextual information for characterizing and marginalizing over environmental differences when curating standardizable SN Ia samples. DESC could leverage these correlations to improve its survey simulations of the extragalactic time-domain sky in successors of the PLAsTiCC and ELAsTiCC challenges.
Coordination within the Roman Space Telescope presents possibly the most transformative opportunity for DESC. The OpenUniverse2024 simulations ($\sim$70 deg$^2$, $\sim$400 TB publicly available; \citealp{OpenUniverse2024}) enable immediate testing of how Roman’s infrared imaging and superior resolution can be used for full multi-wavelength characterization of static sources in a joint data foundation model. Roman will also reveal blends invisible in Rubin data, allowing for validation of existing deblending/image segmentation methods for Rubin. Characterization with the high-redshift SN Ia population in Roman will also provide a laboratory for exploring any redshift-dependent systematics (e.g., changes in progenitor properties across cosmic time) that should be included in DESC cosmological analysis pipelines. Finally, the enhanced resolution of Roman may improve the constraints on galaxy morphologies used in DESC shear analyses.

\paragraph{Coordination with AI Institutes}
Three NSF-Simons AI institutes have been launched as of September 2024, with \$60M funding and explicit scientific themes targeting LSST and cosmology. The SkAI Institute between Northwestern, University of Illinois, and University of Chicago, is developing a foundation model for transient science that can serve as a precursor to upcoming DESC models. CosmicAI at the University of Texas at Austin is developing LLM-powered AI ``copilots'' for research. This creates an opportunity for DESC members to identify DESC pipelines most amenable to automation and provide CosmicAI with datasets for beta testing of their models. Across MIT, Harvard, Northeastern, and Tufts Universities, the NSF Institute for AI and Fundamental Interactions (IAIFI) has explored the use of generative models for field-level inference and multi-modal foundation models for transient science, which DESC can validate with synthetic Rubin datasets such as CosmoDC2 
\citep{2019Korytov_cosmodc2} and PLAsTiCC/ELAsTiCC transients \citep{PLAsTiCC1810.00001,2023AAS_ELAsTiCC}.  Further, the focus of DESC on AI integration at multiple stages of data processing will benefit the efforts of these institutes in incorporating realistic atmospheric effects and detector systematics directly into model architectures.

\paragraph{Coordination with European Networks}
The European Coalition for AI in Fundamental Physics (EuCAIF) coordinates AI infrastructure and research across European institutions, and has produced white papers on infrastructure needs \citep[e.g.][]{caron25}, and LLMs/foundation models \citep{barman25}. DESC members at European institutions can for instance join EuCAIF Working Group 4 (WG4; Machine Learning and Artificial Intelligence Infrastructure) to contribute cosmology-specific challenges to EuCAIF's methods repositorym or Working Group 1 (WG1; Foundation models \& discovery) to coordinate the development of Foundation Models. These connections may facilitate successful applications for EuroHPC resources where DESC could test whether cutting-edge architectures will scale to LSST volumes. EuroHPC systems (Leonardo with 240 petaflops on NVIDIA A100 GPUs, LUMI with 380 petaflops on AMD MI250x GPUs, or the exascale JUPITER with NVIDIA GH200 superchips) enable training foundation models on billions of galaxy images, computationally infeasible on current NERSC allocations. EuroHPC access follows a staged pathway from Benchmark (testing code scaling) to Development (algorithm validation) to Extreme Scale (production runs of up to 8M GPU hours). DESC could pursue Benchmark Access to validate early algorithms before committing to larger allocations. 


\paragraph{Collaborations with Industry}
Tech partnerships can provide expertise, computational resources, and opportunities to stress-test DESC methods at scale. NVIDIA's Academic Grant Program, along with complementary access through Google Cloud and Amazon AWS, could allow DESC to rapidly prototype architectures and objective functions for foundation models at LSST scale.

Partnerships between DESC and LLM providers (e.g., Anthropic, OpenAI) should also be encouraged. Research credits would allow multiple Working Groups to simultaneously explore the strengths and failure modes of the current generation of models. This compute could also be used to conduct systematic benchmarking of these models (through, e.g., HuggingFace) on targeted, science-specific use-cases. One potential deployment is through a containerized instance of leading models hosted on Azure through a formal Microsoft agreement, with private networking, complete audit trails, and explicit no-train/no-retain clauses. This arrangement would allow LLM use across DESC without violating LSST data rights policies, and would yield reproducible evaluation suites that can be advertised as a case study of the science readiness of language models.

\paragraph{Rubin Alert Brokers}
The seven full-stream Rubin alert brokers are ALeRCE \citep{alerce}, AMPEL \citep{ampel}, ANTARES \citep{antares}, Babamul \citep{babamul}, Fink \citep{fink}, Lasair \citep{lasair}, and Pitt-Google. These systems provide the primary filtering layer between Rubin streams and science-specific transient samples, turning raw alerts into ranked candidates, host associations, and early labels that will drive spectroscopic follow up and downstream analyses. Tight coordination with these brokers will give DESC direct leverage over the quantities that contribute to cosmological systematics: the completeness and purity of SN Ia samples, characterization of selection effects, and calibration of host-galaxy priors. Characterizing the selection functions of alert brokers as part of the analysis pipeline will help align transient discovery with the DESC requirements.

Brokers should maintain rigorous, queryable provenance for features, associations, and classification/anomaly scores so that DESC can reproduce every selection cut. In return, collaboration with DESC will provide the alert brokers with clear, science-driven validation frameworks for their infrastructure. Algorithms developed in the early years of the Rubin LSST can be ported upstream to broker environments after public release, providing additional metadata (e.g., embeddings from a data foundation model or concise, text-based descriptions of a subset of high-priority alerts) and allowing the broader scientific community and all Science Collaborations to benefit from DESC efforts without violating LSST data rights policies.






