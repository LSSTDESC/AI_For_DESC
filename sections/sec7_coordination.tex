\newpage
\section{Opportunities for Broader AI/ML Coordination}
\label{sec7:broader_coordination}

\acrshort{desc} does not operate in isolation. The broader scientific impact of \acrshort{ai}/\acrshort{ml}-enabled analysis will depend critically on how well we coordinate with the rest of the Rubin ecosystem, other Stage-IV surveys, AI institutes, and large-scale compute providers. Many of these connections already exist in the form of shared personnel, joint projects, or informal collaborations. Here, we outline a non-exhaustive snapshot of this landscape and highlight opportunities to deepen and systematize these links. This section should be read as a living document: the specific institutes, infrastructures, and programs will evolve over the \acrshort{lsst} decade, but the underlying goal will remain positioning DESC as both a demanding scientific user and an influential driver of AI methodologies for fundamental science.

\paragraph{Coordination across the Rubin Community}
The Rubin LSST survey provides the most immediate opportunity for DESC coordination. \acrshort{lincc}, funded and supported by Schmidt Futures, prioritizes open-source and cross-project infrastructure for faint object detection, time-series data analysis, and photometric redshift estimation. Beyond producing intermediate Rubin data products that will be used in targeted ML pipelines, LINCC will serve a key role in maintaining a software ecosystem which has the potential to substantially accelerate the development of large-scale data foundation models and language models for science. DESC efforts in the initial years of the LSST survey can benefit LINCC Frameworks in optimizing its infrastructure toward these goals: for example, by stress-testing the throughput and accuracy of newly released pipelines and providing benchmark scientific datasets that LINCC can use to develop new detection and analysis algorithms.

In parallel with DESC, other LSST Science Collaborations are advancing AI methods for science, and coordinated work would accelerate progress. The \acrshort{issc} brings 150+ scientists from both academic and industry positions to shared discussions on large-scale data analysis with LSST. The ISSC could run independent audits of DESC \acrshort{photoz}, shear, and transient pipelines and partner with DESC to explore methodological advancements in AI/ML (\autoref{sec4:aiml_research}). The \acrshort{tvs} actively develops time-series analysis tools, and could partner with DESC in stress-testing broker infrastructure and classification benchmarks to evaluate cosmological readiness using photometric LSST samples. The \acrshort{galsc} collaboration could co-develop deblending and morphology benchmarks, in order to prevent biases in cosmological inference from mischaracterized shear and clustering constraints. The \acrfull{smwlv} analysis of crowded Milky Way fields will stress test deblending and photometric calibration, while the \acrfull{sssc} identification of solar-system objects and bogus alerts should inform the use in DESC of image products in large-scale scientific models. Across these collaborations, DESC should also encourage reproducible \acrshort{rsp} notebooks, tagged releases, and quarterly cross-collaboration readiness reviews in which independent teams reproduce results and report failure modes, potentially through the Rubin Community Forum\footnote{\url{https://community.lsst.org/}}.

\paragraph{Coordination of Stage IV Experiments}
Coordination between DESC (on behalf of the Rubin LSST) and other Stage-IV cosmological experiments will create additional opportunities for mitigating cross-survey systematics and optimizing the cosmological yield of LSST data.

\acrshort{desi} Data Release 2 is expected to contain $\sim$18.7M spectra across 4,000 deg$^2$ of overlap with the LSST footprint. This spectroscopic sample can be used as a primary calibrator for LSST redshift and lensing systematics. DESC should use DESI’s public releases to improve training sets for photometric-redshift models, require uncertainty coverage checks against those sets, and deploy clustering-redshift cross-correlations to validate $n(z)$ across tomographic bins. For weak lensing, DESC could cross-correlate LSST shear measurements with DESI density fields and test magnification and selection effects with controlled changes in DESI target completeness and fiber-assignment weights. 
The \acrshort{4most} \acrshort{tides} program \citep{tides}  will provide $\sim30,000$ spectroscopic transients. These data will enable real-time validation of classification algorithms and a sub-2\% measurement of the dark energy equation of state. The TiDES survey will also produce $>$200,000 spectroscopically-confirmed transient host galaxies, providing valuable contextual information for characterizing and marginalizing over environmental differences when curating standardizable SN~Ia samples. DESC could leverage these correlations to improve its survey simulations of the extragalactic time-domain sky in successors of the \acrshort{plasticc} and \acrshort{elasticc} challenges.
Coordination within the Roman Space Telescope presents possibly the most transformative opportunity for DESC. The OpenUniverse2024 simulations ($\sim$70 deg$^2$, $\sim$400 TB publicly available; \citealp{OpenUniverse2024}) enable immediate testing of how Roman's infrared imaging and superior resolution can be used for full multi-wavelength characterization of static sources in a joint data foundation model. Roman will also reveal blends invisible in Rubin data, allowing for validation of existing deblending/image segmentation methods for Rubin. Characterization with the high-redshift \acrshort{snia} population in Roman will also provide a laboratory for exploring any redshift-dependent systematics (e.g., changes in progenitor properties across cosmic time) that should be included in DESC cosmological analysis pipelines.
In addition, Schmidt Sciences announced in January 2026 the Eric and Wendy Schmidt Observatory System, a privately funded ``system-of-observatories'' designed for open-access time-domain and multi-messenger science complementary to LSST. The system comprises four facilities: the Argus Array \citep{2022Law_ArgusArray}, a $\sim$900-telescope optical array delivering $\sim$8,000~deg$^2$ instantaneous field of view with cadences down to $\sim$1~s; the Deep Synoptic Array \citep[DSA;][]{2019Hallinan_DSA2000}, a 1650$\times$6.15~m dish radio interferometer spanning 0.7--2~GHz with real-time imaging; the Large Fiber Array Spectroscopic Telescope \citep[LFAST;][]{2024Berkson_LFAST}, a scalable fiber-fed array of 0.76~m unit telescopes targeting ELT-class collecting area for photon-starved spectroscopy and rapid follow-up; and the Lazuli Space Observatory \citep{2026Roy_Lazuli}, a 3~m rapid-response optical--NIR facility (400--1700~nm) in lunar-resonant orbit with a wide-field imager and integral-field spectrograph capable of responding to targets of opportunity in $<$4 hours. With planned operations beginning as early as 2029 and a commitment to open data and shared analysis tools, this privately funded infrastructure could provide valuable cross-wavelength and high-cadence coverage for DESC time-domain and multi-messenger science, particularly for SN~Ia cosmology and transient follow-up. As private investment in astronomical infrastructure grows, DESC should monitor these developments for coordination opportunities.

\paragraph{Coordination with AI Institutes}
Two \acrshort{nsf}-Simons AI institutes have been launched as of September 2024, with funding and explicit scientific themes targeting LSST and cosmology. The \acrshort{skai} Institute between Northwestern, University of Illinois, and University of Chicago, is developing an \acrshort{fm} for transient science that can serve as a precursor to upcoming DESC models. CosmicAI at the University of Texas at Austin is developing \acrshort{llm}-powered AI ``copilots'' for research. These efforts create opportunities for DESC members to identify DESC pipelines most amenable to automation and provide CosmicAI with datasets for beta testing of their models. Across MIT, Harvard, Northeastern, and Tufts Universities, the \acrfull{iaifi} has also explored the use of generative models for field-level inference and multi-modal foundation models for transient science, which DESC can validate with synthetic Rubin datasets such as \acrshort{cosmodc2} 
\citep{2019Korytov_cosmodc2} and PLAsTiCC/ELAsTiCC \citep{PLAsTiCC1810.00001,2023AAS_ELAsTiCC}.  Further, the focus of DESC on AI integration at multiple stages of data processing will benefit the efforts of these institutes in incorporating realistic atmospheric effects and detector systematics directly into model architectures.

A feasible path toward training generalizable FMs for cosmology is to split the work. Models could be prototyped at individual universities or AI Institutes with support from LINCC Frameworks, and scaled through the pre-training of backbones on national and European supercomputers (pooled across \acrshort{doe} and \acrshort{eurohpc} facilities), since this will likely require hundreds of \acrshortpl{gpu} and training across multiple days. The models could then be fine‑tuned and calibrated near the data on DESC computing facilities such as\acrshort{nersc}, with brokers providing smaller fine-tuned heads as software filters for targeted streaming objectives.

A primary bottleneck to achieving this widespread scientific coordination is the development of robust, well-documented, and well-maintained software infrastructure. LINCC Frameworks, supported by the Schmidt Sciences, provides this support for the Rubin Observatory LSST, but this should be equally supported across all major observatories and international collaborations such as DESC in the coming years to enable the emerging technologies outlined in \autoref{sec5:emerging_tech}.

\paragraph{Coordination with European Networks}
\acrshort{eucaif} coordinates AI infrastructure and research across European institutions, and has produced white papers on infrastructure needs \citep[e.g.,][]{caron25}, and LLMs/FMs \citep{barman25}. DESC members at European institutions can, for instance, join EuCAIF WG4 (machine learning and artificial intelligence infrastructure) to contribute cosmology-specific challenges to EuCAIF's methods repository, or WG1 (foundation models \& discovery) to coordinate the development of FMs. These connections may facilitate successful applications for EuroHPC resources where DESC could test whether cutting-edge architectures will scale to LSST volumes. EuroHPC systems (Leonardo with 240 petaflops on NVIDIA A100 GPUs, \acrshort{lumi} with 380 petaflops on AMD MI250x GPUs, or the exascale \acrshort{jupiter} with NVIDIA GH200 superchips) enable training foundation models on billions of galaxy images, computationally infeasible on current NERSC allocations. These systems are already being deployed in astrophysics as a testbed for exascale and GPU-optimized implementations of simulation codes \citep[see e.g.][]{shukla25, lacopo26}. EuroHPC access follows a staged pathway from Benchmark (testing code scaling) to Development (algorithm validation) to Extreme Scale (production runs of up to 8M GPU hours). DESC could pursue Benchmark Access to validate early algorithms before committing to larger allocations. 

\paragraph{Collaborations with Industry}
Tech partnerships can provide expertise, computational resources, and opportunities to stress-test DESC methods at scale. NVIDIA's Academic Grant Program, along with complementary access through Google Cloud and Amazon Web Services, could allow DESC to rapidly prototype architectures and objective functions for foundation models at LSST scale.

Partnerships between DESC and \acrshort{llm} providers (e.g., Anthropic, OpenAI) should also be encouraged. Research credits would allow DESC to simultaneously explore the strengths and failure modes of the current generation of models. This compute could also be used to conduct systematic benchmarking of these models (through, e.g., HuggingFace) on targeted, science-specific use-cases. Any formal arrangement for \acrshort{llm} use across DESC would need to comply with LSST data rights policies (e.g., through private networking, complete audit trails, and explicit no-train/no-retain clauses). Such an arrangement would yield reproducible evaluation suites that could serve as a case study of language model readiness for science applications.

\paragraph{Rubin Alert Brokers}
The seven full-stream Rubin alert brokers are \acrshort{alerce} \citep{alerce}, \acrshort{ampel} \citep{ampel}, \acrshort{antares} \citep{antares}, Babamul \citep{babamul}, Fink \citep{fink}, Lasair \citep{lasair}, and Pitt-Google. These systems provide the primary filtering layer between Rubin streams and science-specific transient samples, turning raw alerts into ranked candidates, host associations, and early labels that will drive spectroscopic follow up and downstream analyses. Tight coordination with these brokers will give DESC direct leverage over the quantities that contribute to cosmological systematics: the completeness and purity of SN Ia samples, characterization of selection effects, and calibration of host-galaxy priors. Characterizing the selection functions of alert brokers as part of the analysis pipeline will help align transient discovery with the DESC requirements.

Brokers should maintain rigorous provenance tracking for all derived data features, host-galaxy associations, and classification/anomaly scores so that DESC can understand the selection effects of deployed algorithms. In return, collaboration with DESC can provide the alert brokers with benchmark datasets and the targeted science objectives used to validate their infrastructure and foster additional software development. Algorithms developed in the early years of the Rubin LSST can be ported upstream to broker environments after public release, providing additional metadata (e.g., embeddings from a data foundation model or concise, text-based descriptions of a subset of high-priority alerts) and allowing the broader scientific community and all Science Collaborations to benefit from DESC efforts without violating LSST data rights policies.

