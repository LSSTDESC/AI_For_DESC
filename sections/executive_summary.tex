\newpage
\section{Executive Summary}


\paragraph{Guiding Principles for the DESC AI/ML Strategy} We detail below the core driving principles that guide our strategic thinking around developing AI/ML for DESC:
\begin{itemize}
    \item \textbf{Facilitating the delivery of DESC science} through the integration of AI/ML tools, while \textbf{fulfilling the stringent requirements} of precision cosmology.

    \item Building and maintaining a \textbf{sustainable AI/ML ecosystem} required for the collaborative development, validation, and deployment \textbf{of production-grade AI/ML tools}.

    \item Seeking out opportunities to leverage \textbf{DESC’s unique position} as an  international collaboration using community-accessible data, and pursuing extremely demanding scientific objectives, to \textbf{establish DESC as a pioneer in the development of robust AI/ML practices for fundamental physics}.

    \item Integrating AI/ML into the DESC in ways that \textbf{preserve and amplify the human-centric nature of research}, strengthen collaboration quality, and \textbf{keep contributors’ work at the center}.
\end{itemize}


\paragraph{Key Opportunities and Recommendations} 
\begin{itemize}

  \item \textbf{R1: Develop DESC-wide AI/ML best practices.} Extend Publication Policy with an AI/ML checklist, consolidate a small set of supported AI/ML stacks, establish DESC-wide model registry with automated robustness tests, establish a standard of full reproducibility for AI/ML results. 

  \item \textbf{R2: Establish governance for LLMs and agentic systems.} Coordinate DESC-wide activities involving LLM/agents, establish best practices and evaluation/review/red-teaming of pilot studies, include critical discussions of the limits of the technology and its effects on human researchers, involving experts in other domains. DESC should engage with the LSST Data Management team to allow developed agentic AI to interface with data products effectively and reliably.

  % \item \textbf{R3: Champion Differentiable Programming and Hybridization of Physical models with Generative Modeling.} Promote differentiable programming and hybrid physics-ML models that embed cosmological theory and simulators directly into AI architectures.

  \item \textbf{R4: Build Strategic Methodological Partnerships in AI and Computer Science.} Establish long-term collaborations with computer science departments and AI institutes focused on the foundational challenges identified in this document (robust uncertainty quantification, model misspecification and covariate shift, validation of neural posteriors and generative models).

   \item \textbf{Standardization and benchmarking.} Rigorous protocols for benchmarking, validation, domain adaptation, and uncertainty quantification must be established as standard practice for all DESC machine learning projects and outlined within AI/ML policy guidelines. Furthermore, cross-working group deliverables that use machine learning, such as foundation models and simulations, should be supported by comprehensive benchmarks that are solicited from DESC members and represent a broad array of science cases.
\item \textbf{Developing the human-machine interface.} Close connections should be developed between DESC, other LSST science collaborations, in-kind follow-up programmes, alert broker teams, LSST data management and citizen scientists, to facilitate active learning for classification and anomaly detection.
\item \textbf{Compute and workforce needs.} Investment in computational infrastructure and specialized technical expertise must be secured to keep pace with the escalating demand for AI research and deployment within DESC.

  \item \textbf{O1: Lead Rubin-wide Development of Foundation Models.} DESC can play a key role in coordinating various actors, ranging from other Rubin LSST Science Collaborations to a wide range of academic institutions and AI institutes, on the development of frontier AI models for the Rubin Community.

  \item \textbf{O2: Pioneer Agentic AI for Scientific Rigor and Reproducibility.} Develop “DESC research agents” that automate the execution, documentation, and validation of analyses against standardized DESC benchmarks, coupling these systems to clear governance and red-teaming procedures so that agentic workflows enhance transparency, provenance, and trust in DESC results.

  \item \textbf{O3: Establish DESC as a Hub for AI-for-Cosmology Training.} Create a sustained Rubin-focused AI/ML training program (schools, tutorials, and mentoring) built around DESC’s open-source software and challenging datasets, in partnership with leading AI institutes and industry, to train the next generation of researchers in trustworthy, physics-informed AI for cosmology.

 \item \textbf{Strategic leadership in inference.} DESC should capitalize on its leadership position in simulation-based inference (SBI) and broader Bayesian methodologies, with a specific focus on mitigating model misspecification. To sustain this expertise, collaboration meetings and Dark Energy Schools should be leveraged for knowledge transfer between research groups.
\end{itemize}